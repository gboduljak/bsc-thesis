\subsection{Elementary definitions and notation}
\begin{definition}[$\sigma$-algebra]
Let $\Omega$ be a set. Let $\mathcal{F}$ be  a family of subsets of $\Omega$. We say that $\mathcal{F}$ is a $\sigma$-algebra on $\Omega$ if it satisfies all of the following
\begin{axioms}{S}
  \item \label{defn:measure:salg:S1} $\Omega \in \mathcal{F}$;
  \item \label{defn:measure:salg:S2} if $E \in \mathcal{F}$ then $\Omega \setminus E \in \mathcal{F}$;
  \item \label{defn:measure:salg:S3} if $\{ E_n \}_{n=1}^\infty$ where for every $n \in \N$, $E_n \in \mathcal{F}$ then $\bigcup_{n=1}^\infty E_n \in \mathcal{F}$.
\end{axioms}
\end{definition}
\begin{definition}[measurable set]
Let $\Omega$ be a set and $\mathcal{F}$ be a $\sigma$-algebra on $\Omega$. A subset $E \subseteq \Omega$ is said to be $\mathcal{F}$-measurable or simply measurable if $E \in \mathcal{F}$.
\end{definition}
\begin{definition}[measurable space]
Let $\Omega$ be a set and $\mathcal{F}$ be a $\sigma$-algebra on $\Omega$. Then the ordered pair $(\Omega, \mathcal{F})$ is a measurable space.
\end{definition}

\begin{definition}[$\sigma$-algebra generated by the set]
Let $\Omega$ be a set and let $A \subseteq \mathcal{P}(\Omega)$. The $\sigma$-algebra generated by $A$ is the smallest $\sigma$-algebra on $\Omega$ containing $A$.
\end{definition}

\begin{definition}[Borel $\sigma$-algebra]
Let $(X,\Gamma)$ be a topological space. The Borel $\sigma$-algebra of $X$ is the $\sigma$-algebra generated by open sets of $X$, denoted by $\B(X)$.
Hence $\B(X) = \sigma(\Gamma)$.
\end{definition}

\begin{definition}[a measure]
Let $(\Omega, \mathcal{F})$ be a measurable space. The function $\mu : \mathcal{F} \to [0, \infty]$ is a measure if it satisfies all of the following:
\begin{axioms}{M}
  \item \label{defn:measure:salg:M1} $\mu(\emptyset) = 0$;
  \item \label{defn:measure:salg:M2}  if $\{ E_n \}_{n=1}^\infty$ are pairwise disjoint $\mathcal{F}$-measurable sets, $\mu(\bigcup_{n=1}^\infty E_n) = \sum_{n=1}^\infty \mu(E_n)$.
\end{axioms}
\end{definition}
A very useful property of a measure is the notion of continuity.
\begin{lemma}[Continuity of a measure, Proposition 1.2.5 in \cite{cohn_2013_measure}]
Let $(\Omega, \mathcal{F})$ be a measurable space and let $\mu$ be a measure on $(\Omega, \mathcal{F})$. 
\begin{description}
\item If $\{ E_n \}_{n = 1}^{\infty}$ is an increasing sequence of sets in $\mathcal{F}$, then $\mu \left (\bigcup_{n = 1} ^ {\infty} E_n \right) = \lim_{n \to \infty} \mu(E_n)$.
\item  If $\{ E_n \}_{n = 1}^{\infty}$  is a decreasing sequence of sets in $\mathcal{F}$ such that $\mu(E_n)< \infty$ for some $n \in \mathbb{N}$, then $    \mu \left (\bigcap_{n = 1} ^ {\infty} E_n \right) = \lim_{n \to \infty} \mu(E_n)$.
\end{description}
\end{lemma}
\begin{definition}[finite measure]
Let $(\Omega, \mathcal{F}, \mu)$ be a measure space. We say that $\mu$ is a finite measure if for every $E \in \mathcal{F}$, $\mu(E) < \infty$.
\end{definition}
\begin{definition}[$\sigma$-finiteness]
Let $(\Omega, \mathcal{F}, \mu)$ be a measure space. We say that $\mu$ is a $\sigma$- finite measure if there exists a sequence $\{ E_n \}_{n = 1}^{\infty}$ of $\mathcal{F}$-measurable sets with $\mu(E_n) < \infty$ and $\Omega = \bigcup_{n=1}^\infty E_n$.
\end{definition}

\subsection{Construction of a measure and Carathéodory's theorem}
Direct construction of a measure is often tedious and sometimes even impossible process. However, it is often possible to simplify the construction by starting with a somewhat weaker set function - outer measure.
\begin{definition}[outer measure]
Let $\Omega$ be a set and let $\mathcal{P}(\Omega)$ be a powerset of $\Omega$. A function $\mu^{\ast} : \mathcal{P}(\Omega) \to [0, \infty]$ is an outer measure if it satisfies all of the following:
\begin{axioms}{O}
  \item \label{defn:measure:salg:OM1} $\mu^{\ast}(\emptyset) = 0$;
  \item \label{defn:measure:salg:OM2}  if $A \subseteq B \subseteq \Omega$, then $\mu^{\ast} (A) \leq \mu^{\ast}(B)$;
  \item \label{defn:measure:salg:OM3}  if $\{ E_n \}_{n=1}^\infty$ are  $\mathcal{F}$-measurable sets, $\mu^{\ast}(\bigcup_{n=1}^\infty E_n) \leq \sum_{n=1}^\infty \mu^{\ast}(E_n)$.
\end{axioms}
\end{definition}
Using the concept of an outer measure, we can define outer measurable sets.
\begin{definition}[outer measurable sets]
Let $\Omega$ be a set and let $\mu^{\ast}$ be an outer measure on $\mathcal{P}(\Omega)$. A subset $E \subseteq \Omega$ is said to be $\mu^{\ast}$-measurable if for every $A \subseteq \Omega$, $\mu^{\ast}(A) = \mu^{\ast}(A \cap E) + \mu^{\ast}(A \cap (\Omega \setminus E))$.
\end{definition}
\nameref{thm:measure:caratheodory} is a very important tool, guaranteeing existence of a measure induced by the outer measure.
\begin{theorem}[Carathéodory's theorem]
\label{thm:measure:caratheodory}
Let $\Omega$ be a set and let $\mu^{\ast}$ be an outer measure on $\mathcal{P}(\Omega)$. Suppose that $\mathcal{M}_{\mu^{\ast}}$ is a collection of all $\mu^{\ast}$-measurable subsets of $\Omega$. Then $\mathcal{M}_{\mu^{\ast}}$ is a $\sigma$-algebra and the restriction $\mu^{\ast}_{| \mathcal{M}_{\mu^{\ast}}}$ is a measure on $\mathcal{M}_{\mu^{\ast}}$.
\end{theorem}
\begin{proof}
See Theorem 1.3.6 in \cite{cohn_2013_measure}.
\end{proof}
For an application of Carathéodory's theorem and an example of a construction of
a measure, see the proof of \nameref{thm:fcs:rrt-positive}.

Closely related to the construction of a measure is the concept of Dynkin classes, also known as $\lambda$-systems.
\begin{definition}[Dynkin class ($\lambda$-system)]
Let $\Omega$ be a set and let $\mathcal{P}(\Omega)$ be a powerset of $\Omega$. The collection $\Lambda \subseteq \mathcal{P}(\Omega)$ is a Dynkin class or a $\lambda$-system if
\begin{axioms}{L}
  \item \label{defn:measure:lambda:L1}  $\Omega \in \Lambda$;
  \item \label{defn:measure:lambda:L2}  if $A \in \Lambda$ and $B \in \Lambda$ such that $B \subseteq A$, then $A \setminus B \in \Lambda$;
  \item \label{defn:measure:lambda:L3}  if $\{ A_n \}_{n=1}^\infty$ is a collection of disjoint sets in $\Lambda$, then $\bigcup_{n=1}^\infty A_n \in \Lambda$.
\end{axioms}
\end{definition}
\begin{definition}[$\pi$-system]
Let $\Omega$ be a set and let $\mathcal{P}(\Omega)$ be a powerset of $\Omega$. The collection $\Pi \subseteq \mathcal{P}(\Omega)$ is a  a $\pi$-system if it is closed under finite intersections.
\end{definition}

\begin{theorem}[Dynkin's $\lambda-\pi$ theorem]
\label{thm:measure:lambda-pi}
Let $\Omega$ be a set and let $\Pi$ be a $\pi$-system on $\Omega$. Then the $\sigma$-algebra generated by $\Pi$, denoted $\sigma(\Pi)$, coincides with the $\lambda$-system generated by $\Pi$.
\end{theorem}
\begin{proof}
See Theorem 1.6.2 in \cite{cohn_2013_measure}.
\end{proof}

\begin{proposition}[Proposition 1.5.6 in \cite{cohn_2013_measure}]
\label{proposition:measure:regularity_borel_finite}
Let $\mu$ be a finite measure on $(\R^d, \B(\R^n))$. Then $\mu$ is regular. Moreover, each Borel subset A of $\R^d$ satisfies \[
\mu(A) = \sup \{ \mu (K) : K \subseteq A, \text{ K compact} \}.
\]
\end{proposition}

\subsection{Measurable functions and their properties}

\begin{definition}[a measurable function]
Let $(X, \mathcal{A})$ and $(Y, \mathcal{B})$ be measurable spaces. A function $f : X \to Y$ is measurable if for every $B \in \mathcal{B}$, $f^{-1}(B) \in \mathcal{A}$.
\end{definition}

In this thesis, we say that $f : X \to \R$ is measurable if it is measurable with respect to $\B(\R)$.
As one might expect, measurable functions are closed under various algebraic operations.

\begin{proposition}[Proposition 5.7 in \cite{bass2011real}]
\label{thm:measure:measurablefnsalgclosure}
Let $c \in \R$. If $f$ and $g$ are measurable real-valued functions, so are $f + g$, $cf$, $fg$, $\max(f, g)$, $\min(f, g)$.
\end{proposition}

The following decomposition of a measurable function and corresponding characterization of measurability are often applicable.
\begin{proposition}
\label{thm:measure:measurablefnplusminus}
Let $(\Omega, \mathcal{F})$ be a measurable space. For $f : \Omega \to \R$, define
$f^{+} : \Omega \to \R$ by $f^{+} = \max(f, 0)$ and define $f^{-} : \Omega \to \R$ by $f^{-} = \max(-f, 0)$. Then $f = f^{+} - f^{-}$ and $f$ is measurable if and only $f^{+}$ and $f^{-}$ are measurable.
\end{proposition}
\begin{proof}
$f = f^{+} - f^{-}$ follows directly from definition of $f^{+}$ and $f^{-}$ and the rest follows from \nameref{thm:measure:measurablefnsalgclosure}.
\end{proof}
One of the most useful facts about measurable functions is the closure under various limiting operations.

\begin{proposition}[Proposition 2.1.5 in \cite{cohn_2013_measure}]
Let $(\Omega, \mathcal{F})$ be a measurable space, suppose $A \subseteq \Omega$ is $\mathcal{F}$-measurable.  Let $\{ f_n \}_{n=1}^\infty$ be a sequence of $\overline{\R}$-measurable functions on $A$. Then $\sup_{n} f_n, \inf_{n} f_n, \limsup_{n} f_n, \liminf_n f_n$ are measurable.
\end{proposition}

The simplest form of a measurable function is a characteristic function.

\begin{definition}[characteristic function]
Let $\Omega$ be a set. If $E \subseteq \Omega$, we define the characteristic function of $E$, denoted $\chi_E$, by
\[ \chi_E(x) = \begin{cases} 
      1 & x \in E \\
      0 & x \not \in E
   \end{cases}.
\]
\end{definition}

Linear combinations of characteristic functions are simple functions.

\begin{definition}[simple function]
Let $(\Omega, \mathcal{F})$ be a measurable space. A function $\varphi : \Omega \to \mathcal{F}$ is simple if it is of the form $\varphi = \sum_{k=1}^n a_k \chi_{A_k}$ where $\{ A_k \}_{k=1}^n$ are $\mathcal{F}$-measurable sets and $\{ a_k \}_{k=1}^n$ are real numbers. 
\end{definition}

It is interesting to note that measurable functions are pointwise limits of simple functions. This property is often used in proofs.
\begin{proposition}[Proposition 5.14 in \cite{bass2011real}]
Suppose $f$ is a non-negative and measurable function. Then there exists a sequence of non-negative measurable simple functions $\{ \varphi_n \}_{n=1}^\infty$ increasing to $f$. Hence for every $n \in \N$, $0 \leq \varphi_n \leq \varphi_{n + 1}$ and $\lim_{n \to \infty} \varphi_n = f$ pointwise.
\end{proposition}

\subsection{Lebesgue Integration}
Development of the integration theory usually starts by defining the integral of some class of simple functions. In this case, we begin with non-negative simple functions.
\begin{definition}[an integral of a non-negative, simple function]
Let $(\Omega, \mathcal{F}, \mu)$ be a measure space. Suppose that $\varphi$ is a non-negative, simple function of the form
$\varphi = \sum_{k=1}^n a_k \chi_{A_k}$. We define an integral of a $\varphi$ by \[
    \int_{\Omega} \varphi \, d\mu = \sum_{k = 1}^n a_k \mu(A_k).
\]
\end{definition}
\begin{remark}
The proof that the integral of a non-negative, simple function is indeed well-defined is discussed on page 53 in \cite{cohn_2013_measure}.
\end{remark}

Now we generalize the integral to non-negative, measurable functions.

\begin{definition}
Let $(\Omega, \mathcal{F}, \mu)$ be a measure space. Suppose that $f$ is a non-negative, measurable function. 
The integral of $f$ is defined by
\begin{align*}
        \int_{\Omega} f \,d\mu = \sup \left \{ \int_{\Omega} g \,d\mu : 0 \leq g \leq f, \text{ g simple} \right \}.
\end{align*}
\end{definition}

Theorem \ref{thm:measure:measurablefnplusminus} guarantees that every measurable function $f$ can be expressed in the form
$f = f^{+} - f^{-}$, where $f^{+}$ and $f^{-}$ are non-negative and measurable.  We use this decomposition to define the integral of an arbitrary measurable function.

\begin{definition}
Let $(\Omega, \mathcal{F}, \mu)$ be a measure space. Suppose that $f$ is arbitrary, extended real-valued measurable function.
If at least one of $ \int_{\Omega} f^{+} \,d\mu, \int_{\Omega} f^{-} \,d\mu$ is finite, we define integral of $f$ by
\begin{align*}
        \int_{\Omega} f \,d\mu = \int_{\Omega} f^{+} \,d\mu - \int_{\Omega} f^{-} \,d\mu.
\end{align*}
\end{definition}

\begin{definition}
Let $(\Omega, \mathcal{F}, \mu)$ be a measure space. Suppose that $f$ is extended real-valued measurable function. We say that $f$ is integrable if $\int_\Omega |f| \, d\mu < \infty$.
\end{definition}

The following criteria for a function to be zero almost everywhere is very helpful.

\begin{proposition}[Proposition 8.1 in \cite{bass2011real}]
\label{proposition:measure:characterzerointergral_one_dir}
Let $(\Omega, \mathcal{F}, \mu)$ be a measure space. Suppose that $f : \Omega \to [0, \infty]$ is $\mathcal{F}$-measurable,
satisfying $\int_{\Omega} f \, d\mu = 0$. Then $f = 0$ $\mu$-almost everywhere. 
\end{proposition}

\begin{proposition}
\label{proposition:measure:characterzerointergral_two_dir}
Let $(\Omega, \mathcal{F}, \mu)$ be a measure space. Suppose that $f : \Omega \to [0, \infty]$ is $\mathcal{F}$-measurable.
Then $f = 0$ $\mu$-almost everywhere if and only if $\int_{\Omega} f \, d\mu = 0$. 
\end{proposition}
\begin{proof}
By Proposition \ref{proposition:measure:characterzerointergral_one_dir}, it is sufficient to prove that $f = 0$ $\mu$-almost everywhere implies $\int_\Omega f = 0, \, d\mu$.
Suppose $f = 0$ $\mu$-almost everywhere. Then there exists a set $E \subseteq \Omega$ such that $\mu(E) = 0$ and $f > 0$ on $E$ while $f = 0$ on $\Omega \setminus E$. Then $\int_\Omega f \, d\mu = \int_E f \, d\mu + \int_{\Omega \setminus E} f \, d\mu = 0, $
since $f$ is identically zero on $\Omega \setminus E$ and $E$ is a set of measure zero.
\end{proof}

The following inequality is a well-known neat result.

\begin{proposition}[Markov's Inequality]
\label{proposition:measure:ineqn:markov}
Let $(\Omega, \mathcal{F}, \mu)$ be a measure space. Suppose that $f : \Omega \to [0, \infty]$ is $\mathcal{F}$-measurable.
If $\epsilon$ is a positive real number, then \[
    \mu( \{ \omega \in \Omega : f(\omega) \geq \epsilon \}) \leq \frac{1}{\epsilon} \int_\Omega f \, d\mu.
\]
\end{proposition}
\begin{proof}
See Proposition 2.3.10 in \cite{cohn_2013_measure}.
\end{proof}

Proposition \ref{proposition:measure:nonnegative-integral-props} and Proposition \ref{proposition:measure:nonnegative-integral-props} guarantee that the integral satisfies expected properties.

\begin{proposition}[Proposition 2.3.4 in \cite{cohn_2013_measure}]
\label{proposition:measure:nonnegative-integral-props}
Let $(\Omega, \mathcal{F}, \mu)$ be a measure space. Let $f, g : \Omega \to [0, \infty]$ be measurable and suppose that $\alpha \geq 0$. Then 
\begin{enumerate}[noitemsep]
    \item $\int_{\Omega} \alpha f \, d\mu =  \alpha \int_{\Omega} f \, d\mu,$
    \item $\int_{\Omega} (f+g) \, d\mu =  \int_{\Omega} f \, d\mu +  \int_{\Omega} g \, d\mu,$ and
    \item if $f(\omega) \leq g(\omega)$ for every $\omega 
    \in \Omega$, then $\int_\Omega f \, d\mu \leq \int_\Omega g \, d\mu$.
\end{enumerate}
\end{proposition}

\begin{proposition}[Proposition 2.3.6 in \cite{cohn_2013_measure}]
\label{proposition:measure:integral-props}
Let $(\Omega, \mathcal{F}, \mu)$ be a measure space. Let $f, g : \Omega \to [-\infty, \infty]$ be measurable and suppose that $\alpha \in \R$. Then 
\begin{enumerate}[noitemsep]
    \item $\alpha f$ is integrable and $\int_{\Omega} \alpha f \, d\mu =  \alpha \int_{\Omega} f \, d\mu,$
    \item $f + g$ is integrable and $\int_{\Omega} (f+g) \, d\mu =  \int_{\Omega} f \, d\mu +  \int_{\Omega} g \, d\mu,$ and
    \item if $f(\omega) \leq g(\omega)$ for every $\omega 
    \in \Omega$, then $\int_\Omega f \, d\mu \leq \int_\Omega g \, d\mu$.
\end{enumerate}
\end{proposition}

The following three theorems are known as limiting or convergence theorems. Those theorems are fundamental results in Lebesgue integration theory and one of the main reasons for its success.

\begin{theorem}[Monotone Convergence Theorem]
\label{thm:mct}
Let $(\Omega, \mathcal{F}, \mu)$ be a measure space. Suppose that $\{ f_n \}_{n=1}^\infty$ is a sequence of $[0, \infty]$-valued $\mathcal{F}$-measurable functions on $\Omega$ and $f : \Omega \to [0, \infty]$ such that \begin{align*}
    f_n (\omega) &\leq f_{n + 1}(\omega) \text{ for every $n \in \N$ } \text{ and } f (\omega) = \lim_{n \to \infty} f_n(\omega),
\end{align*}
hold at $\mu$-almost every $\omega \in \Omega$. Then \[ 
    \int_{\Omega} f \, d\mu = \lim_{n \to \infty} \int_\Omega f_n \, d\mu.
\]
\end{theorem}
\begin{proof}
See Theorem 2.4.1 and its proof in \cite{cohn_2013_measure}.
\end{proof}
\nameref{thm:fatou} is a very useful corollary of \nameref{thm:mct}.
\begin{theorem}[Fatou's Lemma]
\label{thm:fatou}
Let $(\Omega, \mathcal{F}, \mu)$ be a measure space. Suppose that $\{ f_n \}_{n=1}^\infty$ is a sequence of $[0, \infty]$-valued $\mathcal{F}$-measurable functions on $\Omega$. Then
\begin{align*}
    \int_{\Omega} \liminf_{n \to \infty} f_n \, d\mu \leq  \liminf_{n \to \infty} \int_{\Omega} f_n \, d\mu.
\end{align*}
\end{theorem}
\begin{proof}
See Theorem 2.4.4 and its proof in \cite{cohn_2013_measure}.
\end{proof}
\begin{remark}
Usefulness of \nameref{thm:fatou} follows from the fact it imposes almost no requirements on $f_n$. An example of such an use-case is in the proof of \nameref{thm:lp:rrt}. Using the \nameref{thm:mct} and \nameref{thm:fatou}, it is possible to prove the following very important theorem in Lebesgue Integration.
\end{remark}
\begin{theorem}[Dominated Convergence Theorem]
\label{thm:dct}
Let $(\Omega, \mathcal{F}, \mu)$ be a measure space. Suppose that $g : \Omega \to [0, \infty]$ is an integrable function on $\Omega$. Let $\{ f_n \}_{n=1}^\infty$ be a sequence of $[-\infty, \infty]$-valued $\mathcal{F}$-measurable functions on $\Omega$ such that
\begin{align*}
      |f_n (\omega)| &\leq g(\omega) \text{ for every $n \in \N$}, \text{ and }  \\
      f (\omega) &= \lim_{n \to \infty} f_n(\omega) ,
\end{align*}
hold at $\mu$-almost every $\omega \in \Omega$. Then $f$ and $f_n$ are integrable for each $n \in \N$. Moreover, $\int_\Omega f \, d \mu = \lim_{n \to \infty} \int_\Omega f_n \, d \mu$.
\end{theorem}
\begin{proof}
See Theorem 2.4.5 and its proof in \cite{cohn_2013_measure}.
\end{proof}

The following theorem will help us evaluate Lebesgue integrals when Riemann integrals exist.
\begin{theorem}[Equivalence Riemann - Lebesgue]
A bounded Borel measurable real-valued function $f$ on $[a,b]$ is Riemann integrable if and only if the set of points at which $f$ is discontinuous has Lebesgue measure zero. In that case the Riemann integral of $f$ is equal in value to the Lebesgue integral of $f$.
\end{theorem}
\begin{proof}
See Theorem 9.1 and its proof in \cite{bass2011real}.
\end{proof}
\subsection{Modes of Convergence}
The following three propositions are standard results addressing convergence in measure.
\begin{proposition}[Proposition 3.1.2 in \cite{cohn_2013_measure}]
\label{proposition:measure:convergence:ae_implies_mu}
Let $(\Omega, \mathcal{F}, \mu)$ be a measure space and suppose that $f$ and $\{ f_n \}_{n=1}^\infty$ are real-valued, $\mathcal{F}$-measurable functions on $\Omega$. If $\mu$ is finite and if $f_n \to f$ $\mu$-almost everywhere, then $f_n \to f$ in $\mu$.
\end{proposition}

\begin{proposition}[Proposition 3.1.3 in \cite{cohn_2013_measure}]
\label{proposition:measure:convergence:mu_implies_subseqn_ae}
Let $(\Omega, \mathcal{F}, \mu)$ be a measure space and suppose that $f$ and $\{ f_n \}_{n=1}^\infty$ are real-valued, $\mathcal{F}$-measurable functions on $\Omega$. If $f_n \to f$ in $\mu$, then there exists a subsequence of $\{ f_n \}_{n = 1}^\infty$ converging to $f$ $\mu$-almost everywhere.
\end{proposition}
\begin{proposition}[Proposition 3.1.5 in \cite{cohn_2013_measure}]
\label{proposition:measure:convergence:lone_implies_ae}
Let $(\Omega, \mathcal{F}, \mu)$ be a measure space and suppose that $f$ and $\{ f_n \}_{n=1}^\infty$ belong to $\Lone(\Omega, \mathcal{F}, \mu)$. If $f_n \to f$ in $\norm{\cdot}_1$, then  $f_n \to f$ in measure $\mu$.
\end{proposition}

\nameref{thm:measure:lusin} relates continuous and measurable functions on locally compact Hausdorff space.

\begin{theorem}[Lusin's Theorem, Theorem 7.4.4 in \cite{cohn_2013_measure}]
\label{thm:measure:lusin}
Let $X$ be a locally compact Hausdorff space and let $\mathcal{A}$ be a $\sigma$-algebra that includes $\B(X)$. Let $\mu$ be a regular measure on $(X, \mathcal{A})$ and suppose $f : X \to \R$ is measurable. If $A \in \mathcal{A}$ and satisfies $\mu(A) < \infty$ and if $\epsilon > 0$, then there is a compact set $K \subseteq A$ such that $\mu(A \setminus K) < \epsilon$ and $f_{|K}$ is continuous. Moreover, there is a function $g \in \C(X)$ that agrees with $f$ on $K$.  If $A \neq \emptyset$ and $f$ is bounded on $A$, $g$ can be chosen to satisfy $\sup \{ |g(x)| : x \in X \} \leq \sup \{ |f(x)| : x \in A \}$.
\end{theorem}
\begin{proof}
See Theorem 7.4.4 in \cite{cohn_2013_measure}.
\end{proof}

\subsection{Product Measure and Fubini's Theorem}

\begin{definition}[product $\sigma$-algebra]
Let $(X, \mathcal{A})$ and $(Y, \mathcal{B})$ be two measurable spaces. We define the product $\sigma$-algebra, written $\mathcal{A} \otimes \mathcal{B}$ by $\mathcal{A} \otimes \mathcal{B} = \sigma(\mathcal{A} \times \mathcal{B})$. 
\end{definition}
It is possible to construct the unique measure $\mu \times \nu$ on $\mathcal{A} \otimes \mathcal{B}$ satisfying 
\begin{align*}
    (\mu \times \nu) (A \times B) = \mu(A) \cdot \nu(B), \text{ for every $A \in \mathcal{A}$, $B \in \mathcal{B}$}.
\end{align*}
That measure is called the product measure. The construction of product measure and verification of the claim above is thoroughly discussed in \cite{bass2011real}. The \nameref{thm:measure:fubini} helps us evaluate integrals with respect to the product measure. We will use this result extensively in the section discussing \nameref{section:appendix:fourier}.
\begin{theorem}[Fubini-Tonelli Theorem]
\label{thm:measure:fubini}
Suppose $f : \Omega \times \Gamma \to \R$ is measurable with respect to $\Omega \times \Gamma$. If either $f$ is non-negative or $\int_{\Omega \times \Gamma} |f| \,d (\mu \times \nu) < \infty$ then
\begin{align*}
       \int_{\Omega \times \Gamma} f (x, y) \,d (\mu \times \nu) = \int_{\Omega} \left [ \int_{\Gamma} f(x,y) \, d\nu \right] \, d\mu = \int_{\Gamma} \left [ \int_{\Omega} f(x,y) \, d\mu \right] \, d\nu.
\end{align*}
\end{theorem}
\begin{proof}
See Theorem 11.3 and its proof in \cite{bass2011real}.
\end{proof}
\begin{remark}
\nameref{thm:measure:fubini} also holds if we replace the assumption $\int_{\Omega \times \Gamma} |f| \,d (\mu \times \nu) < \infty$ with $\int_{\Omega} \left [ \int_{\Gamma} |f(x,y)| \, d\nu \right] \, d\mu < \infty$ or $\int_{\Gamma} \left [ \int_{\Omega} |f(x,y)| \, d\mu \right] \, d\nu < \infty$. For the justification, see the discussion following Theorem 11.3 in \cite{bass2011real}.
\end{remark}