\subsection{Signed measures and their decompositions}
This section is devoted to signed measures, which can be seen as a generalization of measures. We will also discuss the relationship between signed measures and measures. We begin with a definition.
\begin{definition}[a signed measure]
Let $(\Omega, \mathcal{F})$ be a measurable space. The function $\nu : \mathcal{F} \to \overline{\R}$ is a signed measure if it satisfies all of the following:
\begin{axioms}{S}
  \item \label{defn:measure:smeasure:S1} $\nu(\emptyset) = 0$;
  \item \label{defn:measure:smeasure:S2} $\nu$ does not attain both $+\infty$ and $-\infty$;
  \item \label{defn:measure:smeasure:S3}  if $\{ E_n \}_{n=1}^\infty$ are pairwise disjoint $\mathcal{F}$-measurable sets, $\nu(\bigcup_{n=1}^\infty E_n) = \sum_{n=1}^\infty \nu(E_n)$.
\end{axioms}
\end{definition}
As one might expect, the continuity results extend from measures.
\begin{lemma}[Continuity of a signed measure]
Let $(\Omega, \mathcal{F})$ be a measurable space and let $\mu$ be a measure on $(\Omega, \mathcal{F})$. 
\begin{description}
\item If $\{ E_n \}_{n = 1}^{\infty}$ is an increasing sequence of sets in $\mathcal{F}$, then $\nu \left (\bigcup_{n = 1} ^ {\infty} E_n \right) = \lim_{n \to \infty} \nu(E_n)$.
\item  If $\{ E_n \}_{n = 1}^{\infty}$  is a decreasing sequence of sets in $\mathcal{F}$ such that $\nu(E_n)$ is finite for some $n \in \mathbb{N}$, then $\nu \left (\bigcap_{n = 1} ^ {\infty} E_n \right) = \lim_{n \to \infty} \nu(E_n)$.
\end{description}
\end{lemma}
\begin{proof}
See Lemma 4.1.2 in \cite{cohn_2013_measure}.
\end{proof}