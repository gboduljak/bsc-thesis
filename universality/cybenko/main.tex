\newpage
\section{Universal approximation of continuous functions via Cybenko's method}
\label{section:universality:continuous:cybenko}

In this section, we will focus on the approximation in the space $\C([0,1]^n)$, where $[0,1]^n$ denotes the $n$-dimensional unit hypercube. It is worth noting that most of the results established in this section generalize to any compact subset of $\R^n$. However, we focus on $[0,1]^n$ for the sake of compatibility with \cite{cybenko_1989_approximation}. We will discuss one of the most famous results regarding the approximation power of single-layer, fully-connected neural networks. The main result of this section is the following theorem, proved by G. Cybenko in the paper \cite{cybenko_1989_approximation} from 1989.
\begin{theorem*}[Cybenko, 1989]
Let $\mathcal{H}_{\sigma}$ denote the family of single-layer fully-connected neural networks with the logistic sigmoid activation function, given by \begin{align*}
\mathcal{H}_{\sigma} = \left \{ \vec{x} \to \sum_{k=1}^{m} \alpha_k \sigma{\left (\langle \vec{w_k}, \vec{x} \rangle + \beta_k \right)} : m \in \N, \alpha_1 \ldots \alpha_m , \beta_1 \ldots \beta_m \in \R, \vec{w_k} \in \R^n \right \}.
\end{align*}
The family $\mathcal{H}_{\sigma}$ is dense in $\C([0,1]^n)$.
\end{theorem*}
\begin{remark}
We will prove a slightly more general version.
\end{remark}

The proof of this theorem introduces a few novel concepts, such as the notion of discriminatory activation function and a generalization of the logistic sigmoid. Apart from those concepts, the proof relies on standard results from the functional analysis, \nameref{thm:funct:hahn-banach} and \nameref{thm:fcs:rrt-bounded}. To establish the stated theorem, we will develop the necessary concepts in order very similar to \cite{cybenko_1989_approximation}. The structure of the argument is outlined below.

\begin{description}[noitemsep]
\item[\nameref{subsection:universality:cybenko:1}] In this subsection, we will introduce the concept of a discriminatory activation function. We will discuss a few examples of such functions and develop a lemma to identify discriminatory activation functions.
\item[\nameref{subsection:universality:cybenko:2}] In this subsection, we will discuss a generalization of the logistic sigmoid. Functions belonging to this generalized family are examples of discriminatory activation functions.
\item[\nameref{subsection:universality:cybenko:3}] In this subsection, we will explore the relationship between density in a normed linear space and its dual space. The analysis will use methods from functional analysis and measure theory.
\item[\nameref{subsection:universality:cybenko:4}] We will state the main theorem and present an elegant proof based on results developed in the previous three subsections. 
\item[\nameref{subsection:universality:cybenko:5}] We will generalize \nameref{thm:universality:cybenko} to $\C([0,1]^n, \R^m)$.
\end{description}

\subsection{Discriminatory activation functions}
\label{subsection:universality:cybenko:1}
An essential part of Cybenko's argument is the notion of discriminatory activation function with respect to a given (signed) measure.

\begin{definition}[discriminatory activation function with respect to a measure]
\label{defn:discrim:discrimactfn}
Let $\mu$ be a finite signed Borel measure on $[0,1]^n$. A function $\sigma : \R \to \R$ is called discriminatory for $\mu$ if
\[
    \int_{[0,1]^n}\sigma(\langle \vec{w}, \vec{x} \rangle + b) \, d \mu(\vec{x}) = 0, \text{ for every } \vec{w} \in \R^n, b \in \R  \implies \mu = 0.
\]
\end{definition}
\begin{remark}
In the definition above, $\sigma$ is not necessarily the logistic sigmoid.
\end{remark}
\begin{definition}[discriminatory activation function]
A function $\sigma : \R \to \R$ is called discriminatory if it is discriminatory for every finite signed Borel measure on $[0,1]^n$.
\end{definition}

At first glance, the meaning of definition of a \nameref{defn:discrim:discrimactfn} is somewhat unclear. Thus, it is worth discussing the intuition behind this definition. By contrapositive, if $\sigma$ is discriminatory for $\mu$ and $\mu$ is nonzero, then there exists at least one configuration of weights $\vec{w} \in \R^n$ and a bias $b \in \R$ such that $\int_{[0,1]^n} \sigma(\langle \vec{w}, \vec{x} \rangle + b) \, d \mu (\vec{x}) \neq 0$. Informally, if $\sigma$ is discriminatory for $\mu$, then $\sigma$ is volumetrically non-destructive when it acts on the affine space $\{ \langle \vec{w}, \vec{x} \rangle + b : \vec{x} \in [0,1]^n \}$. Recall that the affine space $\{ \langle \vec{w}, \vec{x} \rangle + b : \vec{x} \in [0,1]^n \}$ is precisely the range of a single neuron parameterized by weights $\vec{w}$ and a bias $b \in \R$, immediately before the application of the activation function. This section aims to answer the following two questions.
\begin{itemize}[noitemsep]
\item How to prove that an activation function is discriminatory for a given measure?
\item Which practically useful activation functions are discriminatory?
\end{itemize}

To address the first question, we will develop a lemma which will help us prove that a given activation function is discriminatory for a given measure.
To simplify claims of the following results, we shall introduce a bit of notation for hyperplanes and open half spaces of $[0,1]^n$. We define
\begin{align*}
    \Pi_{\vec{w}, b} = \left \{ \vec{x} : \vec{x} \in [0,1]^n, \langle \vec{w}, \vec{x} \rangle + b = 0 \right \}, & \\
    \Pi_{\vec{w}, b}^{+} = \left \{ \vec{x} : \vec{x} \in [0,1]^n, \langle \vec{w}, \vec{x} \rangle + b > 0 \right \}, & \\
    \Pi_{\vec{w}, b}^{-} = \left \{ \vec{x} : \vec{x} \in [0,1]^n, \langle \vec{w}, \vec{x} \rangle + b < 0 \right \}.
\end{align*}

The following lemma will provide us with a method to identify discriminatory activation functions.

\begin{lemma}
\label{lemma:discrim:vanishhyper}
Let $\mu$ be a finite signed Borel measure on $[0,1]^n$. If $\mu$ vanishes on all hyperplanes and open half-spaces of $[0,1]^n$, then $\mu$ is identically zero. More formally, if for every configuration consisting of weights $\vec{w} \in \R^n$ and a bias $b \in \R$, \begin{align*}
    \mu (\Pi_{\vec{w}, b}) &= 0 \text{ and } \mu(\Pi_{\vec{w}, b}^{+}) = 0,
\end{align*}
then $\mu = 0$.
\end{lemma}

\begin{proof-idea*}
The idea is to apply "Lebesgue induction" to the cleverly constructed functional $F : \Linfty(\R) \to \R$ given by \[
    F_{\vec{w}}(h) = \int_{[0,1]^n} h ( \langle \vec{w}, \vec{x} \rangle) \, d \mu( \vec{x}).
\]
We will "Lebesgue-inductively" show that $F_{\vec{w}} = 0$ on $\Linfty(\R)$. Surprisingly, the main difficulty will be proving that $F_{\vec{w}}(\chi_{B}) = 0$ for every Borel set $B \in \B(\R)$. To prove this, we will use \nameref{thm:measure:lambda-pi}. After proving that the functional $F_\vec{w}$ vanishes on indicator functions of Borel sets, we will show that it vanishes on measurable simple functions. By appealing to \nameref{thm:lp:density}, we will be able to generalize the result to $\Linfty(\R)$. Using $F_{\vec{w}}(\sin)$ and $F_{\vec{w}}(\cos)$ we will show that the Fourier transform of the finite signed Borel measure $\mu$, denoted $\widehat{\mu}$, satisfies $\widehat{\mu} = 0$. But, this implies $\mu = 0$. The application of the Fourier transform of $\mu$ demystifies the definition and use of $F_\vec{w}$.
\end{proof-idea*}
\begin{proof}
\setcounter{step}{0}
\begin{step}[$F_\vec{w}$ is a bounded linear functional.]
Fix $\vec{w} \in \R^n$, and define the function $F_\vec{w} : \Linfty(\R) \to \R$ by \[
    F_{\vec{w}}(h) = \int_{[0,1]^n} h ( \langle \vec{w}, \vec{x} \rangle  ) \, d \mu  (\vec{x}).
\]
We claim that $F_\vec{w}$ is a bounded linear functional on $\Linfty(\R)$. Linearity follows from the linearity of an integral. To prove boundedness, suppose $h \in \Linfty(\R)$. By definition of $\Linfty(\R)$, without loss of generality, $h \leq \norm{h}_\infty < \infty$. Then \begin{align}
    \label{ineqn:discrim:vanishdiscrimlemma:1}
     | F_{\vec{w}}(h) | &= \left | \int_{[0,1]^n} h ( \langle \vec{w}, \vec{x} \rangle) \, d \mu  (\vec{x}) \right | \leq  \int_{[0,1]^n} \left |  h ( \langle \vec{w}, \vec{x} \rangle)  \right |  \, d  |\mu| (\vec{x}) \leq \norm{h}_\infty  |\mu| ([0,1]^n).
\end{align}
Since $\mu$ is finite, by \nameref{thm:hahn-jordan}, so is its total variation $|\mu|$. Hence $ |\mu| ([0,1]^n) < \infty$. By \ref{ineqn:discrim:vanishdiscrimlemma:1}, $| F_{\vec{w}}(h) | < \infty$.
\end{step}
\begin{step}[$F_\vec{w}$ vanishes on indicators of Borel sets in $\R$.]
We begin by proving $F_\vec{w}$ vanishes on indicator functions of intervals.
Consider the indicator function $\chi_{[b, \infty)}$, for $b \in \R$. We have
\begin{subequations}\label{eqn:discrim:vanishdiscrimlemma:vanish_chi_b_infty_closed}
\begin{align*}
    F_{\vec{w}}(\chi_{[b, \infty)}) &= \int_{[0,1]^n} \chi_{[b, \infty)}( \langle \vec{w}, \vec{x} \rangle  ) \, d \mu  (\vec{x}) \\
                                &= \mu (\left \{ \vec{x} \in [0,1]^n : b \leq \langle \vec{w}, \vec{x} \rangle < \infty  \right \}) \\
                                &= \mu (\left \{ \vec{x} \in [0,1]^n : 0 \leq \langle \vec{w}, \vec{x} \rangle -b\right \}) \\
                                &= \mu(\Pi_{\vec{w}, -b}) +  \mu(\Pi_{\vec{w}, -b}^{+}) = 0.
     \tag{\ref{eqn:discrim:vanishdiscrimlemma:vanish_chi_b_infty_closed}} 
\end{align*}
\end{subequations}
To establish \ref{eqn:discrim:vanishdiscrimlemma:vanish_chi_b_infty_closed}, we applied the assumption that $\mu(\Pi_{\vec{w}, -b}) = \mu(\Pi_{\vec{w}, -b}^{+}) = 0$. Similarly, \begin{align}
    \label{eqn:discrim:vanishdiscrimlemma:vanish_chi_b_infty_open}
    F_{\vec{w}}(\chi_{(b, \infty)}) = \mu(\Pi_{\vec{w}, -b}^{+}) = 0.
\end{align}
For every $a, b \in \R$ such that $a < b$, $\chi_{(a,b)} = \chi_{(a, \infty)} - \chi_{[b, \infty)}$. By linearity of $F_\vec{w}$ and by \ref{eqn:discrim:vanishdiscrimlemma:vanish_chi_b_infty_closed} and  \ref{eqn:discrim:vanishdiscrimlemma:vanish_chi_b_infty_open},
\begin{align}
    \label{eqn:discrim:vanishdiscrimlemma:openints}
    F_{\vec{w}}(\chi_{(a,b)}) = F_{\vec{w}}(\chi_{(a, \infty)} - \chi_{[b, \infty)}) = F_{\vec{w}}(\chi_{(a, \infty)}) - F_{\vec{w}}(\chi_{[b, \infty)}) = 0.
\end{align} We claim that $F_{\vec{w}}(\chi_B) = 0$, for every Borel set $B \subseteq \R$.
To show that $F_\vec{w}$ vanishes on the indicator function of every Borel set, we will appeal to \nameref{thm:measure:lambda-pi}.
Define collections $\Pi$ and $\Lambda$ by 
\begin{align*}
    \Pi = \left \{ (a,b) : - \infty \leq a \leq b \leq \infty \right \} \text{ and } \Lambda = \left \{ A : A \in \B(\R) \text{ and } F_{\vec{w}}(\chi_{A}) = 0 \right \}.
\end{align*}
Since a finite intersection of open intervals is again an open, possibly degenerate interval, $\Pi$ is a $\pi$-system. We will show that $\Lambda$ is a $\lambda$-system. By \ref{eqn:discrim:vanishdiscrimlemma:openints}, $\Lambda$ contains $\Pi$. Clearly, $\R \in \Lambda$. Suppose that $A, B \in \Lambda$ where $B \subseteq A$. Then $\chi_{A \setminus B} = \chi_A - \chi_B$ so $F_{\vec{w}}(\chi_{A \setminus B}) = F_{\vec{w}}(\chi_A - \chi_B)  =F_{\vec{w}}(\chi_A ) - F_{\vec{w}}(\chi_B) = 0$, since $A, B \in \Lambda$. Thus $A \setminus B \in \Lambda$. Suppose that $\{ B_n \}_{n =1}^\infty$ is a collection of disjoint sets in $\Lambda$. We will show that $B = \bigcup_{n=1}^\infty B_n \in \Lambda$. Clearly, $\chi_B = \sum_{k=1}^\infty \chi_{B_k}$ so $\sum_{k=1}^m \chi_{B_k} \uparrow \chi_B$, as $m \to \infty$. Then
\begin{align*}
    F_{\vec{w}}(\chi_{B}) &=  \int_{[0,1]^n} \chi_B ( \langle \vec{w}, \vec{x} \rangle  ) \, d \mu  (\vec{x}) =  \int_{[0,1]^n} \sum_{k=1}^\infty \chi_{B_k} ( \langle \vec{w}, \vec{x} \rangle  ) \, d \mu  (\vec{x}) \\
               &=   \int_{[0,1]^n} \lim_{m \to \infty} \sum_{k=1}^m \chi_{B_k} ( \langle \vec{w}, \vec{x} \rangle  ) \, d \mu  (\vec{x})  \\
               &=  \lim_{m \to \infty}\int_{[0,1]^n} \sum_{k=1}^m \chi_{B_k} ( \langle \vec{w}, \vec{x} \rangle  ) \, d \mu  (\vec{x})  \text{ by \nameref{thm:mct}}  \\
               &= \lim_{m \to \infty} \sum_{k=1}^m \int_{[0,1]^n} \chi_{B_k}  ( \langle \vec{w}, \vec{x} \rangle  ) \, d \mu  (\vec{x}) \\ 
               &=  \lim_{m \to \infty} \sum_{k=1}^m F_{\vec{w}}(\chi_{B_k}) = 0.
\end{align*}
It is worth justifying the application of \nameref{thm:mct}. In the original form, \nameref{thm:mct} applies only to measures. In the calculation above, we are integrating with respect to a signed measure. However, by \nameref{thm:hahn-jordan}, $\mu$ admits decomposition $\mu = \mu^{+} - \mu^{-}$, where $\mu^{+}$ and $\mu^{-}$ are measures. We apply \nameref{thm:mct} to integrals with respect to $\mu^{+}$ and $\mu^{-}$. By definition of the integral with respect to $\mu$, we obtain the desired conclusion.
Hence $\Lambda$ is indeed a $\lambda$-system. By \nameref{thm:measure:lambda-pi}, $\B(\R) = \sigma(\Pi) = \lambda(\Pi) \subseteq \Lambda$.
Since $\Lambda \subseteq \B(\R)$, $\B(\R) = \Lambda$, as desired.
\end{step}
\begin{step}[$F_\vec{w}$ vanishes on measurable simple functions]
Suppose that $\varphi$ is a $\B(\R)$-measurable simple function. Without loss of generality, $\varphi = \sum_{k=1}^m \alpha_k \chi_{A_k} $, where $A_k$ are disjoint $\B(\R)$-measurable sets. By linearity of $F_\vec{w}$ and Step 2,
\begin{align}
    \label{ineqn:discrim:vanishdiscrimlemma:simple}
    F_{\vec{w}}(\varphi) = F \left (\sum_{k=1}^m \alpha_k \chi_{A_k} \right ) = \sum_{k=1}^m F_{\vec{w}}(\alpha_k \chi_{A_k}) =  \sum_{k=1}^m  \alpha_k F_{\vec{w}}( \chi_{A_k}) = 0.
\end{align}
\end{step}
\begin{step}[$F_\vec{w}$ vanishes on $\Linfty(\R)$]
Let $f \in \Linfty(\R)$. By \nameref{thm:lp:density}, there exists a sequence of $\B(\R)$-measurable simple functions $\{ \varphi_n \}_{n=1}^\infty$ converging to $f$ in $\norm{\cdot}_\infty$. For every $m \in \N$, $f - \varphi_m \in \Linfty(\R)$. Without loss of generality, $| f - \varphi_m | \leq \norm{f - \varphi_m}_\infty$. Then
\begin{subequations}\label{ineqn:discrim:vanishdiscrimlemma:vanish_goal}
\begin{align*}
    \left | F_{\vec{w}}(f) - F_{\vec{w}}(\varphi_m) \right | &\leq \left | \int_{[0,1]^n} (f - \varphi_m) ( \langle \vec{w}, \vec{x} \rangle) \, d \mu  (\vec{x}) \right | & \\
    & \leq \int_{[0,1]^n} |  (f - \varphi_m) ( \langle \vec{w}, \vec{x} \rangle) | \, d |\mu| ( \vec{x}) &\\
    & \leq \norm{f - \varphi_m}_\infty |\mu| ([0,1]^n).
    \tag{\ref{ineqn:discrim:vanishdiscrimlemma:vanish_goal}} 
\end{align*}
\end{subequations}
Since $\lim_{m \to \infty}  \norm{f - \varphi_m}_\infty = 0$ and $ |\mu| ([0,1]^n) < \infty$, by \ref{ineqn:discrim:vanishdiscrimlemma:vanish_goal}, $\left | F_{\vec{w}}(f) - F_{\vec{w}}(\varphi_m) \right | \to 0$, as $m \to \infty$.
Combining   $\left | F_{\vec{w}}(f) - F_{\vec{w}}(\varphi_m) \right | \to 0$ as $m \to \infty$ with  \ref{ineqn:discrim:vanishdiscrimlemma:simple} gives
\begin{align}
   \label{ineqn:discrim:vanishdiscrimlemma:linfty}
   F_{\vec{w}}(f) = \lim_{m \to \infty} F_{\vec{w}}(\varphi_m) = 0.
\end{align}
Since $f$ was arbitrary, $F_\vec{w}$ vanishes on $\Linfty(\R)$.
\end{step}
\begin{step}[$\mu$ is identically zero.]
We will compute the Fourier transform of $\mu$.
Since $\cos$ and $\sin$ are bounded and measurable, $\cos, \sin \in \Linfty(\R)$. By \ref{ineqn:discrim:vanishdiscrimlemma:linfty}, we have that $F_\vec{w}(\cos) = F_\vec{w}(\sin) = 0$. This implies  
\begin{align*}
    \widehat{\mu}(\vec{w}) &= \int_{[0,1]^n} e^{i \langle \vec{w}, \vec{x} \rangle } \, d\mu(\vec{x}) & \\
                           &=  \int_{[0,1]^n} \cos (\langle \vec{w}, \vec{x} \rangle) \, d\mu(\vec{x}) + i \int_{[0,1]^n} \sin (\langle \vec{w}, \vec{x} \rangle) \, d\mu(\vec{x})  &\\
                           &= F_\vec{w}(\cos) + i F_\vec{w}(\sin) &\\
                           &= 0.
\end{align*}
By Corollary \ref{corr:fourier:uniqmeasure}, $\mu = 0$.
\end{step}
\end{proof}


We are ready to address the second question.

\subsection{Sigmoidal activation functions}
\label{subsection:universality:cybenko:2}
\begin{definition}[sigmoidal activation function]
A function $\sigma : \R \to \R$ is called sigmoidal if \[
    \lim_{x \to \infty} \sigma(x) = 1 \text{ and } \lim_{x \to -\infty} \sigma(x) = 0.
\]
\end{definition}

\begin{proposition}[Bounded measurable sigmoidal functions are discriminatory]
\label{prop:discrim:boundedmeasdiscrim}
Let $\mu$ be a finite signed Borel measure on $[0,1]^n$. Suppose that $\sigma : \R \to \R$ is bounded, Borel measurable and sigmoidal function. Then $\sigma$ is discriminatory for $\mu$. 
\end{proposition}
\begin{proof-idea*}
We will reduce the problem to the application of Lemma \ref{lemma:discrim:vanishhyper}.
\end{proof-idea*}
\begin{proof}
Suppose that $\sigma$ satisfies \begin{align}
    \label{eqn:discrim:bdsigm:sigma}
    \int_{[0,1]^n}\sigma(\langle \vec{w}, \vec{x} \rangle + b) \, d \mu  (\vec{x}) = 0, \text{ for every } \vec{w} \in \R^n, b \in \R.
\end{align}
We need to show that $\mu = 0$. We aim to apply Lemma \ref{lemma:discrim:vanishhyper}. For fixed $\lambda, a \in \R$, define $\sigma_{\lambda, a} : [0,1]^n \to \R$ by \[
    \sigma_{\lambda, a} (\vec{x}) = \sigma \left ( \lambda (\langle \vec{w}, \vec{x} \rangle + b) + a \right ) = \sigma \left (\langle \lambda \vec{w}, \vec{x} \rangle + b\lambda + a \right ).
\]
By \ref{eqn:discrim:bdsigm:sigma}, 
 \begin{align}
    \label{eqn:discrim:bdsigm:sigmalambda}
    \int_{[0,1]^n}\sigma_{\lambda, a} (\vec{x}) \, d \mu  (\vec{x}) = 0.
\end{align}
Define $\gamma : [0,1]^n \to \R$ by $\gamma(\vec{x}) = \lim_{\lambda \to \infty} \sigma_{\lambda, a} (\vec{x})$. Observe that  \begin{equation}
    \label{eqn:discrim:bdsigm:gamma}
    \gamma(\vec{x}) =
    \begin{cases}
      1         & \text{ if } \langle \vec{w}, \vec{x} \rangle + b > 0 \\
      \sigma(a) & \text{ if } \langle \vec{w}, \vec{x} \rangle + b = 0 \\
      0         & \text{ if } \langle \vec{w}, \vec{x} \rangle + b < 0
    \end{cases}.
  \end{equation}
  
By \ref{eqn:discrim:bdsigm:gamma}, \begin{align*}
    \int_{[0,1]^n} \gamma(\vec{x}) d \mu  (\vec{x}) &= \int_{\Pi_{\vec{w}, b}^{+}} \gamma(\vec{x}) d \mu  (\vec{x})  + \int_{\Pi_{\vec{w}, b}} \gamma(\vec{x}) d \mu  (\vec{x})  + \int_{\Pi_{\vec{w}, b}^{-}} \gamma(\vec{x}) d \mu  (\vec{x})  & \\
                                                    &=\int_{\Pi_{\vec{w}, b}^{+}} 1 d \mu  (\vec{x})  + \int_{\Pi_{\vec{w}, b}} \sigma(a) d \mu  (\vec{x})  + \int_{\Pi_{\vec{w}, b}^{-}} 0 d \mu  (\vec{x})  & \\
                                                   &= \mu(\Pi_{\vec{w}, b}^{+}) + \sigma(a) \cdot \mu (\Pi_{\vec{w}, b}).
\end{align*}
Taking $\lim_{a \to \infty}$ and applying the fact $\sigma$ is sigmoidal gives
\begin{align}
      \label{eqn:discrim:bdsigm:vanish_1}
      \int_{[0,1]^n} \gamma(\vec{x}) d \mu  (\vec{x}) = \mu(\Pi_{\vec{w}, b}^{+}) + \lim_{a \to \infty} \sigma(a) \cdot \mu (\Pi_{\vec{w}, b}) = \mu(\Pi_{\vec{w}, b}^{+}) + \mu (\Pi_{\vec{w}, b}).
\end{align}
Taking $\lim_{a \to -\infty}$ and applying the fact $\sigma$ is sigmoidal gives
\begin{align}
      \label{eqn:discrim:bdsigm:vanish_2}
      \int_{[0,1]^n} \gamma(\vec{x}) d \mu  (\vec{x}) = \mu(\Pi_{\vec{w}, b}^{+}) + \lim_{a \to -\infty} \sigma(a) \cdot \mu (\Pi_{\vec{w}, b}) = \mu(\Pi_{\vec{w}, b}^{+}).
\end{align}
Equating \ref{eqn:discrim:bdsigm:vanish_1} and  \ref{eqn:discrim:bdsigm:vanish_2} gives
\begin{align}
     \mu(\Pi_{\vec{w}, b}^{+}) + \mu (\Pi_{\vec{w}, b}) = \mu(\Pi_{\vec{w}, b}^{+}) \implies \mu (\Pi_{\vec{w}, b}) = 0.
\end{align}
By \ref{eqn:discrim:bdsigm:vanish_2}, to prove $\mu(\Pi_{\vec{w}, b}^{+}) = 0$, it is equivalent to prove $\int_{[0,1]^n} \gamma(\vec{x}) d \mu  (\vec{x}) = 0$. We will appeal to the \nameref{thm:dct}.
By \nameref{thm:hahn-jordan}, $\mu$ can be decomposed as $\mu = \mu^{+} - \mu^{-}$, where $\mu^{+}$ and $\mu^{-}$ are measures and at least one of them is finite. Since $\mu$ is finite, so are both $\mu^{+}$ and $\mu^{-}$. \newpage
Since $\sigma$ is bounded, for every $\lambda \in \R$, for every $a \in \R$, for every $\vec{x} \in [0,1]^n$,  \[ 
    |\sigma_{\lambda, a} (\vec{x})| \leq \norm{\sigma}_\infty.
\]
Since $\mu^{+}$ and $\mu^{-}$ are finite, $\vec{x} \to \norm{\sigma}_\infty$ is $\mu^{+}$ and $\mu^{-}$ integrable. Now \begin{align}
    \label{eqn:discrim:bdsigm:vanish_int_main}
    \int_{[0,1]^n} \gamma(\vec{x}) d \mu  (\vec{x}) &=  \int_{[0,1]^n} \gamma(\vec{x}) d \mu^{+} (\vec{x}) -  \int_{[0,1]^n} \gamma(\vec{x}) d \mu^{-} (\vec{x}).
\end{align}
By \ref{eqn:discrim:bdsigm:gamma} and \nameref{thm:dct}, we have \begin{align}
     \label{eqn:discrim:bdsigm:vanish_int_1}
     \int_{[0,1]^n} \gamma(\vec{x}) d \mu^{+} (\vec{x}) = \lim_{\lambda \to \infty}  \int_{[0,1]^n} \sigma_{\lambda, a} (\vec{x}) d \mu^{+} (\vec{x}).
\end{align}
Similarly, we have \begin{align}
     \label{eqn:discrim:bdsigm:vanish_int_2}
     \int_{[0,1]^n} \gamma(\vec{x}) d \mu^{-} (\vec{x}) = \lim_{\lambda \to \infty}  \int_{[0,1]^n} \sigma_{\lambda, a} (\vec{x}) d \mu^{-} (\vec{x}).
\end{align}
Applying \ref{eqn:discrim:bdsigm:vanish_int_1} and \ref{eqn:discrim:bdsigm:vanish_int_2} to \ref{eqn:discrim:bdsigm:vanish_int_main} gives
\begin{align*}
    \int_{[0,1]^n} \gamma(\vec{x}) d \mu  (\vec{x}) &= \lim_{\lambda \to \infty}  \int_{[0,1]^n} \sigma_{\lambda, a} (\vec{x}) d \mu^{+} (\vec{x}) - \lim_{\lambda \to \infty}  \int_{[0,1]^n} \sigma_{\lambda, a} (\vec{x}) d \mu^{-} (\vec{x}) & \\ 
                                                   &= \lim_{\lambda \to \infty} \left (\int_{[0,1]^n} \sigma_{\lambda, a} (\vec{x}) d \mu^{+} (\vec{x})  -\int_{[0,1]^n} \sigma_{\lambda, a} (\vec{x}) d \mu^{-} (\vec{x}) \right ) & \\ 
                                                   &= \lim_{\lambda \to \infty}\left (\int_{[0,1]^n} \sigma_{\lambda, a} (\vec{x}) d \mu  (\vec{x}) \right) &\\
                                                   &= 0,
\end{align*}
since by \ref{eqn:discrim:bdsigm:sigmalambda}, $\int_{[0,1]^n} \sigma_{\lambda, a} (\vec{x}) d \mu  (\vec{x}) = 0. $ We have shown that $\mu$ vanishes on $\Pi_{\vec{w}, b}^{+}$ and $\Pi_{\vec{w}, b}$. By Lemma \ref{lemma:discrim:vanishhyper}, $\mu$ is identically zero.
We conclude $\sigma$ is discriminatory for $\mu$, as claimed.
\end{proof}

\begin{proposition}[Continuous sigmoidal functions are discriminatory]
\label{prop:discrim:contsigmoidalarediscrim}
Let $\mu$ be a finite signed Borel measure on $[0,1]^n$. Suppose that $\sigma : \R \to \R$ is a continuous sigmoidal function. Then $\sigma$ is discriminatory for $\mu$. 
\end{proposition}
\begin{proof}
Since $\sigma$ is continuous, it is Borel measurable. Since it is continuous and sigmoidal, $\sigma$ is bounded. The result follows directly from
Proposition \ref{prop:discrim:boundedmeasdiscrim}.
\end{proof}

\begin{corollary}
Consequently, any continuous sigmoidal function is discriminatory.
\end{corollary}
\begin{proof}
Apply Proposition \ref{prop:discrim:contsigmoidalarediscrim} to arbitrary $\mu$.
\end{proof}
\begin{corollary}
Logistic sigmoid is discriminatory.
\end{corollary}
\begin{corollary}
Heaviside step function is discriminatory.
\end{corollary}
It turns out that a wide variety of bounded functions are discriminatory.
\begin{theorem}[Theorem 5 in \cite{hornik_1991_approximation}]
Whenever $\sigma$ is bounded and nonconstant, it is discriminatory.
\end{theorem}
\begin{proof}
See Theorem 5 in \cite{hornik_1991_approximation}.
\end{proof}
\subsection{Density and the dual space}
\label{subsection:universality:cybenko:3}

In this section, we will focus on a real normed linear space $\mathcal{X}$ and its \textbf{non-dense} linear subspace $\mathcal{U}$. The density is understood with respect to the topology induced by the norm $\norm{\cdot}$ on $\mathcal{X}$. It turns out that the fact $\mathcal{U}$ is not dense implies the existence of a non-trivial bounded linear functional which vanishes on $\mathcal{U}$.
\begin{lemma}[Separation functional lemma]
\label{lemma:univ:sepfunclemma}
Let $\mathcal{U}$ be a \textbf{non-dense} linear subspace of a real normed linear space $\mathcal{X}$.
Then there exists a bounded linear functional $L$ on $\mathcal{X}$ such that $L \neq 0$ on $\mathcal{X}$ and $L_{| \mathcal{U}} = 0$.
\end{lemma}
\begin{proof-idea*}
To prove the lemma, we will give an explicit construction of the desired functional. We will begin with a subspace of $\mathcal{X}$, denoted by $\mathcal{T}$ where the construction of a candidate functional is more evident and the verification of required properties is relatively easy. The extension of the construction to the entire space $\mathcal{X}$ will follow from the \nameref{thm:funct:hahn-banach}. 
\begin{theorem*}[Hahn-Banach Theorem]
Let $X$ be a real vector space, with a sublinear functional $\rho$ defined on $X$.
Suppose that $W$ is a linear subspace of $X$ and $f_W$ a linear functional on $W$ satisfying
\begin{equation}
    f_W(w) \leq \rho(w), w \in W.
\end{equation}
Then $f_W$ has an extension $f$ on $X$ such that 
\begin{equation}
    f(x) \leq \rho(x), x \in X.
\end{equation}
\end{theorem*}
\begin{proof}
See \nameref{thm:funct:hahn-banach} in Appendix.
\end{proof}
\end{proof-idea*}

\begin{proof}
Since $\mathcal{U}$ is not dense in $\mathcal{X}$, there exists $\vec{x_0} \in \mathcal{X}$ and there exists $\delta > 0$ such that \begin{equation}
    \label{ineqn:univ:seplemma:assumption}
    \norm{\vec{x_0} - \vec{u}} \geq \delta, \text{ for every } \vec{u} \in \mathcal{U}.
\end{equation}
\setcounter{step}{0}
\begin{step}[Construction of a suitable linear subspace $\mathcal{T}$]
To define the desired functional, we restrict our attention to a subset $\mathcal{T} \subseteq \mathcal{X}$
defined by \begin{equation*}
    \mathcal{T} = \left \{ \vec{u} + \lambda \vec{x_0} : \lambda \in \R, \vec{u} \in \mathcal{U} \right \}.
\end{equation*}
We claim that $\mathcal{T}$ is a linear subspace of $\mathcal{X}$. We begin by proving that every element in $\mathcal{T}$ has unique representation. To prove this, suppose that for $\vec{t} \in \mathcal{T}$, we have two representations, \begin{equation}
    \label{eqn:univ:seplemma:eqn1}
    \vec{t} = \vec{u} + \alpha \vec{x_0} = \vec{v} + \beta \vec{x_0}, 
\end{equation}
where $\vec{u}, \vec{v} \in \mathcal{U}$ and $\alpha, \beta \in \R$. Rearranging \ref{eqn:univ:seplemma:eqn1} gives \begin{align}
    \label{eqn:univ:seplemma:rearranged}
    \vec{u} - \vec{v} &= (\beta - \alpha) \vec{x_0}.
\end{align}
By \ref{ineqn:univ:seplemma:assumption}, $\vec{x_0} \not \in \mathcal{U}$. Since $\vec{u} - \vec{v} \in \mathcal{U}$, the only possible solution to \ref{eqn:univ:seplemma:rearranged} is $\vec{u} - \vec{v} = \vec{0} = (\beta - \alpha) \vec{x_0}$. This forces $\vec{u} = \vec{v}$ and $\alpha = \beta$. Thus, the representation of elements in $\mathcal{T}$ is indeed unique. Since $\mathcal{U}$ is linear, $\vec{0} \in \mathcal{U}$ and we may choose $\lambda = 0$ to deduce $\vec{0} \in \mathcal{T}$. Since $\mathcal{U}$ is closed under addition and scalar multiplication, so is $\mathcal{T}$.
\end{step} 
\begin{step}[Construction of a desired functional on $\mathcal{T}$]
Now define $L : \mathcal{T} \to \R$ by \begin{equation*}
    L (\vec{t}) = L (\vec{u} + \lambda \vec{x_0}) = \lambda \delta.
\end{equation*}
Since the representation of elements in $\mathcal{T}$ is unique, $L$ is well defined. We claim that $L$ is a bounded linear functional on $\mathcal{T}$. Let $\alpha \in \R$ and $\vec{t_1}, \vec{t_2} \in \mathcal{T}$ and suppose that $\vec{t_1} = \vec{u_1} + \lambda_1 \vec{x_0}, \vec{t_2} = \vec{u_2} + \lambda_2 \vec{x_0} \in \mathcal{T}$,
for $\vec{u_1}, \vec{u_2} \in \mathcal{U}, \lambda_1, \lambda_2 \in \R$. Now \begin{align*}
    L (\vec{t_1} + \vec{t_2}) &= L (\vec{u_1} + \vec{u_2} + (\lambda_1 + \lambda_2) \vec{x_0}) & \\
                              &=  (\lambda_1 + \lambda_2) \delta = \lambda_1 \delta + \lambda_2 \delta &\\
                              & = L(\vec{t_1}) + L(\vec{t_2}), & \\
    L( \alpha \vec{t_1}) &= L (\alpha \vec{u_1} + \alpha \lambda_1 \vec{x_0})  & \\
                         &= \alpha \lambda_1 \delta   & \\
                         &= \alpha L(\vec{t_1}).
\end{align*}
Hence, $L$ is linear on $\mathcal{T}$. We will show that for every $\vec{t} \in \mathcal{T}$, $L(\vec{t}) \leq \norm{\vec{t}}$. Write $\vec{t} = \vec{u} + \lambda \vec{x_0}$. If $\lambda = 0$ then $L(\vec{t}) = 0 \leq \norm{\vec{t}}$. Now suppose $\lambda \neq 0$. Notice that if $\vec{u} \in \mathcal{U}$, then $\frac{\vec{u}}{\lambda} 
\in \mathcal{U}$. Since $\vec{x_0} + \frac{\vec{u}}{\lambda} \in \mathcal{T}$, by \ref{ineqn:univ:seplemma:assumption},  \begin{align}
    \label{ineqn:univ:seplemma:ineqn2}
    \norm{\vec{x_0} + \frac{\vec{u}}{\lambda}} \geq \delta > 0.
\end{align}
By \ref{ineqn:univ:seplemma:ineqn2}, $\frac{\delta}{\norm{\vec{x_0} + \frac{\vec{u}}{\lambda} }} \leq 1$. Since $\frac{1}{\lambda} (\lambda \vec{x_0} + \vec{u}) = \vec{x_0} + \frac{1}{\lambda} \vec{u}$, we have \begin{equation}
    \label{ineqn:univ:seplemma:ineqn1}
    | \lambda | \delta \leq \norm{\vec{u} + \lambda \vec{x_0}}.
\end{equation}
By \ref{ineqn:univ:seplemma:ineqn1}, \begin{align*}
    L(\vec{t}) = L (\vec{u} + \lambda \vec{x_0}) = \lambda \delta \leq |\lambda| \delta \leq \norm{\vec{u} + \lambda \vec{x_0}} = \norm{\vec{t}}.
\end{align*}
\end{step}
\begin{step}[Extension to $\mathcal{X}$]
By \nameref{thm:funct:hahn-banach} applied with the norm $p(\vec{x}) = \norm{\vec{x}}$, the constructed linear functional $L$ can be extended to a linear functional $\widetilde{L} : \mathcal{X} \to \R$ such that $\widetilde{L} (\vec{x}) \leq \norm{\vec{x}}$, for every $\vec{x} \in \mathcal{X}$. This implies $\norm{\widetilde{L}} \leq 1$ so $\widetilde{L}$ is bounded. By definition of $L$ and the fact $\widetilde{L}_{| \mathcal{T}} = L$,
\begin{align*}
    \widetilde{L} (\vec{u})  &= L (\vec{u} + 0 \cdot \vec{x_0}) = 0 \cdot \delta = 0, & \\
    \widetilde{L}(\vec{x_0}) = L (\vec{x_0}) &= L (\vec{0} + 1 \cdot \vec{x_0}) = 1 \cdot \delta > 0.
\end{align*}
This implies $\widetilde{L}_{| \mathcal{U}} = 0$ and $\widetilde{L} \neq 0$ on $\mathcal{X}$.
\end{step}
\end{proof}
The \nameref{lemma:univ:sepfunclemma} is a quite general result that guarantees the existence of a separation functional vanishing on a non-dense linear subspace of some normed linear space. Unfortunately, this result does not reveal any structure of the desired functional.
\newpage 
However, in the context of a normed linear space $\mathcal{C}([0,1]^n)$, there is a natural correspondence between bounded linear functionals on  $\mathcal{C}([0,1]^n)$ and finite signed Borel measures on $[0,1]^n$. This is a consequence of a result known as the \nameref{thm:fcs:rrt-bounded}, stated below.
\begin{theorem*}[Riesz Representation Theorem for bounded linear functionals on $\C(X)$]
Let $X$ be a compact metric space and $I$ be a bounded linear functional on $\C(X)$. Then there exists the unique finite signed regular measure $\mu$ on $\B(X)$ such that
\begin{equation}
    I (f) = \int_{X} f \,d\mu, \text{    for every $f \in \C(X)$}.
\end{equation}
\end{theorem*}
\begin{proof}
See \nameref{thm:fcs:rrt-bounded}.
\end{proof}
We will apply \nameref{thm:fcs:rrt-bounded} to reveal the structure of a functional given by \nameref{lemma:univ:sepfunclemma}.
\begin{lemma}
\label{lemma:contr}
Let $\mathcal{U}$ be a \textbf{non-dense} linear subspace of a normed linear space $\mathcal{C}([0,1]^n)$.
Then there exists the unique finite signed regular measure $\mu$ on $\B([0,1]^n)$ such that \begin{align*}
    \int_{[0,1]^n} h \, d \mu  = 0, \text{ for every $h \in \mathcal{U}$,}
\end{align*}
but $\mu \neq 0$.
\end{lemma}
\begin{proof}
By \nameref{lemma:univ:sepfunclemma} applied to $\mathcal{X} = \mathcal{C}([0,1]^n)$ equipped with the sup-norm, there exists a bounded linear functional on $\mathcal{C}([0,1]^n)$, denoted by $L : \mathcal{C}([0,1]^n) \to \R$, such that $L \neq 0$ on $\mathcal{C}([0,1]^n)$ and $L_{| \mathcal{U}} = 0$. By \nameref{thm:fcs:rrt-bounded}, there exists the unique finite signed regular measure $\mu$ on $\B([0,1]^n)$ such that
\begin{align}
    \label{eqn:univ:sepmeasure:ineqn1}
    L (f) = \int_{[0,1]^n} f \, d\mu \text{, for every $f \in\mathcal{C}([0,1]^n)$.}
\end{align}
Combining \ref{eqn:univ:sepmeasure:ineqn1} and the fact $L_{| \mathcal{U}} = 0$ gives
\begin{align*}
    L (h) = 0 =  \int_{[0,1]^n} h \, d\mu, \text{ for every } h \in \mathcal{U}.
\end{align*}
Since $L \neq 0$ on $\mathcal{C}([0,1]^n)$, by \ref{eqn:univ:sepmeasure:ineqn1}, $\mu \neq 0$.
\end{proof}

\newpage
\subsection{The Universal Approximation Theorem for $\C([0,1]^n)$ }
\label{subsection:universality:cybenko:4}
\begin{theorem}[The Universal Approximation Theorem for continuous functions]
\label{thm:universality:continuousdiscrim}
Let $\mathcal{H}_{\sigma}$ denote the family of single-layer, fully-connected neural networks with any continuous discriminatory activation function, given by \begin{align*}
\mathcal{H}_{\sigma} = \left \{ \vec{x} \to \sum_{k=1}^{m} \alpha_k \sigma{\left (\langle \vec{w_k}, \vec{x} \rangle + \beta_k \right)} : m \in \N, \alpha_1 \ldots \alpha_m, \beta_1 \ldots \beta_m \in \R, \vec{w_k} \in \R^n \right \}.
\end{align*}
The family $\mathcal{H}_{\sigma}$ is dense in $\C([0,1]^n)$.
\end{theorem}
\begin{proof-idea*}
We will argue by contradiction and apply Lemma \ref{lemma:contr}.
\end{proof-idea*}
\begin{proof}
Since $\sigma$ is continuous, the family $\mathcal{H}_{\sigma}$ is a linear subspace of $C([0,1]^n)$. To prove that the family $\mathcal{H}_{\sigma}$ is dense in $\C([0,1]^n)$, we will argue by contradiction. Suppose that $\mathcal{H}_{\sigma}$ is not dense. By Lemma \ref{lemma:contr}, there exists the unique finite signed regular measure $\mu$ on $\B([0,1]^n)$ such that for every $ h \in \mathcal{\mathcal{H}_{\sigma}}$, \begin{align}
    \label{eqn:discrim:uap-general:measure}
    \int_{[0,1]^n} h \, d\mu = 0, \text{ but } \mu \neq 0.
\end{align}
By linearity of the integral and definition of the family $\mathcal{H}_{\sigma}$, \ref{eqn:discrim:uap-general:measure} is equivalent to
\begin{align}
    \label{eqn:discrim:uap-general:measure2}
     \sum_{k=1}^{N} \alpha_k \int_{[0,1]^n} \sigma{\left (\langle \vec{w_k}, \vec{x} \rangle + \beta_k \right)}  \, d\mu(\vec{x}) = 0, \forall N \in \N, \vec{w_k} \in \R^n, \alpha_k, \beta_k \in \R.
\end{align}
Let $\vec{w} \in \R^n$, $b \in \R$ be arbitrary.
By \ref{eqn:discrim:uap-general:measure2}, \begin{align}
    \label{eqn:discrim:uap-general:measure3}
     \int_{[0,1]^n} \sigma{\left (\langle \vec{w}, \vec{x} \rangle + b \right)}  \, d\mu(\vec{x})  = 0.
\end{align}
Since $\sigma$ is discriminatory for $\mu$ and \ref{eqn:discrim:uap-general:measure3} holds for arbitrary configuration of weights $\vec{w}$ and a bias $b$, $\mu$ is identically zero. However, by \ref{eqn:discrim:uap-general:measure}, $\mu \neq 0$. This is a contradiction.
We conclude $\mathcal{H}_{\sigma}$ is indeed dense, as required.
\end{proof}
As a corollary of Theorem \ref{thm:universality:continuousdiscrim}, we present the original \nameref{thm:universality:cybenko}.
\begin{theorem}[Cybenko's Universal Approximation Theorem, \cite{cybenko_1989_approximation}]
\label{thm:universality:cybenko}
Let $\mathcal{H}_{\sigma}$ denote the family of single-layer, fully-connected neural networks with any continuous sigmoidal activation function, given by \begin{align*}
\mathcal{H}_{\sigma} = \left \{ \vec{x} \to \sum_{k=1}^{m} \alpha_k \sigma{\left (\langle \vec{w_k}, \vec{x} \rangle + \beta_k \right)} : m \in \N, \alpha_1 \ldots \alpha_m, \beta_1 \ldots \beta_m \in \R, \vec{w_k} \in \R^n \right \}.
\end{align*}
The family $\mathcal{H}_{\sigma}$ is dense in $\C([0,1]^n)$.
\end{theorem}
\begin{proof}
By Proposition \ref{prop:discrim:contsigmoidalarediscrim}, any continuous sigmoidal function is discriminatory. The result follows from \nameref{thm:universality:continuousdiscrim}.
\end{proof}
\pagebreak
\subsection{The Universal Approximation Theorem for $\C([0,1]^n, \R^m)$}
\label{subsection:universality:cybenko:5}

In this subsection, we will focus on the metric space $\C([0,1]^n, \R^m)$ equipped with the sup-norm distance $\delta_\infty$.
We begin with a generalization of single-layer, real-valued, fully-connected neural networks - $\mathcal{H}_\sigma$.
\begin{definition}
Let $\mathcal{H}_{\sigma}$ denote the family of single-layer, real-valued, fully-connected neural networks with any continuous sigmoidal activation function, given by \begin{align*}
\mathcal{H}_{\sigma} = \left \{ \vec{x} \to \sum_{k=1}^{N} \alpha_k \sigma{\left (\langle \vec{w_k}, \vec{x} \rangle + \beta_k \right)} : N \in \N, \alpha_1 \ldots \alpha_m, \beta_1 \ldots \beta_m \in \R, \vec{w_k} \in \R^n \right \}.
\end{align*}
We define the family $\mathcal{H}_{\sigma}^{m}$ by \begin{align*}
\mathcal{H}_{\sigma}^m = \left \{ \vec{x} \to \begin{bmatrix}
           h_\sigma^{(1)}(\vec{x}) \\
           h_\sigma^{(2)}(\vec{x}) \\
           \vdots \\
           h_\sigma^{(m)}(\vec{x})
         \end{bmatrix} : h_\sigma^{(k)} \in \mathcal{H}_\sigma, \text{ for every } 1 \leq k \leq m \right \} .
\end{align*} 
\end{definition}
We are ready to state and prove the main theorem of this subsection.
\begin{theorem}[The Universal Approximation Theorem for continuous, vector-valued functions on compact sets]
\label{thm:universality:cybenko:vector-valued:uap}
$\mathcal{H}_{\sigma}^{m}$ is dense in $\C([0,1]^n, \R^m)$.
\end{theorem}
\begin{proof}
Let $\vec{f} \in \C([0,1]^n, \R^m)$. Then $f(\vec{x}) = \begin{bmatrix} f^{(1)}(\vec{x}) & f^{(2)}(\vec{x}) & \ldots & f^{(m)}(\vec{x}) \end{bmatrix}^{\top}$, where for every $k \in \{ 1, 2, \ldots, m \}$, $f^{(k)} \in \C([0,1]^n)$. Let $\epsilon > 0$. By \nameref{thm:universality:cybenko}, there exists a finite family $\{  h_\sigma^{(k)} \}_{k=1}^m$ such that $h_\sigma^{(k)} \in \mathcal{H}_{\sigma}$ and \begin{align}
    \label{eqn:thm:universality:cybenko:vector-valued:uap:1}
    \delta_\infty(f^{(k)}, h_\sigma^{(k)}) = \sup_{\vec{x} \in [0,1]^n} | f^{(k)} (\vec{x}) - h_\sigma^{(k)}(\vec{x})  | < \frac{\epsilon}{\sqrt{m}}, \text{ for every $1 \leq k \leq m$.}
\end{align}
Set $\vec{h}(\vec{x}) = \begin{bmatrix}  h_\sigma^{(1)}(\vec{x}) & h_\sigma^{(2)}(\vec{x}) & \ldots & h_\sigma^{(m)}(\vec{x}) \end{bmatrix}^{\top}$. Clearly, $\vec{h} \in \mathcal{H}_{\sigma}^m$. We will show that $\delta_\infty(\vec{f}, \vec{h})  \leq \epsilon$. For every $\vec{x} \in [0,1]^n$,
\begin{subequations}\label{ineqn:thm:universality:cybenko:vector-valued:uap:2}
\begin{align*}
    \norm{f(\vec{x}) - h(\vec{x})}_2 &= \left(\sum_{k=1}^m | f_k(\vec{x}) - h_k(\vec{x})| ^ 2  \right )^{\frac{1}{2}} & \\ 
                   &\leq \left ( \sum_{k=1}^m \delta_\infty(f^{(k)}, h_\sigma^{(k)})^2 \right) ^{\frac{1}{2}} & \\
                   &< \left( \sum_{k=1}^m  \left (\frac{\epsilon}{\sqrt{m}} \right)^2 \right )^{\frac{1}{2}} & \text{by \ref{eqn:thm:universality:cybenko:vector-valued:uap:1}}  \\
                   &\leq \epsilon.
     \tag{\ref{ineqn:thm:universality:cybenko:vector-valued:uap:2}}
\end{align*}
\end{subequations}
By \ref{ineqn:thm:universality:cybenko:vector-valued:uap:2}, $\delta_\infty(\vec{f}, \vec{h}) \leq \epsilon$. 
\end{proof}
\newpage