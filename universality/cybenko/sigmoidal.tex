\subsection{Sigmoidal activation functions}
\label{subsection:universality:cybenko:2}
\begin{definition}[sigmoidal activation function]
A function $\sigma : \R \to \R$ is called sigmoidal if \[
    \lim_{x \to \infty} \sigma(x) = 1 \text{ and } \lim_{x \to -\infty} \sigma(x) = 0.
\]
\end{definition}

\begin{proposition}[Bounded measurable sigmoidal functions are discriminatory]
\label{prop:discrim:boundedmeasdiscrim}
Let $\mu$ be a finite signed Borel measure on $[0,1]^n$. Suppose that $\sigma : \R \to \R$ is bounded, Borel measurable and sigmoidal function. Then $\sigma$ is discriminatory for $\mu$. 
\end{proposition}
\begin{proof-idea*}
We will reduce the problem to the application of Lemma \ref{lemma:discrim:vanishhyper}.
\end{proof-idea*}
\begin{proof}
Suppose that $\sigma$ satisfies \begin{align}
    \label{eqn:discrim:bdsigm:sigma}
    \int_{[0,1]^n}\sigma(\langle \vec{w}, \vec{x} \rangle + b) \, d \mu  (\vec{x}) = 0, \text{ for every } \vec{w} \in \R^n, b \in \R.
\end{align}
We need to show that $\mu = 0$. We aim to apply Lemma \ref{lemma:discrim:vanishhyper}. For fixed $\lambda, a \in \R$, define $\sigma_{\lambda, a} : [0,1]^n \to \R$ by \[
    \sigma_{\lambda, a} (\vec{x}) = \sigma \left ( \lambda (\langle \vec{w}, \vec{x} \rangle + b) + a \right ) = \sigma \left (\langle \lambda \vec{w}, \vec{x} \rangle + b\lambda + a \right ).
\]
By \ref{eqn:discrim:bdsigm:sigma}, 
 \begin{align}
    \label{eqn:discrim:bdsigm:sigmalambda}
    \int_{[0,1]^n}\sigma_{\lambda, a} (\vec{x}) \, d \mu  (\vec{x}) = 0.
\end{align}
Define $\gamma : [0,1]^n \to \R$ by $\gamma(\vec{x}) = \lim_{\lambda \to \infty} \sigma_{\lambda, a} (\vec{x})$. Observe that  \begin{equation}
    \label{eqn:discrim:bdsigm:gamma}
    \gamma(\vec{x}) =
    \begin{cases}
      1         & \text{ if } \langle \vec{w}, \vec{x} \rangle + b > 0 \\
      \sigma(a) & \text{ if } \langle \vec{w}, \vec{x} \rangle + b = 0 \\
      0         & \text{ if } \langle \vec{w}, \vec{x} \rangle + b < 0
    \end{cases}.
  \end{equation}
  
By \ref{eqn:discrim:bdsigm:gamma}, \begin{align*}
    \int_{[0,1]^n} \gamma(\vec{x}) d \mu  (\vec{x}) &= \int_{\Pi_{\vec{w}, b}^{+}} \gamma(\vec{x}) d \mu  (\vec{x})  + \int_{\Pi_{\vec{w}, b}} \gamma(\vec{x}) d \mu  (\vec{x})  + \int_{\Pi_{\vec{w}, b}^{-}} \gamma(\vec{x}) d \mu  (\vec{x})  & \\
                                                    &=\int_{\Pi_{\vec{w}, b}^{+}} 1 d \mu  (\vec{x})  + \int_{\Pi_{\vec{w}, b}} \sigma(a) d \mu  (\vec{x})  + \int_{\Pi_{\vec{w}, b}^{-}} 0 d \mu  (\vec{x})  & \\
                                                   &= \mu(\Pi_{\vec{w}, b}^{+}) + \sigma(a) \cdot \mu (\Pi_{\vec{w}, b}).
\end{align*}
Taking $\lim_{a \to \infty}$ and applying the fact $\sigma$ is sigmoidal gives
\begin{align}
      \label{eqn:discrim:bdsigm:vanish_1}
      \int_{[0,1]^n} \gamma(\vec{x}) d \mu  (\vec{x}) = \mu(\Pi_{\vec{w}, b}^{+}) + \lim_{a \to \infty} \sigma(a) \cdot \mu (\Pi_{\vec{w}, b}) = \mu(\Pi_{\vec{w}, b}^{+}) + \mu (\Pi_{\vec{w}, b}).
\end{align}
Taking $\lim_{a \to -\infty}$ and applying the fact $\sigma$ is sigmoidal gives
\begin{align}
      \label{eqn:discrim:bdsigm:vanish_2}
      \int_{[0,1]^n} \gamma(\vec{x}) d \mu  (\vec{x}) = \mu(\Pi_{\vec{w}, b}^{+}) + \lim_{a \to -\infty} \sigma(a) \cdot \mu (\Pi_{\vec{w}, b}) = \mu(\Pi_{\vec{w}, b}^{+}).
\end{align}
Equating \ref{eqn:discrim:bdsigm:vanish_1} and  \ref{eqn:discrim:bdsigm:vanish_2} gives
\begin{align}
     \mu(\Pi_{\vec{w}, b}^{+}) + \mu (\Pi_{\vec{w}, b}) = \mu(\Pi_{\vec{w}, b}^{+}) \implies \mu (\Pi_{\vec{w}, b}) = 0.
\end{align}
By \ref{eqn:discrim:bdsigm:vanish_2}, to prove $\mu(\Pi_{\vec{w}, b}^{+}) = 0$, it is equivalent to prove $\int_{[0,1]^n} \gamma(\vec{x}) d \mu  (\vec{x}) = 0$. We will appeal to the \nameref{thm:dct}.
By \nameref{thm:hahn-jordan}, $\mu$ can be decomposed as $\mu = \mu^{+} - \mu^{-}$, where $\mu^{+}$ and $\mu^{-}$ are measures and at least one of them is finite. Since $\mu$ is finite, so are both $\mu^{+}$ and $\mu^{-}$. \newpage
Since $\sigma$ is bounded, for every $\lambda \in \R$, for every $a \in \R$, for every $\vec{x} \in [0,1]^n$,  \[ 
    |\sigma_{\lambda, a} (\vec{x})| \leq \norm{\sigma}_\infty.
\]
Since $\mu^{+}$ and $\mu^{-}$ are finite, $\vec{x} \to \norm{\sigma}_\infty$ is $\mu^{+}$ and $\mu^{-}$ integrable. Now \begin{align}
    \label{eqn:discrim:bdsigm:vanish_int_main}
    \int_{[0,1]^n} \gamma(\vec{x}) d \mu  (\vec{x}) &=  \int_{[0,1]^n} \gamma(\vec{x}) d \mu^{+} (\vec{x}) -  \int_{[0,1]^n} \gamma(\vec{x}) d \mu^{-} (\vec{x}).
\end{align}
By \ref{eqn:discrim:bdsigm:gamma} and \nameref{thm:dct}, we have \begin{align}
     \label{eqn:discrim:bdsigm:vanish_int_1}
     \int_{[0,1]^n} \gamma(\vec{x}) d \mu^{+} (\vec{x}) = \lim_{\lambda \to \infty}  \int_{[0,1]^n} \sigma_{\lambda, a} (\vec{x}) d \mu^{+} (\vec{x}).
\end{align}
Similarly, we have \begin{align}
     \label{eqn:discrim:bdsigm:vanish_int_2}
     \int_{[0,1]^n} \gamma(\vec{x}) d \mu^{-} (\vec{x}) = \lim_{\lambda \to \infty}  \int_{[0,1]^n} \sigma_{\lambda, a} (\vec{x}) d \mu^{-} (\vec{x}).
\end{align}
Applying \ref{eqn:discrim:bdsigm:vanish_int_1} and \ref{eqn:discrim:bdsigm:vanish_int_2} to \ref{eqn:discrim:bdsigm:vanish_int_main} gives
\begin{align*}
    \int_{[0,1]^n} \gamma(\vec{x}) d \mu  (\vec{x}) &= \lim_{\lambda \to \infty}  \int_{[0,1]^n} \sigma_{\lambda, a} (\vec{x}) d \mu^{+} (\vec{x}) - \lim_{\lambda \to \infty}  \int_{[0,1]^n} \sigma_{\lambda, a} (\vec{x}) d \mu^{-} (\vec{x}) & \\ 
                                                   &= \lim_{\lambda \to \infty} \left (\int_{[0,1]^n} \sigma_{\lambda, a} (\vec{x}) d \mu^{+} (\vec{x})  -\int_{[0,1]^n} \sigma_{\lambda, a} (\vec{x}) d \mu^{-} (\vec{x}) \right ) & \\ 
                                                   &= \lim_{\lambda \to \infty}\left (\int_{[0,1]^n} \sigma_{\lambda, a} (\vec{x}) d \mu  (\vec{x}) \right) &\\
                                                   &= 0,
\end{align*}
since by \ref{eqn:discrim:bdsigm:sigmalambda}, $\int_{[0,1]^n} \sigma_{\lambda, a} (\vec{x}) d \mu  (\vec{x}) = 0. $ We have shown that $\mu$ vanishes on $\Pi_{\vec{w}, b}^{+}$ and $\Pi_{\vec{w}, b}$. By Lemma \ref{lemma:discrim:vanishhyper}, $\mu$ is identically zero.
We conclude $\sigma$ is discriminatory for $\mu$, as claimed.
\end{proof}

\begin{proposition}[Continuous sigmoidal functions are discriminatory]
\label{prop:discrim:contsigmoidalarediscrim}
Let $\mu$ be a finite signed Borel measure on $[0,1]^n$. Suppose that $\sigma : \R \to \R$ is a continuous sigmoidal function. Then $\sigma$ is discriminatory for $\mu$. 
\end{proposition}
\begin{proof}
Since $\sigma$ is continuous, it is Borel measurable. Since it is continuous and sigmoidal, $\sigma$ is bounded. The result follows directly from
Proposition \ref{prop:discrim:boundedmeasdiscrim}.
\end{proof}

\begin{corollary}
Consequently, any continuous sigmoidal function is discriminatory.
\end{corollary}
\begin{proof}
Apply Proposition \ref{prop:discrim:contsigmoidalarediscrim} to arbitrary $\mu$.
\end{proof}
\begin{corollary}
Logistic sigmoid is discriminatory.
\end{corollary}
\begin{corollary}
Heaviside step function is discriminatory.
\end{corollary}
It turns out that a wide variety of bounded functions are discriminatory.
\begin{theorem}[Theorem 5 in \cite{hornik_1991_approximation}]
Whenever $\sigma$ is bounded and nonconstant, it is discriminatory.
\end{theorem}
\begin{proof}
See Theorem 5 in \cite{hornik_1991_approximation}.
\end{proof}