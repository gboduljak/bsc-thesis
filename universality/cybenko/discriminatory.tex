\subsection{Discriminatory activation functions}
\label{subsection:universality:cybenko:1}
An essential part of Cybenko's argument is the notion of discriminatory activation function with respect to a given (signed) measure.

\begin{definition}[discriminatory activation function with respect to a measure]
\label{defn:discrim:discrimactfn}
Let $\mu$ be a finite signed Borel measure on $[0,1]^n$. A function $\sigma : \R \to \R$ is called discriminatory for $\mu$ if
\[
    \int_{[0,1]^n}\sigma(\langle \vec{w}, \vec{x} \rangle + b) \, d \mu(\vec{x}) = 0, \text{ for every } \vec{w} \in \R^n, b \in \R  \implies \mu = 0.
\]
\end{definition}
\begin{remark}
In the definition above, $\sigma$ is not necessarily the logistic sigmoid.
\end{remark}
\begin{definition}[discriminatory activation function]
A function $\sigma : \R \to \R$ is called discriminatory if it is discriminatory for every finite signed Borel measure on $[0,1]^n$.
\end{definition}

At first glance, the meaning of definition of a \nameref{defn:discrim:discrimactfn} is somewhat unclear. Thus, it is worth discussing the intuition behind this definition. By contrapositive, if $\sigma$ is discriminatory for $\mu$ and $\mu$ is nonzero, then there exists at least one configuration of weights $\vec{w} \in \R^n$ and a bias $b \in \R$ such that $\int_{[0,1]^n} \sigma(\langle \vec{w}, \vec{x} \rangle + b) \, d \mu (\vec{x}) \neq 0$. Informally, if $\sigma$ is discriminatory for $\mu$, then $\sigma$ is volumetrically non-destructive when it acts on the affine space $\{ \langle \vec{w}, \vec{x} \rangle + b : \vec{x} \in [0,1]^n \}$. Recall that the affine space $\{ \langle \vec{w}, \vec{x} \rangle + b : \vec{x} \in [0,1]^n \}$ is precisely the range of a single neuron parameterized by weights $\vec{w}$ and a bias $b \in \R$, immediately before the application of the activation function. This section aims to answer the following two questions.
\begin{itemize}[noitemsep]
\item How to prove that an activation function is discriminatory for a given measure?
\item Which practically useful activation functions are discriminatory?
\end{itemize}

To address the first question, we will develop a lemma which will help us prove that a given activation function is discriminatory for a given measure.
To simplify claims of the following results, we shall introduce a bit of notation for hyperplanes and open half spaces of $[0,1]^n$. We define
\begin{align*}
    \Pi_{\vec{w}, b} = \left \{ \vec{x} : \vec{x} \in [0,1]^n, \langle \vec{w}, \vec{x} \rangle + b = 0 \right \}, & \\
    \Pi_{\vec{w}, b}^{+} = \left \{ \vec{x} : \vec{x} \in [0,1]^n, \langle \vec{w}, \vec{x} \rangle + b > 0 \right \}, & \\
    \Pi_{\vec{w}, b}^{-} = \left \{ \vec{x} : \vec{x} \in [0,1]^n, \langle \vec{w}, \vec{x} \rangle + b < 0 \right \}.
\end{align*}

The following lemma will provide us with a method to identify discriminatory activation functions.

\begin{lemma}
\label{lemma:discrim:vanishhyper}
Let $\mu$ be a finite signed Borel measure on $[0,1]^n$. If $\mu$ vanishes on all hyperplanes and open half-spaces of $[0,1]^n$, then $\mu$ is identically zero. More formally, if for every configuration consisting of weights $\vec{w} \in \R^n$ and a bias $b \in \R$, \begin{align*}
    \mu (\Pi_{\vec{w}, b}) &= 0 \text{ and } \mu(\Pi_{\vec{w}, b}^{+}) = 0,
\end{align*}
then $\mu = 0$.
\end{lemma}

\begin{proof-idea*}
The idea is to apply "Lebesgue induction" to the cleverly constructed functional $F : \Linfty(\R) \to \R$ given by \[
    F_{\vec{w}}(h) = \int_{[0,1]^n} h ( \langle \vec{w}, \vec{x} \rangle) \, d \mu( \vec{x}).
\]
We will "Lebesgue-inductively" show that $F_{\vec{w}} = 0$ on $\Linfty(\R)$. Surprisingly, the main difficulty will be proving that $F_{\vec{w}}(\chi_{B}) = 0$ for every Borel set $B \in \B(\R)$. To prove this, we will use \nameref{thm:measure:lambda-pi}. After proving that the functional $F_\vec{w}$ vanishes on indicator functions of Borel sets, we will show that it vanishes on measurable simple functions. By appealing to \nameref{thm:lp:density}, we will be able to generalize the result to $\Linfty(\R)$. Using $F_{\vec{w}}(\sin)$ and $F_{\vec{w}}(\cos)$ we will show that the Fourier transform of the finite signed Borel measure $\mu$, denoted $\widehat{\mu}$, satisfies $\widehat{\mu} = 0$. But, this implies $\mu = 0$. The application of the Fourier transform of $\mu$ demystifies the definition and use of $F_\vec{w}$.
\end{proof-idea*}
\begin{proof}
\setcounter{step}{0}
\begin{step}[$F_\vec{w}$ is a bounded linear functional.]
Fix $\vec{w} \in \R^n$, and define the function $F_\vec{w} : \Linfty(\R) \to \R$ by \[
    F_{\vec{w}}(h) = \int_{[0,1]^n} h ( \langle \vec{w}, \vec{x} \rangle  ) \, d \mu  (\vec{x}).
\]
We claim that $F_\vec{w}$ is a bounded linear functional on $\Linfty(\R)$. Linearity follows from the linearity of an integral. To prove boundedness, suppose $h \in \Linfty(\R)$. By definition of $\Linfty(\R)$, without loss of generality, $h \leq \norm{h}_\infty < \infty$. Then \begin{align}
    \label{ineqn:discrim:vanishdiscrimlemma:1}
     | F_{\vec{w}}(h) | &= \left | \int_{[0,1]^n} h ( \langle \vec{w}, \vec{x} \rangle) \, d \mu  (\vec{x}) \right | \leq  \int_{[0,1]^n} \left |  h ( \langle \vec{w}, \vec{x} \rangle)  \right |  \, d  |\mu| (\vec{x}) \leq \norm{h}_\infty  |\mu| ([0,1]^n).
\end{align}
Since $\mu$ is finite, by \nameref{thm:hahn-jordan}, so is its total variation $|\mu|$. Hence $ |\mu| ([0,1]^n) < \infty$. By \ref{ineqn:discrim:vanishdiscrimlemma:1}, $| F_{\vec{w}}(h) | < \infty$.
\end{step}
\begin{step}[$F_\vec{w}$ vanishes on indicators of Borel sets in $\R$.]
We begin by proving $F_\vec{w}$ vanishes on indicator functions of intervals.
Consider the indicator function $\chi_{[b, \infty)}$, for $b \in \R$. We have
\begin{subequations}\label{eqn:discrim:vanishdiscrimlemma:vanish_chi_b_infty_closed}
\begin{align*}
    F_{\vec{w}}(\chi_{[b, \infty)}) &= \int_{[0,1]^n} \chi_{[b, \infty)}( \langle \vec{w}, \vec{x} \rangle  ) \, d \mu  (\vec{x}) \\
                                &= \mu (\left \{ \vec{x} \in [0,1]^n : b \leq \langle \vec{w}, \vec{x} \rangle < \infty  \right \}) \\
                                &= \mu (\left \{ \vec{x} \in [0,1]^n : 0 \leq \langle \vec{w}, \vec{x} \rangle -b\right \}) \\
                                &= \mu(\Pi_{\vec{w}, -b}) +  \mu(\Pi_{\vec{w}, -b}^{+}) = 0.
     \tag{\ref{eqn:discrim:vanishdiscrimlemma:vanish_chi_b_infty_closed}} 
\end{align*}
\end{subequations}
To establish \ref{eqn:discrim:vanishdiscrimlemma:vanish_chi_b_infty_closed}, we applied the assumption that $\mu(\Pi_{\vec{w}, -b}) = \mu(\Pi_{\vec{w}, -b}^{+}) = 0$. Similarly, \begin{align}
    \label{eqn:discrim:vanishdiscrimlemma:vanish_chi_b_infty_open}
    F_{\vec{w}}(\chi_{(b, \infty)}) = \mu(\Pi_{\vec{w}, -b}^{+}) = 0.
\end{align}
For every $a, b \in \R$ such that $a < b$, $\chi_{(a,b)} = \chi_{(a, \infty)} - \chi_{[b, \infty)}$. By linearity of $F_\vec{w}$ and by \ref{eqn:discrim:vanishdiscrimlemma:vanish_chi_b_infty_closed} and  \ref{eqn:discrim:vanishdiscrimlemma:vanish_chi_b_infty_open},
\begin{align}
    \label{eqn:discrim:vanishdiscrimlemma:openints}
    F_{\vec{w}}(\chi_{(a,b)}) = F_{\vec{w}}(\chi_{(a, \infty)} - \chi_{[b, \infty)}) = F_{\vec{w}}(\chi_{(a, \infty)}) - F_{\vec{w}}(\chi_{[b, \infty)}) = 0.
\end{align} We claim that $F_{\vec{w}}(\chi_B) = 0$, for every Borel set $B \subseteq \R$.
To show that $F_\vec{w}$ vanishes on the indicator function of every Borel set, we will appeal to \nameref{thm:measure:lambda-pi}.
Define collections $\Pi$ and $\Lambda$ by 
\begin{align*}
    \Pi = \left \{ (a,b) : - \infty \leq a \leq b \leq \infty \right \} \text{ and } \Lambda = \left \{ A : A \in \B(\R) \text{ and } F_{\vec{w}}(\chi_{A}) = 0 \right \}.
\end{align*}
Since a finite intersection of open intervals is again an open, possibly degenerate interval, $\Pi$ is a $\pi$-system. We will show that $\Lambda$ is a $\lambda$-system. By \ref{eqn:discrim:vanishdiscrimlemma:openints}, $\Lambda$ contains $\Pi$. Clearly, $\R \in \Lambda$. Suppose that $A, B \in \Lambda$ where $B \subseteq A$. Then $\chi_{A \setminus B} = \chi_A - \chi_B$ so $F_{\vec{w}}(\chi_{A \setminus B}) = F_{\vec{w}}(\chi_A - \chi_B)  =F_{\vec{w}}(\chi_A ) - F_{\vec{w}}(\chi_B) = 0$, since $A, B \in \Lambda$. Thus $A \setminus B \in \Lambda$. Suppose that $\{ B_n \}_{n =1}^\infty$ is a collection of disjoint sets in $\Lambda$. We will show that $B = \bigcup_{n=1}^\infty B_n \in \Lambda$. Clearly, $\chi_B = \sum_{k=1}^\infty \chi_{B_k}$ so $\sum_{k=1}^m \chi_{B_k} \uparrow \chi_B$, as $m \to \infty$. Then
\begin{align*}
    F_{\vec{w}}(\chi_{B}) &=  \int_{[0,1]^n} \chi_B ( \langle \vec{w}, \vec{x} \rangle  ) \, d \mu  (\vec{x}) =  \int_{[0,1]^n} \sum_{k=1}^\infty \chi_{B_k} ( \langle \vec{w}, \vec{x} \rangle  ) \, d \mu  (\vec{x}) \\
               &=   \int_{[0,1]^n} \lim_{m \to \infty} \sum_{k=1}^m \chi_{B_k} ( \langle \vec{w}, \vec{x} \rangle  ) \, d \mu  (\vec{x})  \\
               &=  \lim_{m \to \infty}\int_{[0,1]^n} \sum_{k=1}^m \chi_{B_k} ( \langle \vec{w}, \vec{x} \rangle  ) \, d \mu  (\vec{x})  \text{ by \nameref{thm:mct}}  \\
               &= \lim_{m \to \infty} \sum_{k=1}^m \int_{[0,1]^n} \chi_{B_k}  ( \langle \vec{w}, \vec{x} \rangle  ) \, d \mu  (\vec{x}) \\ 
               &=  \lim_{m \to \infty} \sum_{k=1}^m F_{\vec{w}}(\chi_{B_k}) = 0.
\end{align*}
It is worth justifying the application of \nameref{thm:mct}. In the original form, \nameref{thm:mct} applies only to measures. In the calculation above, we are integrating with respect to a signed measure. However, by \nameref{thm:hahn-jordan}, $\mu$ admits decomposition $\mu = \mu^{+} - \mu^{-}$, where $\mu^{+}$ and $\mu^{-}$ are measures. We apply \nameref{thm:mct} to integrals with respect to $\mu^{+}$ and $\mu^{-}$. By definition of the integral with respect to $\mu$, we obtain the desired conclusion.
Hence $\Lambda$ is indeed a $\lambda$-system. By \nameref{thm:measure:lambda-pi}, $\B(\R) = \sigma(\Pi) = \lambda(\Pi) \subseteq \Lambda$.
Since $\Lambda \subseteq \B(\R)$, $\B(\R) = \Lambda$, as desired.
\end{step}
\begin{step}[$F_\vec{w}$ vanishes on measurable simple functions]
Suppose that $\varphi$ is a $\B(\R)$-measurable simple function. Without loss of generality, $\varphi = \sum_{k=1}^m \alpha_k \chi_{A_k} $, where $A_k$ are disjoint $\B(\R)$-measurable sets. By linearity of $F_\vec{w}$ and Step 2,
\begin{align}
    \label{ineqn:discrim:vanishdiscrimlemma:simple}
    F_{\vec{w}}(\varphi) = F \left (\sum_{k=1}^m \alpha_k \chi_{A_k} \right ) = \sum_{k=1}^m F_{\vec{w}}(\alpha_k \chi_{A_k}) =  \sum_{k=1}^m  \alpha_k F_{\vec{w}}( \chi_{A_k}) = 0.
\end{align}
\end{step}
\begin{step}[$F_\vec{w}$ vanishes on $\Linfty(\R)$]
Let $f \in \Linfty(\R)$. By \nameref{thm:lp:density}, there exists a sequence of $\B(\R)$-measurable simple functions $\{ \varphi_n \}_{n=1}^\infty$ converging to $f$ in $\norm{\cdot}_\infty$. For every $m \in \N$, $f - \varphi_m \in \Linfty(\R)$. Without loss of generality, $| f - \varphi_m | \leq \norm{f - \varphi_m}_\infty$. Then
\begin{subequations}\label{ineqn:discrim:vanishdiscrimlemma:vanish_goal}
\begin{align*}
    \left | F_{\vec{w}}(f) - F_{\vec{w}}(\varphi_m) \right | &\leq \left | \int_{[0,1]^n} (f - \varphi_m) ( \langle \vec{w}, \vec{x} \rangle) \, d \mu  (\vec{x}) \right | & \\
    & \leq \int_{[0,1]^n} |  (f - \varphi_m) ( \langle \vec{w}, \vec{x} \rangle) | \, d |\mu| ( \vec{x}) &\\
    & \leq \norm{f - \varphi_m}_\infty |\mu| ([0,1]^n).
    \tag{\ref{ineqn:discrim:vanishdiscrimlemma:vanish_goal}} 
\end{align*}
\end{subequations}
Since $\lim_{m \to \infty}  \norm{f - \varphi_m}_\infty = 0$ and $ |\mu| ([0,1]^n) < \infty$, by \ref{ineqn:discrim:vanishdiscrimlemma:vanish_goal}, $\left | F_{\vec{w}}(f) - F_{\vec{w}}(\varphi_m) \right | \to 0$, as $m \to \infty$.
Combining   $\left | F_{\vec{w}}(f) - F_{\vec{w}}(\varphi_m) \right | \to 0$ as $m \to \infty$ with  \ref{ineqn:discrim:vanishdiscrimlemma:simple} gives
\begin{align}
   \label{ineqn:discrim:vanishdiscrimlemma:linfty}
   F_{\vec{w}}(f) = \lim_{m \to \infty} F_{\vec{w}}(\varphi_m) = 0.
\end{align}
Since $f$ was arbitrary, $F_\vec{w}$ vanishes on $\Linfty(\R)$.
\end{step}
\begin{step}[$\mu$ is identically zero.]
We will compute the Fourier transform of $\mu$.
Since $\cos$ and $\sin$ are bounded and measurable, $\cos, \sin \in \Linfty(\R)$. By \ref{ineqn:discrim:vanishdiscrimlemma:linfty}, we have that $F_\vec{w}(\cos) = F_\vec{w}(\sin) = 0$. This implies  
\begin{align*}
    \widehat{\mu}(\vec{w}) &= \int_{[0,1]^n} e^{i \langle \vec{w}, \vec{x} \rangle } \, d\mu(\vec{x}) & \\
                           &=  \int_{[0,1]^n} \cos (\langle \vec{w}, \vec{x} \rangle) \, d\mu(\vec{x}) + i \int_{[0,1]^n} \sin (\langle \vec{w}, \vec{x} \rangle) \, d\mu(\vec{x})  &\\
                           &= F_\vec{w}(\cos) + i F_\vec{w}(\sin) &\\
                           &= 0.
\end{align*}
By Corollary \ref{corr:fourier:uniqmeasure}, $\mu = 0$.
\end{step}
\end{proof}


We are ready to address the second question.

\subsection{Sigmoidal activation functions}
\label{subsection:universality:cybenko:2}
\begin{definition}[sigmoidal activation function]
A function $\sigma : \R \to \R$ is called sigmoidal if \[
    \lim_{x \to \infty} \sigma(x) = 1 \text{ and } \lim_{x \to -\infty} \sigma(x) = 0.
\]
\end{definition}

\begin{proposition}[Bounded measurable sigmoidal functions are discriminatory]
\label{prop:discrim:boundedmeasdiscrim}
Let $\mu$ be a finite signed Borel measure on $[0,1]^n$. Suppose that $\sigma : \R \to \R$ is bounded, Borel measurable and sigmoidal function. Then $\sigma$ is discriminatory for $\mu$. 
\end{proposition}
\begin{proof-idea*}
We will reduce the problem to the application of Lemma \ref{lemma:discrim:vanishhyper}.
\end{proof-idea*}
\begin{proof}
Suppose that $\sigma$ satisfies \begin{align}
    \label{eqn:discrim:bdsigm:sigma}
    \int_{[0,1]^n}\sigma(\langle \vec{w}, \vec{x} \rangle + b) \, d \mu  (\vec{x}) = 0, \text{ for every } \vec{w} \in \R^n, b \in \R.
\end{align}
We need to show that $\mu = 0$. We aim to apply Lemma \ref{lemma:discrim:vanishhyper}. For fixed $\lambda, a \in \R$, define $\sigma_{\lambda, a} : [0,1]^n \to \R$ by \[
    \sigma_{\lambda, a} (\vec{x}) = \sigma \left ( \lambda (\langle \vec{w}, \vec{x} \rangle + b) + a \right ) = \sigma \left (\langle \lambda \vec{w}, \vec{x} \rangle + b\lambda + a \right ).
\]
By \ref{eqn:discrim:bdsigm:sigma}, 
 \begin{align}
    \label{eqn:discrim:bdsigm:sigmalambda}
    \int_{[0,1]^n}\sigma_{\lambda, a} (\vec{x}) \, d \mu  (\vec{x}) = 0.
\end{align}
Define $\gamma : [0,1]^n \to \R$ by $\gamma(\vec{x}) = \lim_{\lambda \to \infty} \sigma_{\lambda, a} (\vec{x})$. Observe that  \begin{equation}
    \label{eqn:discrim:bdsigm:gamma}
    \gamma(\vec{x}) =
    \begin{cases}
      1         & \text{ if } \langle \vec{w}, \vec{x} \rangle + b > 0 \\
      \sigma(a) & \text{ if } \langle \vec{w}, \vec{x} \rangle + b = 0 \\
      0         & \text{ if } \langle \vec{w}, \vec{x} \rangle + b < 0
    \end{cases}.
  \end{equation}
  
By \ref{eqn:discrim:bdsigm:gamma}, \begin{align*}
    \int_{[0,1]^n} \gamma(\vec{x}) d \mu  (\vec{x}) &= \int_{\Pi_{\vec{w}, b}^{+}} \gamma(\vec{x}) d \mu  (\vec{x})  + \int_{\Pi_{\vec{w}, b}} \gamma(\vec{x}) d \mu  (\vec{x})  + \int_{\Pi_{\vec{w}, b}^{-}} \gamma(\vec{x}) d \mu  (\vec{x})  & \\
                                                    &=\int_{\Pi_{\vec{w}, b}^{+}} 1 d \mu  (\vec{x})  + \int_{\Pi_{\vec{w}, b}} \sigma(a) d \mu  (\vec{x})  + \int_{\Pi_{\vec{w}, b}^{-}} 0 d \mu  (\vec{x})  & \\
                                                   &= \mu(\Pi_{\vec{w}, b}^{+}) + \sigma(a) \cdot \mu (\Pi_{\vec{w}, b}).
\end{align*}
Taking $\lim_{a \to \infty}$ and applying the fact $\sigma$ is sigmoidal gives
\begin{align}
      \label{eqn:discrim:bdsigm:vanish_1}
      \int_{[0,1]^n} \gamma(\vec{x}) d \mu  (\vec{x}) = \mu(\Pi_{\vec{w}, b}^{+}) + \lim_{a \to \infty} \sigma(a) \cdot \mu (\Pi_{\vec{w}, b}) = \mu(\Pi_{\vec{w}, b}^{+}) + \mu (\Pi_{\vec{w}, b}).
\end{align}
Taking $\lim_{a \to -\infty}$ and applying the fact $\sigma$ is sigmoidal gives
\begin{align}
      \label{eqn:discrim:bdsigm:vanish_2}
      \int_{[0,1]^n} \gamma(\vec{x}) d \mu  (\vec{x}) = \mu(\Pi_{\vec{w}, b}^{+}) + \lim_{a \to -\infty} \sigma(a) \cdot \mu (\Pi_{\vec{w}, b}) = \mu(\Pi_{\vec{w}, b}^{+}).
\end{align}
Equating \ref{eqn:discrim:bdsigm:vanish_1} and  \ref{eqn:discrim:bdsigm:vanish_2} gives
\begin{align}
     \mu(\Pi_{\vec{w}, b}^{+}) + \mu (\Pi_{\vec{w}, b}) = \mu(\Pi_{\vec{w}, b}^{+}) \implies \mu (\Pi_{\vec{w}, b}) = 0.
\end{align}
By \ref{eqn:discrim:bdsigm:vanish_2}, to prove $\mu(\Pi_{\vec{w}, b}^{+}) = 0$, it is equivalent to prove $\int_{[0,1]^n} \gamma(\vec{x}) d \mu  (\vec{x}) = 0$. We will appeal to the \nameref{thm:dct}.
By \nameref{thm:hahn-jordan}, $\mu$ can be decomposed as $\mu = \mu^{+} - \mu^{-}$, where $\mu^{+}$ and $\mu^{-}$ are measures and at least one of them is finite. Since $\mu$ is finite, so are both $\mu^{+}$ and $\mu^{-}$. \newpage
Since $\sigma$ is bounded, for every $\lambda \in \R$, for every $a \in \R$, for every $\vec{x} \in [0,1]^n$,  \[ 
    |\sigma_{\lambda, a} (\vec{x})| \leq \norm{\sigma}_\infty.
\]
Since $\mu^{+}$ and $\mu^{-}$ are finite, $\vec{x} \to \norm{\sigma}_\infty$ is $\mu^{+}$ and $\mu^{-}$ integrable. Now \begin{align}
    \label{eqn:discrim:bdsigm:vanish_int_main}
    \int_{[0,1]^n} \gamma(\vec{x}) d \mu  (\vec{x}) &=  \int_{[0,1]^n} \gamma(\vec{x}) d \mu^{+} (\vec{x}) -  \int_{[0,1]^n} \gamma(\vec{x}) d \mu^{-} (\vec{x}).
\end{align}
By \ref{eqn:discrim:bdsigm:gamma} and \nameref{thm:dct}, we have \begin{align}
     \label{eqn:discrim:bdsigm:vanish_int_1}
     \int_{[0,1]^n} \gamma(\vec{x}) d \mu^{+} (\vec{x}) = \lim_{\lambda \to \infty}  \int_{[0,1]^n} \sigma_{\lambda, a} (\vec{x}) d \mu^{+} (\vec{x}).
\end{align}
Similarly, we have \begin{align}
     \label{eqn:discrim:bdsigm:vanish_int_2}
     \int_{[0,1]^n} \gamma(\vec{x}) d \mu^{-} (\vec{x}) = \lim_{\lambda \to \infty}  \int_{[0,1]^n} \sigma_{\lambda, a} (\vec{x}) d \mu^{-} (\vec{x}).
\end{align}
Applying \ref{eqn:discrim:bdsigm:vanish_int_1} and \ref{eqn:discrim:bdsigm:vanish_int_2} to \ref{eqn:discrim:bdsigm:vanish_int_main} gives
\begin{align*}
    \int_{[0,1]^n} \gamma(\vec{x}) d \mu  (\vec{x}) &= \lim_{\lambda \to \infty}  \int_{[0,1]^n} \sigma_{\lambda, a} (\vec{x}) d \mu^{+} (\vec{x}) - \lim_{\lambda \to \infty}  \int_{[0,1]^n} \sigma_{\lambda, a} (\vec{x}) d \mu^{-} (\vec{x}) & \\ 
                                                   &= \lim_{\lambda \to \infty} \left (\int_{[0,1]^n} \sigma_{\lambda, a} (\vec{x}) d \mu^{+} (\vec{x})  -\int_{[0,1]^n} \sigma_{\lambda, a} (\vec{x}) d \mu^{-} (\vec{x}) \right ) & \\ 
                                                   &= \lim_{\lambda \to \infty}\left (\int_{[0,1]^n} \sigma_{\lambda, a} (\vec{x}) d \mu  (\vec{x}) \right) &\\
                                                   &= 0,
\end{align*}
since by \ref{eqn:discrim:bdsigm:sigmalambda}, $\int_{[0,1]^n} \sigma_{\lambda, a} (\vec{x}) d \mu  (\vec{x}) = 0. $ We have shown that $\mu$ vanishes on $\Pi_{\vec{w}, b}^{+}$ and $\Pi_{\vec{w}, b}$. By Lemma \ref{lemma:discrim:vanishhyper}, $\mu$ is identically zero.
We conclude $\sigma$ is discriminatory for $\mu$, as claimed.
\end{proof}

\begin{proposition}[Continuous sigmoidal functions are discriminatory]
\label{prop:discrim:contsigmoidalarediscrim}
Let $\mu$ be a finite signed Borel measure on $[0,1]^n$. Suppose that $\sigma : \R \to \R$ is a continuous sigmoidal function. Then $\sigma$ is discriminatory for $\mu$. 
\end{proposition}
\begin{proof}
Since $\sigma$ is continuous, it is Borel measurable. Since it is continuous and sigmoidal, $\sigma$ is bounded. The result follows directly from
Proposition \ref{prop:discrim:boundedmeasdiscrim}.
\end{proof}

\begin{corollary}
Consequently, any continuous sigmoidal function is discriminatory.
\end{corollary}
\begin{proof}
Apply Proposition \ref{prop:discrim:contsigmoidalarediscrim} to arbitrary $\mu$.
\end{proof}
\begin{corollary}
Logistic sigmoid is discriminatory.
\end{corollary}
\begin{corollary}
Heaviside step function is discriminatory.
\end{corollary}
It turns out that a wide variety of bounded functions are discriminatory.
\begin{theorem}[Theorem 5 in \cite{hornik_1991_approximation}]
Whenever $\sigma$ is bounded and nonconstant, it is discriminatory.
\end{theorem}
\begin{proof}
See Theorem 5 in \cite{hornik_1991_approximation}.
\end{proof}