\section{Universal approximation of measurable functions on compact sets}
\label{section:universality:measurable:compact}
The focus in this section remains on the unit hypercube $[0,1]^n$ in $\R^n$. We will work with the measure space $([0,1]^n, \B([0,1]^n), \lambda_{|[0,1]^n})$, where $\lambda_{|[0,1]^n}$ is the restriction of Lebesgue measure on $\R^n$ to $[0,1]^n$. We begin the discussion about approximation of real-valued, measurable functions on $[0,1]^n$ with the following theorem.
\begin{theorem}[Cybenko's Universal Approximation Theorem for measurable functions, \cite{cybenko_1989_approximation}]
\label{thm:universality:cybenko:measurable}
Let $\mathcal{H}_{\sigma}$ denote the family of single-layer fully-connected neural networks with any continuous sigmoidal activation function, given by \begin{align*}
\mathcal{H}_{\sigma} = \left \{ \vec{x} \to \sum_{k=1}^{m} \alpha_k \sigma{\left (\langle \vec{w_k}, \vec{x} \rangle + \beta_k \right)} : m \in \N, \alpha_1 \ldots \alpha_m, \beta_1 \ldots \beta_m \in \R, \vec{w_k} \in \R^n \right \}.
\end{align*}
Suppose that $f : [0,1]^n \to \R$ is Borel measurable and let $\epsilon > 0$. There exists a network $h \in \mathcal{H}_{\sigma}$ and a set $K \subseteq [0,1]^n$ such that \[
    \delta_{\infty}(f_{| K}, h) < \epsilon
\]
where $\lambda(K) > 1 - \epsilon$.
\end{theorem}
Informally, for every Borel measurable function $f$, there exists a fully-connected neural network $h \in \mathcal{H}_{\sigma}$ approximating $f$ to a desired error, except possibly on sets of arbitrarily small measure.

\begin{proof-idea*}
We will apply \nameref{thm:measure:lusin}.
\begin{theorem*}[Lusin's Theorem, Theorem 7.4.4 in \cite{cohn_2013_measure}]
Let $X$ be a locally compact Hausdorff space and let $\mathcal{A}$ be a $\sigma$-algebra that includes $\B(X)$. Let $\mu$ be a regular measure on $(X, \mathcal{A})$ and suppose $f : X \to \R$ is measurable. If $A \in \mathcal{A}$ and satisfies $\mu(A) < \infty$ and if $\epsilon > 0$, then there is a compact set $K \subseteq A$ such that $\mu(A \setminus K) < \epsilon$ and $f_{|K}$ is continuous. Moreover, there is a function $g \in \C(X)$ that agrees with $f$ on $K$.
\end{theorem*}
Lusin's Theorem will provide us with a compact set $K$ and a continuous function $g : [0,1]^n \to \R$  which agrees with $f$ on $K$. The result will follow from \nameref{thm:universality:cybenko}.
\end{proof-idea*}

\begin{proof}
Since $[0,1]^n$ is a compact metric space, it is locally compact and Hausdorff. 
By \nameref{thm:measure:lusin}, there exists a compact set $K \subseteq [0,1]^n$ such that $\lambda([0,1]^n \setminus K) < \epsilon$ and a continuous function $g : [0,1]^n \to \R$ such that $f(\vec{x}) = g(\vec{x}) \text{ for every $\vec{x}$} \in K.$ By \nameref{thm:universality:cybenko}, there exists a neural network $h \in \mathcal{H}_{\sigma}$ such that
$\delta_{\infty}(g, h) < \epsilon$. Since $f_{| K} = g$, we have $\delta_{\infty}(f_{| K}, h) < \epsilon$. Since $\lambda([0,1]^n \setminus K) < \epsilon$, $\lambda(K) > 1 - \epsilon$.
\end{proof}

\begin{remark}
The consequences of this result are briefly discussed in subsection \ref{subsection:universality:measure:relationship}.
\end{remark}
\begin{remark}
\nameref{thm:universality:cybenko:measurable} also holds for any compact set in $\R^n$.
\end{remark}

\section{Universal approximation of measurable functions in probabilistic sense}
\label{section:universality:measurable:probabilistic}

\subsection{Introduction}
Until this section, we focused on the approximation power of neural networks on a compact domain. We focused on spaces of continuous functions and Lebesgue spaces. However, another practically important space is the space of measurable functions whose domain is often unrestricted. An example of such a space would be the space of Borel measurable functions from $\R^n$ to $\R$. Often, neural networks are used to directly or indirectly learn a probability distribution of some random variable. Since random variables are defined as measurable functions, it is sensible and important to investigate approximation properties that neural networks possess when the approximation space is the space of measurable functions. 

In this section, we will focus on the space of all Borel measurable functions from $\R^n$ to $\overline{\R}$, denoted by $\M^n$.
We will drop the assumption about the compactness of an approximation domain and consider the approximation on the entire $\R^n$, albeit in a probabilistic sense. Since measurable functions are not necessarily bounded, it is not immediately obvious how to equip $\M^n$ with a topology with respect to which we can discuss the approximation properties. 

To address those issues, we will begin by introducing a few measure-theoretic assumptions that will enable us to construct a few useful metrics. In this section, we will focus on the approximation in the probabilistic sense, so let $\mu$ be a probability measure on the measurable space $(\R^n, \B(\R^n))$.
\begin{description}
\item[$\mu$-a.e. equivalence] When discussing the approximation in a probabilistic sense, it is reasonable not to distinguish between measurable functions $f, g \in \M^n$ if they are $\mu$-almost everywhere equivalent. Recall that two measurable functions $f, g \in \M^n$ are $\mu$-almost everywhere equivalent if \[ 
    \mu (\{ \vec{x} \in \R^n : f (\vec{x}) \neq g(\vec{x})  \}) = 0.
\]
\newpage
Furthermore, $\mu$-almost everywhere equivalence is compatible with the fact that Lebesgue integral cannot distinguish between two almost everywhere equivalent functions. It is not difficult to show that $\mu$-almost everywhere equivalence on $\M^n$ is in fact an equivalence relation. For the sake of simplicity, we will denote the space of equivalence classes under the $\mu$-almost everywhere equivalence also by $\M^n$. This abuse of notation is analogous to one often used in literature when discussing $\Lp(\R^n, \B(\R^n), \lambda)$.
\item[$\mu$-a.e. finiteness] Another important and mostly technical assumption is the assumption about $\mu$-almost everywhere finiteness. Recall $f \in \M^n$ is $\mu$-almost everywhere finite if \[
    \mu (\{ \vec{x} \in \R^n : | f(\vec{x}) | = \infty \}) = 0.
\]
In this section, we will assume that $\M^n$ consists of equivalence classes of $\mu$-almost everywhere equivalent functions that are also $\mu$-almost everywhere finite.
\end{description}

Under the assumptions above, we can equip $\M^n$ with a metric. A metric will help us formalize the notion of distance between measurable functions and the approximation error. There are many ways one can equip $\M^n$ with a metric. Since we want to discuss approximation in a probabilistic sense, we would like to link our metric to convergence in probability measure $\mu$. An example of such a metric is $\delta_\mu$ metric, where $\delta_\mu : \M^n \times \M^n \to \R$ is given by \[
    \delta_\mu (f, g) = \inf \{ \epsilon > 0 : \mu (\{ \vec{x} \in \R^n : | f (\vec{x}) - g (\vec{x}) | > \epsilon \}) < \epsilon \}.
\]
By \textbf{$\mu$-a.e. finiteness} assumption, $f - g$ can be infinite only on sets of probability measure zero. Another metric we will discuss and use in calculations is the $\rho_\mu$ metric, $\rho_\mu : \M^n \times \M^n \to \R$ given by \[
    \rho_\mu (f, g) = \mathbb{E} [\min(| f - g |, 1)] = \int_{\R^n} \min(| f - g |, 1) \, d\mu.
\]
\begin{remark}
Observe that since $(\R^n, \B(\R^n), \mu)$ is a finite measure space, $\rho_\mu$ is finite for every $f, g \in \M^n$. 
In literature, $\delta_\mu$ metric is also known as \textbf{Ky Fan metric}.
\end{remark}
\begin{remark}
Proofs that $\delta_\mu$ and $\rho_\mu$ are indeed metrics are discussed in detail in the subsection \nameref{subsection:universality:measure:modes_of_convergence}.
\end{remark}
\begin{remark}
In the subsequent sections, we will use the following shortened notation  \[
    \{ | f - g| > \epsilon \} = \{ \vec{x} \in \R^n : | f(\vec{x}) - g(\vec{x}) | > \epsilon \}.
\]
Also, we will denote by $f \wedge g$ the function $\min(f,g)$.
\end{remark}
We will briefly discuss the connection between metrics $\delta_\mu$ and $\rho_\mu$ to the convergence in probability measure $\mu$. Firstly, recall the definition of convergence in measure $\mu$. 

\begin{definition}[convergence in measure]
Let $(\Omega, \mathcal{F}, \mu)$ be a measure space. Let $\{ f_m \}_{m=1}^\infty$ be a sequence of $\mathcal{F}$-measurable, $\overline{\R}$-valued functions and suppose that $f$ is $\mathcal{F}$-measurable. We say that $f_m \to f$ in measure $\mu$ if for every $\epsilon > 0$, \[ 
    \mu(\{ \omega \in \Omega : | f_m (\omega) - f(\omega) | > \epsilon \}) \to 0 \text{ as $m \to \infty.$}
\]
\end{definition}
Interestingly, $\delta_\mu$ and $\rho_\mu$ are equivalent metrics. Consequently, they induce the same topology on $\M^n$. This fact is stated precisely as \nameref{proposition:universality:measure:modes_convergence}, stated below.

\begin{proposition*}[Characterisation of convergence in probability, Lemma 2.1 in \cite{hornik}]
Let $\{ f_m \}_{m=1}^\infty$ be a sequence of functions in $\M^n$ and let $f \in \M^n$. Then the following statements are equivalent:
\begin{enumerate}[label=(\alph*),noitemsep]
\item $\delta_\mu (f_m, f) \to 0$ as $m \to \infty$;
\item $\mu (\{ | f_m - f | > \epsilon \}) \to 0$ as $m \to \infty$;
\item $\rho_\mu (f_m, f) \to 0$ as $m \to \infty$.
\end{enumerate}
\end{proposition*}
\begin{remark}
As a consequence of \nameref{proposition:universality:measure:modes_convergence}, to prove convergence in probability measure $\mu$, it is sufficient and in fact equivalent to prove convergence in either $\delta_\mu$ or $\rho_\mu$. Thanks to the existence of $\delta_\mu$ and $\rho_\mu$, the arguments addressing the convergence in measure $\mu$ will be much easier to read and understand. They will often reduce to exact computation or an estimation of the integral $\rho_\mu$. 
\end{remark}
The main goal of this section is the statement and the proof of \nameref{thm:universality:measure:uap:probabilistic}, stated below.
\begin{theorem*}[The Probabilistic Universal Approximation Theorem]
Let $\mathcal{H}_{\sigma}$ denote the family of single-layer fully-connected neural networks with any continuous sigmoidal activation function $\sigma$, given by \begin{align*}
\mathcal{H}_{\sigma} = \left \{ \vec{x} \to \sum_{k=1}^{m} \alpha_k \sigma{\left (\langle \vec{w_k}, \vec{x} \rangle + \beta_k \right)} : m \in \N, \alpha_1 \ldots \alpha_m, \beta_1 \ldots \beta_m \in \R, \vec{w_k} \in \R^n \right \}.
\end{align*}
Then $\mathcal{H}_{\sigma}$ is $\delta_\mu$-dense in $\M^n$.
\end{theorem*}
To prove this theorem, we will need to develop a few auxiliary results, organised in the subsections described below.
\begin{description}
\item[\nameref{subsection:universality:measure:modes_of_convergence}] In this subsection, we will prove that $\delta_\mu$ and $\rho_\mu$ are indeed metrics on $\M^n$. Apart from this, we will link convergence on compacta to convergence in measure $\mu$ via Proposition \ref{proposition:universality:measure:convg_compacta_implies_mu}.
\item[\nameref{subsection:universality:measure:towards_probabilistic_uap}] In this subsection, we will build upon concepts developed in \nameref{subsection:universality:measure:modes_of_convergence}. We will begin this subsection by discussing density of neural networks on compacta, 
in the sense of Theorem \ref{thm:universality:measure:uap:nnsdensecompacta}. To prove  Theorem \ref{thm:universality:measure:uap:nnsdensecompacta}, we will apply \nameref{thm:universality:cybenko}. The key result in this subsection is the Proposition 
\ref{proposition:universality:measure:continuous_dense_mn}, which guarantees that $\mathcal{C}(\R^n)$ is $\rho_\mu$-dense in $\M^n$.
\item[\nameref{subsection:universality:measure:probabilistic_uap}] In this subsection, we will prove the \nameref{thm:universality:measure:uap:probabilistic}.
The key idea will be to combine density of neural networks on compacta, 
in the sense of Theorem \ref{thm:universality:measure:uap:nnsdensecompacta} with the Proposition 
\ref{proposition:universality:measure:continuous_dense_mn}.
\item[\nameref{subsection:universality:measure:relationship}] In this subsection, we will discuss the application of \nameref{thm:universality:cybenko:measurable} to classification problems.
\end{description}

\subsection{Metrics and modes of convergence}
We begin this subsection with a proof that $\delta_\mu$ is indeed a metric on $\M^n$.
\label{subsection:universality:measure:modes_of_convergence}
\begin{proposition}
\label{proposition:universality:measure:deltamumetric}
$\delta_\mu$ is indeed a metric on $\M^n$.
\end{proposition}
\begin{proof-idea*}
The proof is essentially a verification of properties a metric must satisfy. The key observation is that the statement $\delta_\mu(f, g) = 0$ is equivalent to $f = g$ $\mu$-almost everywhere. The justification of triangle inequality is not difficult, but relatively technical.
\end{proof-idea*}
\begin{proof}
Clearly, $\delta_\mu \geq 0$. Symmetry of $\delta_\mu$ is obvious.
\setcounter{step}{0}
\begin{step}[$\delta_\mu(f, g) = 0 \iff f = g$ $\mu$-almost everywhere]
Suppose that $f = g$ $\mu$-almost everywhere. We will show  $\delta_\mu(f, g) = 0$. Let $\epsilon > 0$. Since $f = g$ $\mu$-almost everywhere, $| f - g | = 0$ $\mu$-almost everywhere. Hence, there exists a set $M \in \B(\R^n)$ such that $\mu(M) = 0$ and $| f - g | > 0$ on $M$, while $| f - g | = 0$ on $\R^n \setminus M$. Since $\epsilon > 0$, by subadditivity of measure $\mu$,
\begin{align}
    \label{ineqn:universality:measure:deltamumetric:muae:-1}
    \mu (\{ | f - g | > \epsilon \}) \leq \mu (\{ | f - g | > 0 \}) \leq \mu (M) = 0 < \epsilon.
\end{align}
Since $\delta_\mu$ is an infimum, by \ref{ineqn:universality:measure:deltamumetric:muae:-1}, $0 \leq \delta_\mu(f,g) < \epsilon$. Since $\epsilon$ was arbitrary, we conclude $\delta_\mu(f,g) = 0$.

Conversely, suppose that $\delta_\mu (f, g) = 0$. We have \begin{align}
    \label{ineqn:universality:measure:deltamumetric:muae:0}
    \{ f \neq g \} = \{ | f - g | > 0 \} \subseteq \bigcup_{n = 1}^\infty  \left \{ |f - g| > \frac{1}{n} \right \}.
\end{align}
For every $n \in \N$, since $\delta_\mu(f,g) < \frac{1}{n}$ and $\delta_\mu(f,g)$ is an infimum, there exists $0 < \epsilon_n < \frac{1}{n}$ such that $\mu(\left \{ | f - g| > \epsilon_n \right \}) < \epsilon_n$. Since $\epsilon_n < \frac{1}{n}$, $\left \{ | f - g | > \frac{1}{n} \right \} \subseteq \{ | f - g | > \epsilon_n \}$ and so 
\begin{align}
    \label{ineqn:universality:measure:deltamumetric:muae:1}
    \mu ( \left \{ |f - g| > \frac{1}{n} \right \}) \leq  \mu(\left \{ |f - g| > \epsilon_n \right \}) < \epsilon_n < \frac{1}{n}.
\end{align}
By \ref{ineqn:universality:measure:deltamumetric:muae:1}, $\lim_{n \to \infty} \mu ( \left \{ |f - g| > \frac{1}{n} \right \}) = 0$. Observe that \begin{align}
    \label{ineqn:universality:measure:deltamumetric:muae:2}
    \left \{ |f - g| > \frac{1}{n} \right \}\subseteq \left \{ |f - g| > \frac{1}{n + 1} \right \}, \text{ for every $n \in \N$.}
\end{align}
By \ref{ineqn:universality:measure:deltamumetric:muae:2}, \begin{align*}
    \mu \left(  \bigcup_{n = 1}^\infty  \left \{ |f - g| > \frac{1}{n} \right \} \right ) = \lim_{n \to \infty} \mu ( \left \{ |f - g| > \frac{1}{n} \right \}) = 0.
\end{align*}
By \ref{ineqn:universality:measure:deltamumetric:muae:0}, $\mu (\{ f \neq g \}) = 0$.
\end{step}
\begin{step}[The triangle inequality]
To prove triangle inequality, let $f, g, h \in \M^n$. Let $\alpha > 0$. Since $\delta_\mu(f, h)$ and $\delta_\mu(h, g)$ are infima, there exist $r_1, r_2 > 0$ such that
\begin{align}
    \label{ineqn:universality:measure:deltamumetric:triangle:1}
    \delta_\mu(f, h) \leq r_1 < \delta_\mu(f, h) + \frac{\alpha}{2} \text{ and } 
    \delta_\mu(h, g) \leq r_2 < \delta_\mu(h, g) + \frac{\alpha}{2},
\end{align}
satisfying \begin{align}
    \label{ineqn:universality:measure:deltamumetric:triangle:2}
    \mu (\left \{ | f - h | > r_1 \right \} ) < r_1 \text{ and } \mu (\left \{ | h - g | > r_2 \right \} ) < r_2 .
\end{align}
For every $\vec{x} \in \R^n$, if $| f(\vec{x}) - h(\vec{x}) | \leq r_1$ and $| h(\vec{x}) - g(\vec{x}) | \leq r_2$ then \[ 
    | f(\vec{x}) - g(\vec{x}) | \leq  | f(\vec{x}) - h(\vec{x}) |  + | h(\vec{x}) - g(\vec{x}) | \leq r_1 + r_2.
\]
By contrapositive, \begin{align}
    \label{ineqn:universality:measure:deltamumetric:triangle:3}
    \left \{ | f - g | > r_1 + r_2 \right \} \subseteq \left \{ | f - h | > r_1 \right \} \cup \left \{ | h - g | > r_2 \right \}.
\end{align}
Applying subadditivity of $\mu$ to \ref{ineqn:universality:measure:deltamumetric:triangle:3} gives \begin{align}
\label{ineqn:universality:measure:deltamumetric:triangle:4}
    \mu (\left \{ | f - g | > r_1 + r_2 \right \}) \leq \mu (\left \{ | f - h | > r_1 \right \}) + \mu (\left \{ | h - g | > r_2 \right \}).
\end{align}
Applying \ref{ineqn:universality:measure:deltamumetric:triangle:2} and \ref{ineqn:universality:measure:deltamumetric:triangle:1} to \ref{ineqn:universality:measure:deltamumetric:triangle:4} gives \begin{align}
\label{ineqn:universality:measure:deltamumetric:triangle:5}
    \mu (\left \{ | f - g | > r_1 + r_2 \right \}) \leq r_1 + r_2 < \delta_\mu(f, h) + \delta_\mu(h, g) + \alpha.
\end{align}
Since $\delta_\mu(f,g)$ is an infimum, by \ref{ineqn:universality:measure:deltamumetric:triangle:5}, $\delta_\mu(f,g) \leq r_1 + r_2 < \delta_\mu(f, h) + \delta_\mu(h, g) + \alpha$. Since $\alpha$ was arbitrary, $\delta_\mu(f,g) \leq \delta_\mu(f, h) + \delta_\mu(h, g)$, as desired.
\end{step}
\end{proof}
We continue with a proof that $\rho_\mu$ is indeed a metric on $\M^n$.
\begin{proposition}
\label{proposition:universality:measure:rhomumetric}
$\rho_\mu$ is indeed a metric on $\M^n$.
\end{proposition}
\begin{proof}
The symmetry and non-negativity of $\rho_\mu$ are obvious.
\setcounter{step}{0}
\begin{step}
The fact $\rho_\mu(f, g) = 0$ is equivalent to $f = g$ $\mu$-almost everywhere follows directly from properties of Lebesgue integral of a non-negative function, namely Proposition \ref{proposition:measure:characterzerointergral_two_dir}. 
\end{step}
\begin{step}[The triangle inequality]
Let $f, g, h \in \M^n$. By triangle inequality on $\R$, \begin{align}
    \label{ineqn:universality:measure:rhomumetric:triangle:1}
    |f(\vec{x}) - g(\vec{x})| \leq |f(\vec{x}) - h(\vec{x})| + |h(\vec{x}) - g(\vec{x})|, \text{ for every $\vec{x} \in \R^n$}.
\end{align}
By \ref{ineqn:universality:measure:rhomumetric:triangle:1}, 
\begin{subequations}\label{ineqn:universality:measure:rhomumetric:triangle:2}
\begin{align*}
    |f(\vec{x}) - g(\vec{x})| \wedge 1 &\leq (|f(\vec{x}) - h(\vec{x})| + |h(\vec{x}) - g(\vec{x})|) \wedge 1, \text{ for every $\vec{x} \in \R^n $}, \\
                                       &\leq |f(\vec{x}) - h(\vec{x})| \wedge 1 + |h(\vec{x}) - g(\vec{x})| \wedge 1, \text{ for every $\vec{x} \in \R^n $} .
     \tag{\ref{ineqn:universality:measure:rhomumetric:triangle:2}} 
\end{align*}
\end{subequations}
Integrating the inequality \ref{ineqn:universality:measure:rhomumetric:triangle:2} gives \begin{align*}
    \rho_\mu(f,g) = \int_{\R^n} | f - g | \wedge 1 \, d\mu &\leq  \int_{\R^n} | f - h | \wedge 1 \, d\mu + \int_{\R^n} | h - g | \wedge 1 \, d\mu \\ &\leq \rho_\mu(f, h) + \rho_\mu(h,g).
\end{align*}
\end{step}
\end{proof}
The following proposition will provide us with a characterization of convergence in $\mu$ in terms of convergence in metrics $\delta_\mu$ and $\rho_\mu$.
\begin{proposition}[Characterization of convergence in probability, Lemma 2.1 in \cite{hornik}]
\label{proposition:universality:measure:modes_convergence}
Let $\{ f_m \}_{m=1}^\infty$ be a sequence of functions in $\M^n$ and let $f \in \M^n$. Then the following statements are equivalent:
\begin{enumerate}[label=(\alph*),noitemsep]
\item $\delta_\mu (f_m, f) \to 0$ as $m \to \infty$;
\item For every $\epsilon > 0$, $\mu (\{ | f_m - f | > \epsilon \}) \to 0$ as $m \to \infty$;
\item $\rho_\mu (f_m, f) \to 0$ as $m \to \infty$.
\end{enumerate}
\end{proposition}
\begin{proof-idea*}
It is reasonable to attempt proving the implication chain $(a) \Longrightarrow (b) \Longrightarrow (c) \Longrightarrow (a)$.
However, it turns out that proving $(c) \Longrightarrow (a)$ is relatively difficult, so we follow a slightly different method.
The strategy is to prove that $(a)$ and $(b)$ are equivalent and separately prove that $(b)$ and $(c)$ are equivalent. The proof will be divided in four steps where each step will be a proof a single implication.
\end{proof-idea*}
\begin{proof}
\setcounter{step}{0}
\begin{step}[$a \Longrightarrow b$]
Suppose $\delta_\mu (f_m, f) \to 0$ as $m \to \infty$. Let $\epsilon > 0$. Then there exists $M \in \N$ such that 
\begin{align}
     \label{ineqn:universality:measure:modes_convergence:ab_1}
     \delta_\mu (f_m, f) = \inf \{ \alpha > 0 : \mu(\{ |f_m - f| > \alpha\}) < \alpha \} < \epsilon, \text{ for $m \geq M$.} 
\end{align}
By definition of $\delta_\mu$, there exists $0 < \alpha < \epsilon$ such that $\mu(\{ |f_m - f| > \alpha\}) < \alpha < \epsilon$,
for $m \geq M$. Since $0 < \alpha < \epsilon$, $\left \{ |f_m - f| > \epsilon  \right \} \subseteq \left \{ |f_m - f| > \alpha \right \}$, so we have 
\begin{align*}
    \mu (\left \{ |f_m - f| > \epsilon  \right \} ) \leq \mu(\left \{ |f_m - f| > \alpha  \right \}) < \alpha \leq \epsilon, \text{ for $m \geq M$.}
\end{align*}
\end{step}
\begin{step}[$b \Longrightarrow a$]
Suppose that for every $\epsilon$, $ \mu (\left \{ |f_m - f| \geq \epsilon  \right \} ) \to 0 $ as $m \to \infty$. Then fix $\epsilon > 0$. Now there exists $M \in \N$ such that 
\begin{align}
     \label{ineqn:universality:measure:modes_convergence:ba_1}
     \mu (\{ | f_m - f | > \epsilon \}) < \epsilon, \text{ for $m \geq M$.} 
\end{align}
Since $\delta_\mu$ is an infimum, by \ref{ineqn:universality:measure:modes_convergence:ba_1}, $\delta_\mu(f_m, f) \leq \epsilon$, for $m \geq M$.
\end{step}
\begin{step}[$b \Longrightarrow c$]
Suppose that for every $\epsilon > 0, \mu (\{ | f_m - f | > \epsilon \}) \to 0$ as $m \to \infty$. Then there exists $M \in \N$ such that
\begin{align}
     \label{ineqn:universality:measure:modes_convergence:bc_1}
     \mu (\left \{ | f_m - f | > \frac{\epsilon}{2} \right \}) < \frac{\epsilon}{2}, \text{ for every $m \geq M$.}
\end{align}
We have \begin{align*}
    \int_{\R^n} |f_m -f| \wedge 1 \, d\mu &=  \int_{\left \{ |f_m - f| \leq \frac{\epsilon}{2} \right \}} |f_m -f| \wedge 1  \, d\mu + \int_{\left \{ |f_m - f| > \frac{\epsilon}{2} \right \}} |f_m -f| \wedge 1  \, d\mu \\
                                          &\leq \int_{\R^n} \frac{\epsilon}{2} \, d\mu +   \int_{\left \{ | f_m - f | > \frac{\epsilon}{2} \right \}} 1 \, d\mu \\
                                          &\leq \frac{\epsilon}{2} \mu(\R^n) + \mu (\left \{ | f_m - f | > \frac{\epsilon}{2}  \right \}) \\
                                          &< \frac{\epsilon}{2} + \frac{\epsilon}{2} = \epsilon, \text{ for $m \geq M$, by \ref{ineqn:universality:measure:modes_convergence:bc_1}}. 
\end{align*}
Hence $\rho_\mu (f_m, f) < \epsilon$, for $m \geq M$.
\end{step}
\begin{step}[$c \Longrightarrow b$]
Suppose $\rho_\mu (f_m, f) \to 0 $ as $m \to \infty$, so $\int_{\R^n} | f_m - f| \wedge 1 \, d\mu \to 0$ as $m \to \infty$. Let $\epsilon > 0$. To prove that $\mu(\{ |f_m - f| > \epsilon \}) \to 0$ as $m \to \infty$, it is sufficient to show that every subsequence $\mu(\{ |f_{m_k} - f| > \epsilon \}) $ has a subsubsequence $\mu(\{ |f_{m_{k_l}} - f| > \epsilon \})$ such that $\mu(\{ |f_{m_{k_l}} - f| > \epsilon \}) \to 0 $ as $l \to \infty$. Thus, let $\{ f_{m_k} \}_{k = 1}^\infty$ be an arbitrary subsequence of $\{ f_m \}_{m = 1}^\infty$. By \nameref{proposition:measure:ineqn:markov},
\begin{align}
    \label{eqn:universality:measure:modes_convergence:cb_1}
    \mu(\{ | f_{m_k} - f| \wedge 1 > \epsilon \}) \leq \frac{1}{\epsilon} \int_{\R^n} |f_{m_k} - f| \wedge 1 \, d\mu.
\end{align}
Since  $\{ f_{m_k} \}_{k = 1}^\infty$ is a subsequence of $\{ f_m \}_{m = 1}^\infty$, \begin{align}
    \label{eqn:universality:measure:modes_convergence:cb_2}
    \lim_{k \to \infty} \int_{\R^n}  |f_{m_k} - f| \wedge 1 \, d\mu = \lim_{m \to \infty} \int_{\R^n} |f_{m} - f| \wedge 1 \, d\mu.
\end{align}
Taking $\lim_{k \to \infty}$ on both sides of \ref{eqn:universality:measure:modes_convergence:cb_1} and applying \ref{eqn:universality:measure:modes_convergence:cb_2} gives
\begin{align}
    \label{eqn:universality:measure:modes_convergence:cb_3}
    \lim_{k \to \infty}   \mu(\{  |f_{m_k} - f| \wedge 1 > \epsilon \}) = 0.
\end{align}
Since $ |f_{m_k} - f| \wedge 1 \to 0$ in $\mu$ by \ref{eqn:universality:measure:modes_convergence:cb_3}, by \nameref{proposition:measure:convergence:mu_implies_subseqn_ae}, there exists a subsubsequence $|f_{m_{k_l}} - f| \wedge 1$ converging to $0$ $\mu$-almost everywhere. Thus, $|f_{m_{k_l}} - f| \to 0$ $\mu$-almost everywhere. By \nameref{proposition:measure:convergence:ae_implies_mu}, $|f_{m_{k_l}} - f|$ converges to $0$ in $\mu$. Therefore, $\mu(\{ |f_{m_{k_l}} - f| > \epsilon \} \to 0 $ as $l \to \infty$.
\end{step}
\end{proof}
\begin{remark}
Consequently, the convergence in probability measure is metrisable. By Proposition \ref{proposition:universality:measure:modes_convergence}, metrics $\delta_\mu$ and $\rho_\mu$ are equivalent. Moreover, they characterize the convergence in probability measure and induce the same topology on $\M^n$.
\end{remark}
 Before stating the next result, we will introduce the notation for closed balls in $\R^n$. Let $\vec{a} \in \R^n$ and $r > 0$. We denote the closed ball of radius $r$ centred at $\vec{a}$ by $B(\vec{a}, r) = \{ \vec{x} \in \R^n : \norm{\vec{x} - \vec{a}} \leq r \}.$ The following proposition provides a link between convergence on compacta and convergence in measure $\mu$.
\begin{proposition}[Uniform convergence on compacta implies convergence in $\mu$]
\label{proposition:universality:measure:convg_compacta_implies_mu}
Let $\{ f_m \}_{m=1}^\infty$ be a sequence of functions in $\M^n$ that converges uniformly to $f \in \M^n$ on compacta. In other words, for every compact set $K \subset \R^n$, \[ 
    \sup_{\vec{x} \in K} \{ | f_m (\vec{x}) - f(\vec{x}) | \} \to 0, \text{ as $m \to \infty$}.
\]
Then $\delta_\mu(f_m, f) \to 0$ as $m \to \infty$.
\end{proposition}
\begin{proof-idea*}
We will appeal to \nameref{proposition:universality:measure:modes_convergence}. Instead of directly proving $\delta_\mu(f_m, f) \to 0$ as $m \to \infty$, 
we will show $\rho_\mu(f_m, f) \to 0$ as $m \to \infty$. The main idea is to estimate the integral $\int_{\R^n} |f_m - f| \wedge 1 \, d\mu$ by integrating over a sufficiently large compact set $K$ where we can control the absolute value of the integrand and the complement of such a set. We will find such a compact set by writing $\R^n$ as an increasing union of compact sets, namely closed balls. The desired compact set will arise from continuity of $\mu$. After extracting the desired compact set, we will estimate above mentioned integrals and combine those estimates to complete the proof.
\end{proof-idea*}
\begin{proof}
By Proposition \ref{proposition:universality:measure:modes_convergence}, it is equivalent to show that $\rho_\mu(f_m, f) \to 0$ as $m \to \infty$. We need to show that \[
    \int_{\R^n} |f_m - f| \wedge 1 \, d\mu \to 0 \text{ as $m \to \infty$}.
\]
Let $\epsilon > 0$. We will estimate the integral $\int_{\R^n} |f_m - f| \wedge 1 \, d\mu$ by integrating over a sufficiently large compact set $K$ and its complement. We begin by constructing such a compact set $K$. For $k > 0$, consider $B(\vec{0}, k)$. By Heine-Borel Theorem, $B(\vec{0}, k)$ is compact. Note that $\R^n = \bigcup_{k=1}^\infty B(\vec{0}, k)$. Since for every $k \in \N$, $B(\vec{0}, k) \subset B(\vec{0}, k + 1)$, by continuity of the probability measure $\mu$, 
\begin{align}
    \label{eqn:universality:measure:convg_compacta_implies_mu:1}
    \mu (\R^n) = 1 = \lim_{k \to \infty} \mu (B(\vec{0}, k)).
\end{align}
By \ref{eqn:universality:measure:convg_compacta_implies_mu:1}, there exists $k_0 \in \N$ such that \[
1 - \frac{\epsilon}{2} < \mu (B(\vec{0}, k)) \leq 1, \text{ for every  $k \geq k_0$}.
\]
Set $K = B(\vec{0}, k_0)$. Then $\mu(K) > 1 - \frac{\epsilon}{2}$. Hence \begin{align}
    \label{eqn:universality:measure:convg_compacta_implies_mu:2}
    \mu (\R^n \setminus K) = \mu (\R^n) - \mu (K) < 1 - \left(1 - \frac{\epsilon}{2}\right) = \frac{\epsilon}{2}.
\end{align}
Write \begin{align}
    \label{eqn:universality:measure:convg_compacta_implies_mu:int_decomp}
    \int_{\R^n} | f_m - f | \wedge 1 \, d\mu =  \int_{K} | f_m - f | \wedge 1 \, d\mu  + \int_{\R^n \setminus K} | f_m - f | \wedge 1 \, d\mu. 
\end{align}
Consider $\int_{K} | f_m - f | \wedge 1 \, d\mu$. 
\newpage 
Since $\sup_{\vec{x} \in K} \{ | f_m (\vec{x}) - f(\vec{x}) \} \to 0$ as $m \to \infty$, there exists $M \in \N$ such that \begin{align}
    \label{eqn:universality:measure:convg_compacta_implies_mu:3}
    \sup_{\vec{x} \in K} \{ | f_m (\vec{x}) - f(\vec{x}) \} < \frac{\epsilon}{2}, \text{ for every $m \geq M$. }
\end{align}
Applying \ref{eqn:universality:measure:convg_compacta_implies_mu:3} gives
\begin{subequations}\label{eqn:universality:measure:convg_compacta_implies_mu:4}
\begin{align*}
    \int_{K} | f_m - f | \wedge 1 \, d\mu &\leq  \int_{K} | f_m - f |  \, d\mu \leq \int_{K} \sup_{\vec{x} \in K} \{ | f_m (\vec{x}) - f(\vec{x}) | \}\, d\mu \\
        &\leq \int_{\R^n} \frac{\epsilon}{2} \, d\mu = \frac{\epsilon}{2} \cdot \mu (\R^n) = \frac{\epsilon}{2}, \text{ for every $m \geq M$.}
     \tag{\ref{eqn:universality:measure:convg_compacta_implies_mu:4}} 
\end{align*}
\end{subequations}
Consider $\int_{\R^n \setminus K} | f_m - f | \wedge 1 \, d\mu$. Applying \ref{eqn:universality:measure:convg_compacta_implies_mu:2} gives \begin{align}
    \int_{\R^n \setminus K} | f_m - f | \wedge 1 \, d\mu \leq  \int_{\R^n \setminus K}  1 \, d\mu = \mu(\R^n \setminus K) < \frac{\epsilon}{2}.
\end{align}
By \ref{eqn:universality:measure:convg_compacta_implies_mu:int_decomp} and \ref{eqn:universality:measure:convg_compacta_implies_mu:4}, \ref{eqn:universality:measure:convg_compacta_implies_mu:2}, we have \begin{align*}
    \int_{\R^n} | f_m - f | \wedge 1 \, d\mu  < \frac{\epsilon}{2} + \frac{\epsilon}{2} = \epsilon, \text{ for every $m \geq M$. }
\end{align*}
\end{proof}
\subsection{Towards the Probabilistic Universal Approximation Theorem}
\label{subsection:universality:measure:towards_probabilistic_uap}

We will begin with a discussion of the uniform density of single-layer fully-connected neural networks on compacta in $\C(\R^n)$.
\begin{theorem}
\label{thm:universality:measure:uap:nnsdensecompacta}
Let $\mathcal{H}_{\sigma}$ denote the family of single-layer fully-connected neural networks with any continuous sigmoidal activation function $\sigma$, given by \begin{align*}
\mathcal{H}_{\sigma} = \left \{ \vec{x} \to \sum_{k=1}^{m} \alpha_k \sigma{\left (\langle \vec{w_k}, \vec{x} \rangle + \beta_k \right)} : m \in \N, \alpha_1 \ldots \alpha_m \in \R,  \beta_1 \ldots \beta_m \in \R, \vec{w_k} \in \R^n \right \}.
\end{align*}
Then $\mathcal{H}_{\sigma}$ is uniformly dense on compacta in $\C(\R^n)$.
\end{theorem}

\begin{proof-idea*}
By \nameref{thm:universality:cybenko}, for a fixed compact set $K \subset \R^n$, $\mathcal{H}_{\sigma}$ is dense on $\C(K)$.
However, we need to show that for every $f \in \C(\R^n)$, there exists a sequence $\{ h_m \}_{m=1}^\infty$, $h_m \in \mathcal{H}_{\sigma}$ such that $h_m \to f$ on every compact set $K \subset \R^n$. Informally, for every $f \in \C(\R^n)$, we need to exhibit a \textbf{single sequence } $\{ h_m \}_{m=1}^\infty$ converging to $f$ uniformly on every compact set $K \subset \R^n$. We will fix $f \in \C(\R^n)$. The key observation is that $\R^n$ can be covered with a countable union of nested compact sets. Using the \nameref{thm:universality:cybenko}, for every compact set, we will exhibit a sequence of neural networks from  $\mathcal{H}_{\sigma}$ converging to $f$ uniformly on the compact set. Using the fact our compact sets are nested, we will apply diagonalisation argument to produce a single sequence of neural networks from  $\mathcal{H}_{\sigma}$ converging to $f$ uniformly on the every compact set.
\end{proof-idea*}
\begin{proof}

Fix $f \in \C(\R^n)$. By Heine-Borel Theorem, for every $r > 0$, $B(\vec{0}, r)$ is compact. Observe that $f_{|  B(\vec{0}, r)} \in \C( B(\vec{0}, r))$, for every $r > 0$. Hence by \nameref{thm:universality:cybenko}, for every $r > 0$, there exists a sequence in $\mathcal{H}_{\sigma}$ denoted by $\{ h_{m}^{(r)} \}_{m = 1}^\infty$ such that $h_{m}^{(r)} \to f_{|  B(\vec{0}, r)}$ uniformly as $m \to \infty$. Hence $h_{m}^{(r)} \to f$ uniformly on $ B(\vec{0}, r)$, as $m \to \infty$. Consider following sequences.
\begin{align*}
    \textbf{h}_1^{(1)} &\quad& h_2^{(1)} &\quad& h_3^{(1)} &\quad& h_4^{(1)} &\quad& \ldots &\quad& \to f, \text{ uniformly on $B(\vec{0}, 1)$} \\
    h_1^{(2)}  &\quad& \textbf{h}_2^{(2)}  &\quad& h_3^{(2)}  &\quad& h_4^{(2)}  &\quad& \ldots  &\quad& \to f, \text{ uniformly on $B(\vec{0}, 2)$} \\
    h_1^{(3)}  &\quad& h_2^{(3)}  &\quad& \textbf{h}_3^{(3)}  &\quad& h_4^{(3)}  &\quad& \ldots  &\quad& \to f, \text{ uniformly on $B(\vec{0}, 3)$} \\ 
    h_1^{(4)}  &\quad& h_2^{(4)}  &\quad& h_3^{(4)}  &\quad& \textbf{h}_4^{(4)}  &\quad& \ldots  &\quad& \to f, \text{ uniformly on $B(\vec{0}, 4)$} \\ 
    \vdots  &\quad& \vdots &\quad& \vdots  &\quad& \vdots &\quad& \ddots &\quad& \vdots
\end{align*}
Clearly, for every $r > 0$, $B(\vec{0}, r) \subset B(\vec{0}, r + 1)$. Thus, for every $r > 0$,  \begin{align}
    \label{eqn:thm:universality:measure:uap:nnsdensecompacta:1}
    h_m^{(r)} \to f \text{ uniformly on every $B(\vec{0}, r')$, as $m \to \infty$}, \text{ for every $r' \leq r$.}
\end{align}
Define $\{ h_m \}_{m = 1}^\infty$ by $h_m = h_m^{(m)}$ for every $m \in \N$.
Note that $\R^n = \bigcup_{r = 1}^\infty B(\vec{0}, r)$. Let $K \subset \R^n$ be a compact set. Since $ B(\vec{0}, r) \uparrow \R^n$ as $r \to \infty$, there exists some $r_0 \in \N$ such that \begin{align}
    \label{eqn:thm:universality:measure:uap:nnsdensecompacta:2}
    K \subseteq B (\vec{0}, r_0) \subset  B (\vec{0}, r_0 + 1) \subset  B (\vec{0}, r_0 + 2) \ldots
\end{align}
Assume, without loss of generality, $r_0$ is the smallest such a natural number.
By \ref{eqn:thm:universality:measure:uap:nnsdensecompacta:2} and \ref{eqn:thm:universality:measure:uap:nnsdensecompacta:1}, $\{ h_m \}_{m \geq r_0} \to f$ uniformly on $B(\vec{0}, r_0)$ as $m \to \infty$. Since $K \subseteq B(\vec{0}, r_0)$ and $K$ was arbitrary, the proof is complete.
\end{proof}
The following two lemmas are auxiliary results that will be help us establish the Proposition \ref{proposition:universality:measure:continuous_dense_mn} - the most important result in this subsection.
\begin{lemma}
\label{lemma:universality:measure:uap:approx_characteristic}
Let $A \in \B(\R^n)$. Suppose that $\mu$ is a probability measure on $\B(\R^n)$. Then for every $\epsilon > 0$, there exist a closed set $F \in \B(\R^n)$ and a continuous function $g \in \C(\R^n)$ such that $g$ and $\chi_A$ agree on $F$ and $\mu (\R^n \setminus F) < \epsilon$.
\end{lemma}
\begin{proof-idea*}
The proof relies on the regularity of the probability measure $\mu$ and Urysohn lemma. By appealing to the regularity of $\mu$, we will extract the compact set approximating $A$ to desired accuracy in measure $\mu$. The existence of a desired function $g$ will follow from \nameref{lemma:topology:urysohn}, stated below.
\begin{lemma*}[Urysohn lemma, Theorem 33.1 \cite{munkres_2014_topology}]
\label{lemma:topology:urysohn}
Let $X$ be a normal space. Let $A, B$ be disjoint closed subsets of $X$. There exists a continuous map $f : X \to [a,b]$ such that $f(x) = a$, for every $a \in A$ and $f(x) = b$, for every $b \in B$.
\end{lemma*}
We will apply \nameref{lemma:topology:urysohn} to the compact set approximating $A$ in measure $\mu$.
\end{proof-idea*}
\pagebreak
\begin{proof}
Since $\mu$ is a probability measure on $(\R^n, \B(\R^n))$, by \nameref{proposition:measure:regularity_borel_finite}, $\mu$ is regular. Hence, there exist a compact set $K$ and an open set $U$ such that $K \subset A \subset U$ satisfying \begin{align}
    \label{eqn:universality:measure:uap:approx_characteristic:1}
    \mu(A \setminus K ) < \frac{\epsilon}{2} \text{ and } \mu (U \setminus A) < \frac{\epsilon}{2}.
\end{align}
Since $\R^n$ with the standard topology is a metric space, $\R^n$ is Hausdorff and normal. Since $K$ is compact and $\R^n$ Hausdorff, $K$ is closed. Since $U$ is open, $\R^n \setminus U$ is closed. Since $K \subset U$, $K \cap (\R^n \setminus U) = \emptyset$. By \nameref{lemma:topology:urysohn}, there exists $g \in \C(\R^n)$ such that $0 \leq g \leq 1$ and $g = 1$ on $K$ while $g = 0$ on $\R^n \setminus U$. Set $F = K \cup (\R^n \setminus U)$. Since $K$ and $ (\R^n \setminus U)$ are closed, $F$ is closed. We will show that $\chi_A$ and $g$ agree on $F$. Suppose that $\vec{x} \in F$. Since $K \cap (\R^n \setminus U) = \emptyset$, either $\vec{x} \in K$ or $\vec{x} \in (\R^n \setminus U)$. If $\vec{x} \in K$, then $\chi_{K}(\vec{x}) = 1 = g(\vec{x})$. If $\vec{x} \in (\R^n \setminus U)$,
then $\chi_{K}(\vec{x}) = 0$ since $K \subset U$. Hence $\chi_{K}(\vec{x}) = 0 = g(\vec{x})$. Therefore, $\chi_{A_{| F}} = g$. Hence \begin{align*}
        \mu (\R^n \setminus F) &= \mu ((\R^n \setminus K) \cap U) = \mu (U \setminus K) \\
                               &= \mu (A \setminus K) + \mu(U \setminus A) \\
                               &< \frac{\epsilon}{2} + \frac{\epsilon}{2} = \epsilon, \text{ by \ref{eqn:universality:measure:uap:approx_characteristic:1}} . 
\end{align*}
\end{proof}
The following lemma is a natural generalization of Lemma \ref{lemma:universality:measure:uap:approx_characteristic} to simple, $\B(\R^n)$-measurable functions.
\begin{lemma}
\label{lemma:universality:measure:uap:approx_simple}
Let $\varphi$ be a simple, $\B(\R^n)$-measurable function. Suppose that $\mu$ is a probability measure on $\B(\R^n)$. Then for every $\epsilon > 0$, there exist a closed set $F \in \B(\R^n)$ and a continuous function $g \in \C(\R^n)$ such that $\varphi$ and $g$ agree on $F$ and $\mu (\R^n \setminus F) < \epsilon$.
\end{lemma}
\begin{proof}
Write $\varphi = \sum_{k = 1}^m a_k \chi_{A_{k}}$ where for every $k \in \{ 1, 2, \ldots, m \}$,
$a_k \in \R$ and $A_k \in \B(\R^n)$. Without loss of generality, for every $i, j \in  \{ 1, 2, \ldots, m \}, i \neq j$, $A_i \cap A_j = \emptyset$.
By Lemma \ref{lemma:universality:measure:uap:approx_characteristic}, for every $k \in \{ 1, 2, \ldots, m \}$, there exist a closed set $F_k \in \B(\R^n)$ and a continuous function $g_k \in \C(\R^n)$ such that
\begin{align}
    \label{eqn:universality:measure:uap:approx_simple:1}
    \text{for every } \vec{x} \in F, \chi_{A_k} (\vec{x}) = g_k (\vec{x}) \text{ and } \mu (\R^n \setminus F_k) < \frac{\epsilon}{m}.
\end{align}
Set $F = \bigcap_{k=1}^m F_k$. Define $g : \R^n \to \R$ by $g = \sum_{k = 1}^m a_k g_k$. Since $g$ is a sum of continuous functions, $g \in \C(\R^n)$. We claim that $\varphi$ and $g$ agree on $F$. Observe
\begin{subequations}\label{eqn:universality:measure:uap:approx_simple:2}
\begin{align*}
    \vec{x} \in F &\iff x \in F_k, \text{ for every $k \in \{ 1, 2, \ldots, m \} $} \\
                  &\implies \chi_{A_{k}}(\vec{x}) = g_k(\vec{x}), \text{ for every $\vec{x} \in F_k$, } \text{for every $k \in \{ 1, 2, \ldots, m \} $} \\ 
                  &\implies a_k  \chi_{A_{k}}(\vec{x}) = a_k g_k(\vec{x}),  \text{ for every $\vec{x} \in F_k$, } \text{for every $k \in \{ 1, 2, \ldots, m \} $}
     \tag{\ref{eqn:universality:measure:uap:approx_simple:2}}
\end{align*}
\end{subequations}
Since for every $k \in \{ 1, 2, \ldots, m \} $, $F \subseteq F_k$, by \ref{eqn:universality:measure:uap:approx_simple:2}, \begin{align}
    \label{eqn:universality:measure:uap:approx_simple:3}
     a_k  \chi_{A_{k}}(\vec{x}) = a_k g_k(\vec{x}), \text{ for every $\vec{x} \in F$}, \text{ for every $k \in \{ 1, 2, \ldots, m \} $}.
\end{align}
Summing over \ref{eqn:universality:measure:uap:approx_simple:3} gives \begin{align}
    \varphi(\vec{x}) = \sum_{k = 1}^m a_k \chi_{A_{k}}(\vec{x}) =  \sum_{k = 1}^m a_k g_k(\vec{x}) = g(\vec{x}), \text{ for every $\vec{x} \in F$.}
\end{align}
It remains to prove that $\mu(\R^n \setminus F) < \epsilon$. By definition of $F$ and \ref{eqn:universality:measure:uap:approx_simple:1}, \begin{align}
    \mu(\R^n \setminus F) = \mu \left (\bigcup_{k=1}^m (\R^n \setminus F_k) \right) \leq \sum_{k = 1}^m \mu (\R^n \setminus F_k) < \sum_{k = 1}^m \frac{\epsilon}{m} = \epsilon.
\end{align}
\end{proof}
The following proposition is the most important result in this section and it will play essential role in the proof of \nameref{thm:universality:measure:uap:probabilistic}.
\begin{proposition}
\label{proposition:universality:measure:continuous_dense_mn}
$\mathcal{C}(\R^n)$ is $\rho_\mu$-dense in $\M^n$.
\end{proposition}
\begin{proof-idea*} The argument will be divided in four steps. We begin by showing that $f$ is bounded, except possibly on a set of arbitrarily small measure.
Using this fact, we will construct a bounded function $h \in \M^n$ which is sufficiently close to $f$ in $\rho_\mu$ metric. Since $h$ is a bounded measurable function, there exists a sequence of measurable simple functions $\{ \varphi_m \}_{m=1}^\infty$ such that $\varphi_m \to h$ pointwise. This will enable us to pick a simple function $\varphi$ which is sufficiently close to $h$ in $\rho_\mu$ metric. Using Lemma \ref{lemma:universality:measure:uap:approx_simple}, we will extract a continuous function $g$ which is sufficiently close to $\varphi$ in $\rho_\mu$ metric. We will complete the proof by putting those estimates together.  
\end{proof-idea*}
\begin{proof}
Let $f \in \M^n$ and let $\epsilon > 0$. We will show that there exists $g \in \C(\R^n)$ such that $\rho_\mu (f, g) < \epsilon$.
\setcounter{step}{0}
\begin{step}[$f$ is almost bounded]
Firstly, we will argue that $f$ is bounded, except possibly on a set of arbitrarily small measure. Recall that $f \in \M^n$ implies that $f$
is $\mu$-almost everywhere finite. This implies \begin{align}
    \label{eqn:universality:measure:continuous_dense_mn:1}
    \mu(\{ |f| = \infty \}) = 0.
\end{align}
Observe that $\{ |f| = \infty \} = \bigcap_{m=1}^\infty \{ |f| > m \}$. Since $\{ |f| > m \} = f^{-1}(-\infty, -m) \cup f^{-1}(m, \infty)$ and $f$ is $\B(\R^n)$-measurable, $\{ |f| > m \} \in \B(\R^n)$. Clearly, $\{ |f| > m + 1\} \subseteq \{ |f| > m \}$, for every $m \in \N$. By continuity of $\mu$ and \ref{eqn:universality:measure:continuous_dense_mn:1}, 
\begin{align}
    \label{eqn:universality:measure:continuous_dense_mn:2}
    \mu(\{ |f| = \infty \}) = 0 = \mu \left( \bigcap_{m=1}^\infty \{ |f| > m \} \right) = \lim_{m \to \infty} \mu ( \{ |f| > m \}).
\end{align}
By \ref{eqn:universality:measure:continuous_dense_mn:2}, there exists $N \in \N$ such that \begin{align}
    \label{ineqn:universality:measure:continuous_dense_mn:3}
    \mu (\{ |f| > m \}) < \frac{\epsilon}{2}, \text{ for every $m \geq N$.}
\end{align}
By \ref{ineqn:universality:measure:continuous_dense_mn:3}, in particular, $\mu (\{ |f| > N \}) < \frac{\epsilon}{2}$.
\end{step}
\begin{step}[Approximate $f$ with a bounded function $h$]
Define $h : \R^n \to \R$ by $h = f \chi_{\{ |f| \leq N \}}$. Since $f \in \M^n$ and $\{ |f| \leq N \}$ is measurable, $h \in \M^n$. By definition of $h$, we have \begin{align}
    \label{eqn:universality:measure:continuous_dense_mn:4}
    | f(\vec{x}) - h(\vec{x}) | = \begin{cases}
      0 & \text{ if $|f(\vec{x})| \leq N$ } \\
      |f(\vec{x})| & \text{ otherwise }
    \end{cases}.
\end{align}
\begin{subequations}\label{ineqn:universality:measure:continuous_dense_mn:8}
\begin{align*}
    \rho_\delta(f, h) = \int_{\R^n} |f - h | \wedge 1 \, d\mu &= \int_{\{ |f| \leq N \}} |f - h | \wedge 1 \, d\mu +  \int_{\{ |f| > N \}} |f - h | \wedge 1 \, d\mu & \quad \\
    &= \int_{\{ |f| > N \}} |f - h | \wedge 1 \, d\mu  \text{ by \ref{eqn:universality:measure:continuous_dense_mn:4} } &\\
    &\leq \int_{\{ |f| > N \}} 1 \, d\mu &  \quad & \\
    &=\mu(\{ |f| > N \}) < \frac{\epsilon}{2}.  \text{ by \ref{ineqn:universality:measure:continuous_dense_mn:3}}
     \tag{\ref{ineqn:universality:measure:continuous_dense_mn:8}} 
\end{align*}
\end{subequations}
\end{step}
\begin{step}[Approximate $h$ with a simple function $\varphi$]
Since $h \in \M^n$, by there exists a sequence of measurable simple functions $\{ \varphi_m \}_{m=1}^\infty$ such that $\varphi_m \to h$ pointwise. Since $|h| \leq N$, without loss of generality, assume that $|\varphi_m| \leq N$, for every $m \in N$. We claim $\rho_\mu(\varphi_m, h) \to 0$ as $m \to \infty$. We aim to apply \nameref{thm:dct}. Observe that for every $\vec{x} \in \R^n$, \begin{align}
    \label{ineqn:universality:measure:continuous_dense_mn:4}
     | h(\vec{x}) - \varphi_m(\vec{x}) | \wedge 1 \leq | h(\vec{x}) -  \varphi_m(\vec{x}) | \leq  | h(\vec{x})| +  |  \varphi_m (\vec{x}) | \leq N + N = 2N.
\end{align}
By \ref{ineqn:universality:measure:continuous_dense_mn:4}, $\vec{x} \to 2N$ is $\mu$-integrable and dominates $|h - \varphi_m| \wedge 1$. Hence, by \nameref{thm:dct}, \begin{align}
     \label{eqn:universality:measure:continuous_dense_mn:5}
     \lim_{m \to \infty} \rho_\mu (h, \varphi_m) = \lim_{m \to \infty} \int_{\R^n} | h  -  \varphi_m | \wedge 1 \, d\mu =  \int_{\R^n} \lim_{m \to \infty} | h  -  \varphi_m | \wedge 1 \, d\mu = 0.
\end{align}
By \ref{eqn:universality:measure:continuous_dense_mn:5}, there exists $M \in \N$ such that \begin{align}
    \label{ineqn:universality:measure:continuous_dense_mn:6}
    \rho_\mu(h, \varphi_m) = \int_{\R^n} | h  -  \varphi_m | \wedge 1 \, d\mu < \frac{\epsilon}{4}, \text{ for every $m \geq M$}.
\end{align}
\end{step}
Set $\varphi = \varphi_M$. In particular, by \ref{ineqn:universality:measure:continuous_dense_mn:6}, 
\begin{align}
    \label{ineqn:universality:measure:continuous_dense_mn:11}
    \rho_\mu (h, \varphi) < \frac{\epsilon}{4}.
\end{align}
\begin{step}[Approximate $\varphi$ with a continuous function $g$]
By Lemma \ref{lemma:universality:measure:uap:approx_simple}, there exist a closed set $F \in \B(\R^n)$ and a continuous function $g \in \C(\R^n)$ such that \begin{align}
    \label{eqn:universality:measure:continuous_dense_mn:12}
    \varphi \text{ and } g \text{ agree on $F$} \text{ and } \mu (\R^n \setminus F) < \frac{\epsilon}{4}.
\end{align}
Then the following computation yields the desired estimate,
\begin{subequations}\label{ineqn:universality:measure:continuous_dense_mn:7}
\begin{align*}
    \rho_\mu(\varphi, g) = \int_{\R^n} | \varphi - g | \wedge 1 \, d\mu &= \int_{F} | \varphi - g | \wedge 1 \, d\mu +  \int_{\R^n \setminus F} | \varphi - g | \wedge 1 \, d\mu & \\
                                                 &= \int_{\R^n \setminus F} | \varphi - g | \wedge 1 \, d\mu \text{ by \ref{eqn:universality:measure:continuous_dense_mn:12}}\\
                                                 &\leq \int_{\R^n \setminus F}  1 \, d\mu  \\
                                                 &= \mu(\R^n \setminus F) < \frac{\epsilon}{4}. \text{ by \ref{eqn:universality:measure:continuous_dense_mn:12}}
     \tag{\ref{ineqn:universality:measure:continuous_dense_mn:7}} 
\end{align*}
\end{subequations}
\end{step}
\begin{step}[Putting estimates regarding $f, h, \varphi, g$ together.] We will show that $g$ is a desired continuous function.
We begin by proving that $\rho_\mu (h, g) < \frac{\epsilon}{2}$. Since $\rho_\mu$ is a metric, by \ref{ineqn:universality:measure:continuous_dense_mn:11} and \ref{ineqn:universality:measure:continuous_dense_mn:7}, \begin{align}
    \label{ineqn:universality:measure:continuous_dense_mn:9}
    \rho_\mu(h, g) \leq \rho_\mu(h, \varphi) + \rho_\mu(\varphi, g) < \frac{\epsilon}{4} + \frac{\epsilon}{4} = \frac{\epsilon}{2}.
\end{align}
Again, applying the triangle inequality of $\rho_\mu$ combined with estimates \ref{ineqn:universality:measure:continuous_dense_mn:8} and \ref{ineqn:universality:measure:continuous_dense_mn:9} yields 
\begin{align*}
    \rho_\mu (f, g) \leq \rho_\mu (f, h) + \rho_\mu (h, g) < \frac{\epsilon}{2} + \frac{\epsilon}{2} = \epsilon.    
\end{align*}
\end{step}
\end{proof}
\begin{corollary}
\label{corollary:universality:measure:continuous_dense_mn}
$\mathcal{C}(\R^n)$ is $\delta_\mu$-dense in $\M^n$.
\end{corollary}
\begin{proof}
Follows directly from Proposition \ref{proposition:universality:measure:continuous_dense_mn} and \nameref{proposition:universality:measure:modes_convergence}.
\end{proof}


\subsection{The Probabilistic Universal Approximation Theorem}
\label{subsection:universality:measure:probabilistic_uap}
We are ready to prove \nameref{thm:universality:measure:uap:probabilistic}.
\begin{theorem}[The Probabilistic Universal Approximation Theorem]
\label{thm:universality:measure:uap:probabilistic}
Let $\mathcal{H}_{\sigma}$ denote the family of single-layer fully-connected neural networks with any continuous sigmoidal activation function $\sigma$, given by \begin{align*}
\mathcal{H}_{\sigma} = \left \{ \vec{x} \to \sum_{k=1}^{m} \alpha_k \sigma{\left (\langle \vec{w_k}, \vec{x} \rangle + \beta_k \right)} : m \in \N, \alpha_1 \ldots \alpha_m, \beta_1 \ldots \beta_m \in \R, \vec{w_k} \in \R^n \right \}.
\end{align*}
Then $\mathcal{H}_{\sigma}$ is $\delta_\mu$-dense in $\M^n$.
\end{theorem}
\begin{proof}
Let $f \in \M^n$ and let $\epsilon > 0$. By Corollary \ref{corollary:universality:measure:continuous_dense_mn}, there exists $g \in \C(\R^n)$ such that $\delta_\mu(f, g) < \frac{\epsilon}{2}$. By Theorem \ref{thm:universality:measure:uap:nnsdensecompacta}, there exists a sequence of neural networks $\{ h_m \}_{m=1}^\infty$ from $\mathcal{H}_\sigma$ such that $h_m \to g$ uniformly on every compact set $K \subset \R^n$. By Proposition \ref{proposition:universality:measure:convg_compacta_implies_mu}, $h_m \to g$ in $\delta_\mu$. Hence there exists $M \in \N$ such that $\delta_\mu(h_m, g) < \frac{\epsilon}{2}$, for every $m \geq M$. Since $\delta_\mu$ is a metric, applying estimates above yields \[
  \delta_\mu(f, h_M) \leq \delta_\mu(f, g) + \delta_\mu(g, h_M) < \frac{\epsilon}{2} + \frac{\epsilon}{2} = \epsilon.
\]
\end{proof}
\begin{remark}
By \nameref{proposition:universality:measure:modes_convergence}, density in $\delta_\mu$ is equivalent to density in probability measure $\mu$.
Informally, a neural network from $\mathcal{H}_{\sigma}$ can approximate any Borel function on $\R^n$ up to desired accuracy, except possibly on set of arbitrarily small probability measure. This interpretation is stated formally as the following corollary of \nameref{thm:universality:measure:uap:probabilistic}.
\end{remark}

\begin{corollary}
Let $\mathcal{H}_{\sigma}$ denote the family of single-layer fully-connected neural networks with any continuous sigmoidal activation function $\sigma$, given by \begin{align*}
\mathcal{H}_{\sigma} = \left \{ \vec{x} \to \sum_{k=1}^{m} \alpha_k \sigma{\left (\langle \vec{w_k}, \vec{x} \rangle + \beta_k \right)} : m \in \N, \alpha_1 \ldots \alpha_m, \beta_1 \ldots \beta_m \in \R, \vec{w_k} \in \R^n \right \}.
\end{align*}
Then for every $f \in \M^n$, for every $\epsilon > 0$, there exists $h \in \mathcal{H}_\sigma$ such that \[
    \mu (\{ | f - h | > \epsilon \}) < \epsilon.
\]
\end{corollary}
\begin{proof}
By \nameref{thm:universality:measure:uap:probabilistic}, $\mathcal{H}_{\sigma}$ is $\delta_\mu$-dense in $\M^n$.
By \nameref{proposition:universality:measure:modes_convergence}, density in $\delta_\mu$ is equivalent to density in probability measure $\mu$.
The result follows directly from definition of convergence in measure.
\end{proof}


\subsection{Relationship between measurable functions and classification}
\label{subsection:universality:measure:relationship}
We will briefly discuss the relationship between measurable functions and classification, using the argument from the discussion following Theorem 2 in \cite{cybenko_1989_approximation}.

Let $\lambda$ denote the restriction of Lebesgue measure to $[0,1]^n$. Let $\{ P_k \}_{k=1}^m$ be a partition of $[0,1]^n$ into disjoint, $\lambda$-measurable subsets of $[0,1]^n$. We can view the classification problem on $[0,1]^n$ as a problem of approximating a classification function $f : [0,1]^n \to \{ 1, 2, \ldots, m \}$, given by \[ 
    f (\vec{x}) = k \iff \vec{x} \in P_k.
\]
\nameref{thm:universality:cybenko:measurable} implies that neural networks with continuous sigmoidal activation functions can approximate classification functions to desired accuracy, except possibly on sets of arbitrarily small measure.