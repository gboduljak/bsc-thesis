\subsection{Appendix}

\subsubsection{\nameref{section:appendix:stone-weierstrass}}
Definitions and proofs in this section are based on Chapter 27 of \cite{ross_2015_elementary} and Chapter 10 of \cite{wade_2014_introduction}. Proof of \nameref{thm:anal:bernstein-approx} is a proof of Theorem 27.4 in \cite{ross_2015_elementary}. \nameref{thm:anal:weierstrass-approx} is Theorem 27.5 in \cite{ross_2015_elementary} whose proof was omitted in the textbook and provided in this thesis. \nameref{lemma:top:closurelattice} is Lemma 10.67 in \cite{wade_2014_introduction}. The proof of \nameref{lemma:top:closurelattice} is an adaption of the proof of Lemma 10.67 in \cite{wade_2014_introduction}. Instead of approximating $|t|$ with the binomial series as in \cite{wade_2014_introduction}, the approximation of $|t|$ was proved using \nameref{thm:anal:weierstrass-approx}. The proof of \nameref{thm:anal:stone-weierstrass} is based on the proof of Theorem 10.69 from \cite{wade_2014_introduction}.

\subsubsection{\nameref{section:appendix:measure}}
Definitions and proofs in this section are based on various chapters from \cite{cohn_2013_measure}
and \cite{bass2011real}. The proof of \nameref{lemma:negative-set} is the proof of Lemma 4.1.4 in \cite{cohn_2013_measure}. The proof of \nameref{thm:hahn-decomp-thm} is an adaption of the proof of Theorem 4.1.5 in \cite{cohn_2013_measure}. The proof of \nameref{thm:hahn-jordan} is the proof of Corollary 4.1.6 in \cite{cohn_2013_measure}. The proof of \nameref{thm:radon-nikodym} is based on the proof of Theorem 4.2.2 in \cite{cohn_2013_measure}. The reduction to $\sigma$-finite case is outlined in \cite{cohn_2013_measure} and completed in the proof of \nameref{thm:radon-nikodym}. The statement of \nameref{thm:radon-nikodym-signed} is based on the statement of Theorem 4.2.4 in \cite{cohn_2013_measure}.

\subsubsection{\nameref{section:appendix:functional-analysis}}
Definitions are based on Chapter 4 from \cite{rynne_2008_linear} and the proof of the \nameref{thm:funct:hahn-banach} is a combination of Theorem 5.6 in \cite{folland_1999_real} and the proof of Theorem 5.13 from \cite{rynne_2008_linear}. However, significant parts of the proof are not discussed in detail in those textbooks. For instance, the fact $\prec$ is a partial order is not proved, the construction of the maximal element on $\Omega$ and verification that it is maximal is omitted. All such issues are clarified in the proof of \nameref{thm:funct:hahn-banach} presented in this thesis. The proof of \nameref{thm:funct:hahn-jordan-cx} is an adaption of the proof of Lemma 8.13 in \cite{bartle_measure_integration}. The result in this thesis addresses bounded linear functionals on $\C(X)$ while Lemma 8.13 addresses $\Lp$. Another similar argument is the proof of Proposition 17.7 in \cite{bass2011real}.

\subsubsection{\nameref{section:appendix:lp}}
Definitions are based on Chapter 15 from \cite{bass2011real} and Section 3.3 from \cite{cohn_2013_measure}. 
The proof of Lemma \ref{lemma:lp:funct_norm} is a combination of proofs of Theorem 15.9 from \cite{bass2011real}, Corollary 15.10 in \cite{bass2011real} and Proposition 15.11 from \cite{bass2011real}. The proof of \nameref{thm:lp:rrt} is based on the proof of Theorem 15.12 from \cite{bass2011real}. However, many subtle parts of the argument are different. For instance, verification that $\nu$ is a signed measure is proved in a different way. The generalization from simple functions to $\Lp$ is performed using a different argument based on Lemma \ref{lemma:lp:funct_norm}.

\subsubsection{\nameref{section:appendix:functionals-cx}}
Definitions in this section are based on \cite{bass2011real} and \cite{munkres_2014_topology}. The proof of Lemma \ref{lemma:fcs:partsunity} is an adaptation of the proof of Proposition 17.2 in \cite{bass2011real}. The proof of Lemma \ref{thm:fcs:approximation} is the proof of Proposition 17.6 in \cite{bass2011real}. However, many details have been added. The proof of Corollary \ref{cor:fcs:regularity} is a straightforward application of Lemma \ref{thm:fcs:approximation}. The proof of \nameref{thm:fcs:rrt-positive} is based on the proof of Theorem 17.3 in \cite{bass2011real}. However, the argument is considerably more detailed. For instance, Theorem 17.3 in \cite{bass2011real} does not address uniqueness and regularity of the resulting measure. Both uniqueness and regularity are proved in this thesis. The proof of \nameref{thm:fcs:rrt-bounded} is based on the proof of Theorem 17.8 in \cite{bass2011real}. However, Theorem 17.8 in \cite{bass2011real} does not discuss uniqueness of the finite signed measure $\mu$. In this thesis, the uniqueness is also proved.

\subsubsection{\nameref{section:appendix:fourier}}
This section is based on Chapters 15 and 16 from \cite{bass2011real}.
The proof of \nameref{prop:fourier:convnorm} is based on the proof of Proposition 15.7 in \cite{bass2011real}.
The proof of \nameref{lemma:fourier:approxident} is the proof of Proposition 16.6 in \cite{bass2011real}.
The proof of \nameref{prop:fourier:gaussian:ft} is based on the discussion in Example 5 (p.107) in \cite{jacod_2004_probability}.
The proof of \nameref{thm:fourier:inversionlone} is based on the proof of Theorem 16.7 in \cite{bass2011real}. The proof in \cite{bass2011real} ends with the fact that $f \ast \widehat{h}_a$ converges to $f$ in $\norm{\cdot}_1$. However, the claim is about the convergence pointwise almost everywhere. The proof in this thesis justifies that the convergence  of $f \ast \widehat{h}_a$ to $f$ in $\norm{\cdot}_1$ implies the convergence of $f \ast \widehat{h}_a$ to $f$ almost everywhere under the conditions relevant to the theorem.