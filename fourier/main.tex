\section{Fourier Analysis}
\label{section:appendix:fourier}
In this section, we will introduce the Fourier Transform and state its main properties. Fourier Transform played an essential role in proving that certain activation functions were discriminatory. In previous sections, we have been working with real-valued functions. In this section, we will assume that $\Lone(\R^n)$ consists of complex-valued functions. In other words, we write $\Lp(\R^n), 1 \leq p < \infty$ for the normed linear space of complex-valued, Borel-measurable functions on $\R^n$ such that
\begin{align*}
    \norm{f}_p = \left(\int_{\R^n} |f|^p \, d\lambda \right )^{\frac{1}{p}} < \infty,
\end{align*}
where two functions are identified if they are $\lambda$-almost everywhere equivalent. Here, $\lambda$ is the $n$-dimensional Lebesgue measure. For the sake of simplicity, we will simply refer to representative functions of those equivalence classes instead
of equivalence classes themselves. Also, it is worth noting that we presented integration theorems only in terms of real-valued functions. However, by definition of the integral of a complex-valued function, all these results generalize to complex-valued functions and we may use them in this section.
\subsection{Fourier Transform on $\Lone(\R^n)$}
We begin by introducing the Fourier Transform on $\Lone(\R^n)$.
\begin{definition}[Fourier Transform on $\Lone(\R^n)$]
Let $f \in \Lone(\R^n)$. We define the Fourier Transform of $f$, denoted $\widehat{f} : \R^n \to \mathbb{C}$, by 
\begin{align*}
    \widehat{f} (\vec{u}) = \int_{\R^n} e^{i \langle \vec{u}, \vec{x} \rangle } f(\vec{x}) \, d \lambda(\vec{x}).
\end{align*}
\begin{remark}
Some authors put a negative sign before the inner product $\langle \vec{u}, \vec{x} \rangle$ or introduce a scaling factor of $(2 \pi)^{-1}$ or $(2 \pi)^{-\frac{1}{2}}$ before the integral. I decided to follow the convention in \cite{bass2011real} for the sake of consistency.
\end{remark}
\begin{remark}
Because in this section we work only with the Lebesgue measure, we will write $\int_{\R^n} e^{i \langle \vec{u}, \vec{x} \rangle } f(\vec{x}) \, d \vec{x}$ instead of $ \int_{\R^n} e^{i \langle \vec{u}, \vec{x} \rangle } f(\vec{x}) \, d \lambda(\vec{x})$.
\end{remark}
The Fourier Transform satisfies various convenient algebraic properties.
\begin{proposition}[Properties of the Fourier Transform on $\Lone(\R^n)$]
\label{prop:fourier:basics}
Suppose $f$ and $g$ are in $\Lone(\R^n)$. Then 
\begin{enumerate}[noitemsep]
    \item $\widehat{f}$ is bounded and continuous;
    \item $\widehat{(f + g)}(\vec{u}) = \widehat{f}(\vec{u})+ \widehat{g}(\vec{u})$;
    \item for every $a \in \mathbb{C}$, $\widehat{(af)}(\vec{u}) = a\widehat{f}(\vec{u})$;
    \item if $\vec{a} \in \R^n$ and $f_\vec{a} (\vec{x}) = f (\vec{x} + \vec{a})$ then $\widehat{f_\vec{a}}(\vec{u}) = e^{-i \langle \vec{u}, \vec{a} \rangle} \widehat{f}(\vec{u})$;
    \item if $\vec{a} \in \R^n$ and $g_\vec{a} (\vec{x}) =  e^{i \langle \vec{a}, \vec{x}  \rangle} g(\vec{x})$ then $\widehat{g_\vec{a}}(\vec{u}) = \widehat{g} (\vec{u} + \vec{a})$;
    \item if $a \in \R, a \neq 0$ and $h_a (x) = f (a \vec{x})$ then $\widehat{h_{a}}(\vec{u}) = a^{-n} \widehat{f}(\frac{\vec{u}}{a})$.
\end{enumerate}
\end{proposition}
\begin{proof}
Continuity follows from \nameref{thm:dct}. All other results are elementary calculations justified by the linearity of the integral or the change of variables. See Proposition 16.1 in \cite{bass2011real} for the details.
\end{proof}
\end{definition}
\subsection{Convolution on $\Lone(\R^n)$}
An important operation related to integral transforms is a convolution.
\begin{definition}[Convolution on $\Lone(\R^n)$]
Let $f, g \in \Lone(\R^n)$. The convolution of $f$ with $g$ is a function $(f \ast g) : \R^n \to \mathbb{C}$ given by
\begin{equation*}
    (f \ast g) (\vec{x}) = \int_{\R^n} f(\vec{x} - \vec{y})g(\vec{y}) \,d \vec{y}.
\end{equation*}
\end{definition}

\begin{lemma}[Convolution Norm Lemma]
\label{prop:fourier:convnorm}
Let $f, g \in \Lone(\R^n)$. Then $(f \ast g)$ is integrable and \begin{align*}
    \norm{f \ast g}_1 \leq \norm{f}_1 \norm{g}_1. 
\end{align*}
\end{lemma}
\begin{proof} We proceed with a direct calculation,
     \begin{align*}
        \norm{(f \ast g)}_1 &= \int_{\R^n} \left | \int_{\R^n} f(\vec{x} - \vec{y})g(\vec{y}) \,d \vec{y} \right | \,d \vec{x} & \\
                                      &\leq \int_{\R^n} \left (  \int_{\R^n} \left | f(\vec{x} - \vec{y})g(\vec{y}) \right | \,d \vec{y} \right )  \,d \vec{x} & \\
                                      &\leq \int_{\R^n} \left ( \int_{\R^n} | f(\vec{x} - \vec{y})| |g(\vec{y}) | \,d \vec{x} \right)  \,d \vec{y} \text{ by \nameref{thm:measure:fubini}} \\
                                      &\leq \int_{\R^n} |g(\vec{y}) |  \left ( \int_{\R^n} | f(\vec{x} - \vec{y})| \,d \vec{x} \right)  \,d \vec{y} &  \\
                                      &\leq \int_{\R^n} |g(\vec{y}) |  \left ( \int_{\R^n} | f(\vec{t})| \,d \vec{t} \right)  \,d \vec{y} & \\
                                      &\leq \left (\int_{\R^n} |g(\vec{y})| \,d \vec{y} \right) \left( \int_{\R^n} | f(\vec{t})| \,d \vec{t} \right) = \norm{f}_1 \cdot \norm{g}_1.
    \end{align*}
The application of \nameref{thm:measure:fubini} was permissible because the integrand is non-negative. Since $f, g \in \Lone(\R^n)$, so is $(f \ast g)$ by the inequality above.
\end{proof}
The following result describes a deep connection between the Fourier Transform and the convolution operation.
\begin{theorem}[Fourier Transform of a convolution is a multiplication]
If $f, g \in \Lone(\R^n)$, then $\widehat{(f \ast g)}(\vec{u}) = \widehat{f}(\vec{u}) \cdot \widehat{g}(\vec{u})$. 
\end{theorem}
\begin{proof} Proof is by direct calculation and it is very similar to the argument above. See Proposition 16.4 in \cite{bass2011real}.
\end{proof}

Although its proof is elementary, the result above is very important in applications of the Fourier transform. For instance, in signal processing, an efficient implementation of signal filters relies on the multiplicative property of convolution guaranteed by the result above.
\subsection{Approximate identites}

We begin by introducing the idea of an approximate identity with respect to the convolution. That is a family of functions that act as an identity with respect to the convolution, under limiting conditions. We will use the approximate identities to establish the \nameref{thm:fourier:inversionlone}. The following lemma justifies the name.

\begin{lemma}[Approximate Identity Lemma]
\label{lemma:fourier:approxident}
Let $\varphi \in \Lone(\R^n)$ and suppose that $\int_{\R^n} \varphi(\vec{x}) d\vec{x} = 1$. For $\delta > 0$, define $\varphi_\delta : \R^n \to \mathbb{C}$ by $\varphi_\delta (\vec{x}) = \frac{1}{\delta^n} \varphi(\frac{\vec{x}}{\delta})$. 
\begin{enumerate}[noitemsep]
\item If $g$ is continuous with compact support, then $g \ast \varphi_\delta \to g$ pointwise as $\delta \to 0$.
\item If $g$ is continuous with compact support, then $g \ast \varphi_\delta \to g$ in $\norm{\cdot}_1$ as $\delta \to 0$.
\item If $f \in \Lone(\R^n)$ then $f \ast \varphi_\delta \to f$ in  $\norm{\cdot}_1$ as $\delta \to 0$.
\end{enumerate}
\end{lemma}
\begin{proof}
By change of variables $\vec{x} = \frac{1}{\delta} \vec{y}$,
\begin{align*}
    \int_{\R^n} \varphi_{\delta} (\vec{y}) \,d\vec{y} &= \int_{\R^n} \frac{1}{\delta^n} \varphi(\vec{y}) \,d\vec{y}= \int_{\R^n} \varphi(\vec{x}) \,d\vec{x} = 1, & \\ 
     \norm{\varphi_\delta}_1 &= \int_{\R^n} |\varphi_{\delta} (\vec{y})| \,d\vec{y} =  \int_{\R^n} |\varphi (\vec{t})| \,d\vec{t} = \norm{\varphi}_1.
\end{align*}
\setcounter{step}{0}
\begin{step}[Proof of $1$]
    Let $\vec{x} \in \R^n$. Then \begin{align*}
        | (g \ast \varphi_\delta)(\vec{x}) - g(\vec{x}) | &=  \left | \int_{\R^n} (g (\vec{x} - \vec{y}) - g(\vec{x})) \varphi_\delta (\vec{y}) \,d\vec{y} \right | & \text{since $ \int_{\R^n} \varphi_{\delta} (\vec{y}) = 1$} & \\
        &\leq  \left |  \int_{\R^n} (g (\vec{x} - \delta \vec{t}) - g(\vec{x})) \varphi (\vec{t}) \,d\vec{t}  \right |& \text{change of variables $\vec{t} = \frac{\vec{y}}{\delta} $} & \\
        &\leq \int_{\R^n} \left | g (\vec{x} - \delta \vec{t}) - g(\vec{x}) \right | |\varphi(\vec{t})| \,d\vec{t}.
    \end{align*}
Since $g$ is continuous and compactly supported, $|g|$ is bounded above by $\norm{g}_\infty$ which is finite. Thus $\left | g (\vec{x} - \delta \vec{t}) - g(\vec{x}) \right | |\varphi| \leq 2 \norm{g}_\infty |\varphi|$. Since $\varphi \in \Lone(\R^n)$, $2 \norm{g}_\infty |\varphi| \in \Lone(\R^n)$. By \nameref{thm:dct} and continuity of $g$, \begin{align*}
    0 \leq \lim_{\delta \to 0}| (g \ast \varphi_\delta)(\vec{x}) - g(\vec{x}) | \leq \int_{\R^n}  \lim_{\delta \to 0} \left | g (\vec{x} - \delta \vec{t}) - g(\vec{x}) \right | |\varphi(\vec{t})| \,d\vec{t} = 0.
\end{align*}
\end{step}

\begin{step}[Proof of $2$]
We proceed by direct calculation, 
\begin{align*}
    \int_{\R^n} | (g \ast \varphi_\delta) (\vec{x}) - g(\vec{x}) | \, d\vec{x}&= \int_{\R^n} \left |     \int_{\R^n} (g (\vec{x} - \vec{y}) - g(\vec{x})) \varphi_\delta(\vec{y}) \,d\vec{y}  \right | \, d\vec{x} & \\
    &= \int_{\R^n} \left | \int_{\R^n} (g (\vec{x} -\delta \vec{t}) - g(\vec{x})) \varphi(\vec{t}) \,d\vec{t}  \right | \, d\vec{x}& \\
    &\leq \int_{\R^n} \int_{\R^n} | g (\vec{x} - \delta \vec{t}) - g(\vec{x})| |\varphi(\vec{t}) | \, d\vec{x}\,d\vec{t} & \\ 
    &\leq \int_{\R^n} |\varphi(\vec{t})| \left ( \int_{\R^n} |g (\vec{x} - \delta \vec{t}) - g(\vec{x})| \, d\vec{x}\right)d\vec{t}.
\end{align*}
Consider $G_\delta (\vec{t}) =  \int_{\R^n} |g (\vec{x} - \delta \vec{t}) - g(\vec{x})| \, d\vec{x}$. Now, $|g (\vec{x} - \delta \vec{t}) - g(\vec{x})| \leq |g(\vec{x} - \delta \vec{t})| + |g(\vec{x})| $.
Now $\int_{\R^n} (|g(\vec{x} - \delta \vec{t})| + |g(\vec{x})|)d\vec{x}= 2 \norm{g}_1$.
Since $g$ is compactly supported and $\lambda$ is regular, $\norm{g}_1 < \infty$. By \nameref{thm:dct}, $\lim_{\delta \to 0} G_\delta(\vec{t}) = \int_{\R^n} \lim_{\delta \to 0} |g (\vec{x} - \delta \vec{t}) - g(\vec{x})| \, d\vec{x}= 0$. Since for every $\vec{t} \in \R^n, | G_\delta (\vec{t}) | | \varphi(\vec{t})| \leq 2 \norm{g}_1 | \varphi(\vec{t})|$ and $\int_{\R^n} \varphi(\vec{x}) d\vec{x}= 1$, by \nameref{thm:dct}, \begin{align*}
    0 \leq \lim_{\delta \to 0} \norm{ (g \ast \varphi_\delta) - g}_1 \leq  \int_{\R^n} |\varphi(\vec{t})| \lim_{\delta \to 0}  G_\delta(\vec{t}) \,d\vec{t} = 0.
\end{align*}
Thus, $ \lim_{\delta \to 0} \norm{ (g \ast \varphi_\delta) - g}_1 = 0$.
\end{step}
\begin{step}[Proof of $3$]
    Let $\epsilon > 0$. By, \nameref{thm:lp:densitycc}, there exists compactly supported $g \in \Lone(\R^n)$ such that $\norm{f - g}_1 < \epsilon$. Set $h = f - g$. Then $\norm{h}_1 < \epsilon$. By \nameref{ineqn:lp:minkowski}, 
    \begin{align}
        \label{lemma:fourier:approxident:ineqn1}
        \norm{(f \ast \varphi_\delta) - f }_1 = \norm{(h + g) \ast \varphi_\delta - (h + g)}_1 \leq  \norm{(g \ast \varphi_\delta) - g}_1 + \norm{(h \ast \varphi_\delta) - h}_1.
    \end{align}
    By \nameref{prop:fourier:convnorm}, $\norm{h \ast \varphi_\delta}_1 \leq \norm{h}_1 \norm{\varphi_\delta}_1$ and $\norm{\varphi_\delta}_1 = \norm{\varphi}_1$ so  \begin{align}
        \label{lemma:fourier:approxident:ineqn2}
         \norm{(h \ast \varphi_\delta) - h}_1 \leq \norm{h \ast \varphi_\delta}_1 + \norm{h}_1 \leq  \norm{h}_1 \norm{\varphi_\delta}_1 + \norm{h}_1 < \epsilon  (1 + \norm{\varphi}_1)
    \end{align}
    By Step 2, $\limsup_{\delta \to 0} \norm{(g \ast \varphi_\delta) - g}_1 = 0$. Combining \ref{lemma:fourier:approxident:ineqn1} and \ref{lemma:fourier:approxident:ineqn2} gives \begin{align*}
        \limsup_{\delta \to 0} \norm{(f \ast \varphi_\delta) - f }_1  \leq \epsilon  (1 + \norm{\varphi}_1).
    \end{align*}
    Since $\epsilon$ was arbitrary, $\limsup_{\delta \to 0} \norm{(f \ast \varphi_\delta) - f }_1 = 0$. Thus, $f \ast \varphi_\delta \to f$ in $\norm{\cdot}_1$.
\end{step}
\end{proof}
\subsection{Gaussians and their Fourier Transforms}
An example of a family of functions satisfying conditions of the \nameref{lemma:fourier:approxident} is the family of Gaussians, parameterized by the variance. In this section, we will explore their Fourier Transforms. We will use Gaussians to prove \nameref{thm:fourier:inversionlone}. We start with the simplest Gaussians.
\begin{proposition}[Fourier Transforms of Gaussians]
\label{prop:fourier:gaussian:ft}
Suppose $f_1 : \R \to \R$ and $f_n : \R^n \to \R$ are given by 
\begin{align*}
        f_1 (x) = \frac{1}{\sqrt{2 \pi}} e^{-\frac{x^2}{2}}, \text{ }  f_n (\vec{x}) = \frac{1}{\left ( 2 \pi \right )^\frac{n}{2}} e^{-\frac{\norm{\vec{x}}^2}{2}}.
\end{align*}
Then  $\widehat{f_1}(u) = e^{-\frac{u^2}{2}}$ and  $\widehat{f_n}(\vec{u}) = e^{-\frac{\norm{\vec{u}}^2}{2}}$.
\end{proposition}
\begin{proof-idea*}
There are several ways to prove this proposition. A common approach is to use contour integration with a Residue Theorem.
However, it is possible to give a more elementary argument, such as the following based on page 107 in \cite{jacod_2004_probability} . 
\end{proof-idea*}
\begin{proof}
We begin by proving the claim for $f_1$.
\setcounter{step}{0}
\begin{step}[Proof for $f_1$]
\label{thm:fourier:gaussian:step:1}
We begin with a direct calculation, \begin{align*}
    \widehat{f_1}(u) &=  \frac{1}{\sqrt{2 \pi}} \int_\R e^{iux} e^{-\frac{x^2}{2}} \, dx & \\ 
                     &=  \frac{1}{\sqrt{2 \pi}} \int_\R \cos(ux) e^{-\frac{x^2}{2}} \, dx + i \left (  \frac{1}{\sqrt{2 \pi}} \int_\R \sin(ux) e^{-\frac{x^2}{2}} \, dx  \right) & \\ 
                     &=  \frac{1}{\sqrt{2 \pi}} \int_\R \cos(ux) e^{-\frac{x^2}{2}} \, dx.
\end{align*}
The imaginary part vanishes because  $x \to \sin(ux) e^{-\frac{x^2}{2}}$ is odd.
Since $|\cos(ux) e^{-\frac{x^2}{2}}| \leq e^{-\frac{x^2}{2}}$ and $\int_\R e^{-\frac{x^2}{2}} \, dx = \sqrt{2 \pi}$, by \nameref{thm:dct}, \begin{align}
    \label{eqn:fourier:gaussian1d:derivative}
    \frac{\partial \widehat{f_1}}{\partial u} (u) = \frac{1}{\sqrt{2 \pi}}  \int_\R \frac{\partial}{\partial u} \left(\cos(ux)\right)  e^{-\frac{x^2}{2}}\, dx = -\frac{1}{\sqrt{2\pi}} \int_\R \sin(ux) x  e^{-\frac{x^2}{2}}\, dx .
\end{align}
Integration by parts gives \begin{align}
    \label{eqn:fourier:gaussian1d:integral}
    \int_{\R} \sin(ux) x  e^{-\frac{x^2}{2}}  \, dx = - u\sin{(ux)} x e^{-\frac{x^2}{2}} \Big|_{x = -\infty}^{x = \infty} + u \int_\R \cos(ux) e^{-\frac{x^2}{2}} \, dx.
\end{align}
Since $|- u\sin{(ux)} x e^{-\frac{x^2}{2}} | \leq |u| |x| e^{-\frac{x^2}{2}}$ and $\lim_{x \to \pm \infty}  |u||x| e^{-\frac{x^2}{2}} = 0$, $- u\sin{(ux)} x e^{-\frac{x^2}{2}} \Big|_{x = -\infty}^{x = \infty}$ vanishes. Applying this observation to \ref{eqn:fourier:gaussian1d:integral} and substituting the result in \ref{eqn:fourier:gaussian1d:derivative} gives the initial value problem 
\begin{align*}
    \begin{cases}
        \frac{\partial \widehat{f_1}}{\partial u} (u) = - u \widehat{f_1}(u) \text{ subject to } \widehat{f_1}(0) = 1. \\
    \end{cases}
\end{align*}
Observe that the differential equation implies $ \frac{\partial}{\partial u} \ln{|\widehat{f_1}(u)|} = -u$. By Fundamental Theorem of Calculus, $\ln | \widehat{f_1}(u) | = -\frac{1}{2}u^2 + C$ for $C \in \R$. Exponentiating both sides gives $ \widehat{f_1}(u) = e^{C} \cdot e^{-\frac{1}{2}u^2 }$. Setting the initial value condition results in $C = 0$. Hence $\widehat{f_1}(u) = e^{-\frac{1}{2}u^2 }$.
\end{step}
\begin{step}[Proof for $f_n$] We proceed with a direct calculation,
    \begin{align*}
        \widehat{f_n}(\vec{u}) &= \frac{1}{\left ( 2 \pi \right )^\frac{n}{2}}  \int_{\R^n}  e^{i \langle \vec{u}, \vec{x} \rangle } e^{-\frac{\norm{\vec{x}}^2}{2}} \, d\vec{x}
                             = \frac{1}{\left ( 2 \pi \right )^\frac{n}{2}}  \int_{\R^n}  e^{i \left ( \sum_{k=1}^n u_k x_k \right) } e^{-\frac{ \sum_{k=1}^n  x_k^2 }{2}} \, d\vec{x}  & \\
                             &=  \int_{\R^n}  \prod_{k=1}^n \frac{1}{\sqrt{2 \pi}} e^{i u_k x_k} e^{-\frac{1}{2} x_k^2} d\vec{x} &\\ 
                             &= \int_{\R} \cdots \int_{\R}  \prod_{k=1}^n \frac{1}{\sqrt{2 \pi}} e^{i u_k x_k} e^{-\frac{1}{2} x_k^2} \, d x_1 \ldots  \, d x_n  \text{ by \nameref{thm:measure:fubini}}  \\ 
                             &= \prod_{k=1}^n \widehat{f_1}(u_k) = \prod_{k=1}^n e^{-\frac{1}{2} u_k^2} = e^{-\frac{1}{2} \norm{\vec{u}}^2}. 
    \end{align*}
\end{step}
\end{proof}
\pagebreak
Using \nameref{prop:fourier:basics} and \nameref{prop:fourier:gaussian:ft}, we can calculate the Fourier Transform of Gaussians which will play essential role in the proof of \nameref{thm:fourier:inversionlone}.
\begin{corollary}
\label{cor:fourier:gaussian}
Let $a \neq 0$ and suppose that $h_a : \R^n \to \R$ is given by \begin{align*}
    h_a (\vec{x}) = \frac{1}{(2 \pi)^n} e^{-\frac{1}{2a^2} \norm{x}^2}.
\end{align*}
Then $\widehat{h_a} (\vec{u}) = (2 \pi)^{-\frac{n}{2}} a^{n} e^{-\frac{a^2 \norm{\vec{u}}^2}{2} }$.
\end{corollary}
\begin{proof}
Write $h_a (\vec{x}) = (2 \pi)^{-\frac{n}{2}} (2 \pi)^{-\frac{n}{2}}  e^{-\frac{1}{2a^2} \norm{x}^2}$. Then $h_a(\vec{x}) =(2 \pi)^{-\frac{n}{2}}  f_n(\frac{1}{a} \vec{x})$.
By Proposition \ref{prop:fourier:basics}, $\widehat{h_a}(\vec{u}) = (2 \pi)^{-\frac{n}{2}}  a^n \widehat{f_n}(a \vec{u})$. By Proposition \ref{prop:fourier:gaussian:ft}, $\widehat{f_n}(a \vec{u}) = e^{-\frac{ a^2 \norm{\vec{u}}^2}{2}}$ so $\widehat{h_a} (\vec{u}) = (2 \pi)^{-\frac{n}{2}} a^{n} e^{-\frac{a^2 \norm{\vec{u}}^2}{2} }$.
\end{proof}
\subsection{Fourier inversion theorem on $\Lone(\R^n)$}

We are ready to state and prove the \nameref{thm:fourier:inversionlone}.

\begin{theorem}[Fourier inversion theorem for $\Lone(\R^n)$]
\label{thm:fourier:inversionlone}
Suppose that $f$ and $\widehat{f}$ are both in $\Lone(\R^n)$. Then \begin{align*}
    f(\vec{y}) = \frac{1}{(2\pi)^n}\int_{\R^n} e^{-i \langle \vec{u}, \vec{y}\rangle} \widehat{f}(\vec{u}) \,d\vec{u} \text{, almost everywhere}.
\end{align*}
\end{theorem}

\begin{proof}
For $a \neq 0$ define $h_a : \R^n \to \R$ by \begin{align*}
    h_a (\vec{x}) = \frac{1}{(2 \pi)^n} e^{-\frac{1}{2a^2} \norm{x}^2}.
\end{align*}
By Corollary \ref{cor:fourier:gaussian}, $\widehat{h_a} (\vec{u}) = (2 \pi)^{-\frac{n}{2}} a^{n} e^{-\frac{a^2 \norm{\vec{u}}^2}{2} }$. For every $\vec{y} \in \R^n$ and $a \neq 0$, \begin{align*}
 \int_{\R^n}  e^{-i \langle \vec{u}, \vec{y}\rangle} \widehat{f}(\vec{u})  h_a(\vec{u}) \,d\vec{u} &=  \int_{\R^n} \left( \int_{\R^n}   e^{i \langle \vec{u}, \vec{x} - \vec{y} \rangle} f(\vec{x}) h_a(\vec{u})  \,d\vec{x} \right) \,d\vec{u}  & \\
                      &= \int_{\R^n} \left( \int_{\R^n}  e^{i \langle \vec{u}, \vec{x} - \vec{y} \rangle} f(\vec{x}) h_a(\vec{u})  \,d\vec{u} \right)d\vec{x} & \\
                      &= \int_{\R^n} f(\vec{x}) \left( \int_{\R^n}   e^{i \langle \vec{u}, \vec{x} - \vec{y} \rangle} h_a(\vec{u})  \,d\vec{u} \right)d\vec{x} & \\
                      &= \int_{\R^n} f(\vec{x}) \left( \int_{\R^n}   e^{i \langle \vec{x} - \vec{y} , \vec{u} \rangle} h_a(\vec{u})  \,d\vec{u} \right)d\vec{x}  & \\
                      &= \int_{\R^n} f(\vec{x}) \widehat{h_a}(\vec{x} - \vec{y}) \,d\vec{x} & \\
                      &= \int_{\R^n} f(\vec{y} - \vec{t}) \widehat{h_a}(- \vec{t}) \,d\vec{t} &\\ 
                      &=\int_{\R^n} f(\vec{y} - \vec{t}) \widehat{h_a}(\vec{t}) \,d\vec{t} = (f \ast h_a)(\vec{y}).
\end{align*}

The interchange of integrals is justified by \nameref{thm:measure:fubini}, since \[\int_{\R^n}\int_{\R^n} |e^{i \langle \vec{u}, \vec{x} - \vec{y} \rangle}  | |f (\vec{x}) | |h_a (\vec{u})|  \,d\vec{x}  \,d\vec{u} \leq \norm{f}_1 \norm{h_a}_1 < \infty.\]
We performed the change of variables with $\vec{t} = \vec{y} - \vec{x}$ and applied the symmetry of $\langle \cdot, \cdot \rangle$ and $\widehat{h_a}$. Observe that $|e^{-i \langle \vec{u}, \vec{y} \rangle} \widehat{f}(\vec{u}) h_a(\vec{u})| \leq |\widehat{f}(u)||h_a(\vec{u})| \leq \frac{1}{(2 \pi)^n} |\widehat{f}(u)|$. Since $\widehat{f}$ in $\Lone(\R^n)$, $ \frac{1}{(2 \pi)^n} |\widehat{f}| \in \Lone(\R^n)$. Since for $\vec{u} \in \R^n$, $\lim_{a \to \infty} h_a (\vec{u}) = 1$, by \nameref{thm:dct}, \begin{equation}
    \label{eqn:fourier:inv:leftlim}
    \lim_{a \to \infty} \int_{\R^n}  e^{-i \langle \vec{u}, \vec{y}\rangle} \widehat{f}(\vec{u})  h_a(\vec{u}) \,d\vec{u}  =  \int_{\R^n}  e^{-i \langle \vec{u}, \vec{y}\rangle} \widehat{f}(\vec{u})   \,d\vec{u}.
\end{equation}
Applying \nameref{lemma:fourier:approxident} with $\delta = \frac{1}{\alpha}$ gives $(f \ast h_a) \to f$ in $\Lone(\R^n)$ as $a \to \infty$. Let $\{ a_n \}_{n =1}^\infty$ be any sequence of strictly increasing real numbers. Then $(f \ast h_{a_n}) \to f$ in $\Lone(\R^n)$ as $n \to \infty$. By \nameref{proposition:measure:convergence:lone_implies_ae}, $(f \ast h_{a_n}) \to f$ in measure. By \nameref{proposition:measure:convergence:mu_implies_subseqn_ae}, there exists a subsequence $\{(f \ast h_ {a_{n_k}})\}_{k=1}^\infty$ such that $(f \ast h_ {a_{n_k}}) \to f$ pointwise almost everywhere as $k \to \infty$. By \ref{eqn:fourier:inv:leftlim}, $(f \ast h_ {a_{n_k}})(\vec{y}) = \int_{\R^n}  e^{-i \langle \vec{u}, \vec{y}\rangle} \widehat{f}(\vec{u})  h_{a_{n_k}}(\vec{u}) \,d\vec{u} \to  \int_{\R^n}  e^{-i \langle \vec{u}, \vec{y}\rangle} \widehat{f}(\vec{u}) \,d\vec{u}$ as $k \to \infty$. By uniqueness of the limit, $f(\vec{y}) = \int_{\R^n}  e^{-i \langle \vec{u}, \vec{y}\rangle} \widehat{f}(\vec{u}) \,d\vec{u}$ almost everywhere, as claimed.
\end{proof}
\subsection{Fourier Transform of a measure}


\begin{definition}
Let $\mu$ be a finite signed measure on $\R^n$. We define the Fourier Transform of $\mu$ by
\begin{align*}
    \widehat{\mu}(\vec{u}) = \int_{\R^n} e^{i \langle \vec{u}, \vec{x} \rangle } \, d\mu(\vec{x}).
\end{align*}
\end{definition}
\begin{remark}
In probability theory, $\widehat{\mu}$ is known as a characteristic function. The following result, which is a nontrivial consequence of \nameref{prop:fourier:gaussian:ft} and \nameref{thm:anal:stone-weierstrass}, justifies that name.
\end{remark}
\begin{theorem}
\label{thm:fourier:uniqmeasure}
Let $\mu$ and $\nu$ be finite signed regular measures on $\B(\R^n)$. If for every $\vec{u} \in \R^n$, $\widehat{\mu}(\vec{u}) = \widehat{\nu}(\vec{u})$, then $\mu = \nu$.
\end{theorem}
\begin{proof}
Omitted. See Exercise 16.6 in \cite{bass2011real}. The proof for probability measures is Theorem 14.1 in \cite{jacod_2004_probability}.
\end{proof}
\begin{corollary}
\label{corr:fourier:uniqmeasure}
Let $\mu$ be a finite signed regular measure on $\B(\R^n)$. If for every $\vec{u} \in \R^n$, $\widehat{\mu}(\vec{u}) = 0$, then $\mu = 0$.
\end{corollary}

\begin{proof}
Define $\nu(A) = 0$ for every $A \in \B(\R^n)$. Clearly, $\widehat{\nu} = 0$. Since $\widehat{\mu} = \widehat{\nu} = 0$, by Theorem \ref{thm:fourier:uniqmeasure}, $\mu = \nu = 0$.
\end{proof}
