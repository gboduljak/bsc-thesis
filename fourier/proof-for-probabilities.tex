
\subsection{Measure Theory Construction Theorems}

\begin{theorem}[Monotone Class Theorem]

\end{theorem}

\subsection{Stone-Weierstrass Theorem for locally compact spaces}

\begin{definition}[vanishing at infinity]
Let $X$ be a locally compact topological space. A continuous function $f : X \to \R$ vanishes at infinity if for every $\epsilon > 0$, there exists a compact set $K \subset X$ such that $|f| < \epsilon$ on $X \setminus K$. We denote the algebra of real-valued continuous functions on $X$ that vanish at infinity by $\mathcal{C}_0(X)$.
\end{definition}

\begin{remark}
It is still possible to equip $\mathcal{C}_0(X)$ with the supremum norm and consequently with a $d_\infty$ metric.
Hence the notion of uniform density still applies.
\end{remark}
\begin{definition}[vanishing nowhere]
Let $\mathcal{A}$ be a subalgebra of $\mathcal{C}_0(X)$. We say $\mathcal{A}$ vanishes nowhere if not all elements of $\mathcal{A}$ vanish simultaneously at a point. In other words, for every $x \in X$, there exists $f \in \mathcal{A}$ such that $f(x) \neq 0$.
\end{definition}

\begin{theorem}[Stone-Weierstrass Theorem for locally compact Hausdorff spaces]
\label{thm:top:stone-weierstrass:lch}
Suppose that $X$ is a locally compact Hausdorff space and $\mathcal{A}$ is a subalgebra of $\mathcal{C}_0(X)$. If $\mathcal{A}$ separates points of $X$ and vanishes nowhere, then $\mathcal{A}$ is uniformly dense in $\mathcal{C}_0(X)$.
\end{theorem}

We will apply  \nameref{thm:top:stone-weierstrass:lch} in the special case of $\R^n$. Thus, it is useful to introduce the following corollary.

\begin{corollary}[Stone-Weierstrass Theorem for $\mathcal{C}_0(\R^n)$]
\label{thm:cor:stone-weierstrass:lchrn}
Suppose that $X$ is a locally compact Hausdorff space and $\mathcal{A}$ is a subalgebra of $\mathcal{C}_0(X)$. If $\mathcal{A}$ separates points of $X$ and vanishes nowhere, then $\mathcal{A}$ is uniformly dense in $\mathcal{C}_0(X)$.
\end{corollary}
\begin{proof} Since $\R^n$ is a metric space, $\R^n$ is Hausdorff. Clearly, every point of $\R^n$ belongs to a closed ball of radius 1 centred at that point. By Heine-Borel Theorem, that ball is compact. Therefore $\R^n$ is locally compact. The claim follows directly from \nameref{thm:top:stone-weierstrass:lch}.
\end{proof}